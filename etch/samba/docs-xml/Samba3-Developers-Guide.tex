% ------------------------------------------------------------	
% Autogenerated LaTeX file for books	
% ------------------------------------------------------------	
\ifx\pdfoutput\undefined
\documentclass[,letterpaper,11pt,openright,twoside,]{sambadoc}
\else
\documentclass[pdftex,,letterpaper,11pt,openright,twoside,]{sambadoc}
\fi
\label{Samba-Developers-Guide}\usepackage{ifthen}
% --------------------------------------------
% Check for PDFLaTeX/LaTeX 
% --------------------------------------------
\newif\ifpdf
\ifx\pdfoutput\undefined
\pdffalse % we are not running PDFLaTeX
\else
\pdfoutput=1 % we are running PDFLaTeX
\pdftrue
\fi
% --------------------------------------------
% Load graphicx package with pdf if needed 
% --------------------------------------------
\ifpdf
\usepackage[pdftex]{graphicx}
\pdfcompresslevel=9
\else
\usepackage{graphicx}
\fi

\makeatletter
% redefine the listoffigures and listoftables so that the name of the chapter
% is printed whenever there are figures or tables from that chapter. encourage
% pagebreak prior to the name of the chapter (discourage orphans).
\let\save@@chapter\@chapter
\let\save@@l@figure\l@figure
\let\the@l@figure@leader\relax
\def\@chapter[#1]#2{\save@@chapter[{#1}]{#2}%
\addtocontents{lof}{\protect\def\the@l@figure@leader{\protect\pagebreak[0]\protect\contentsline{chapter}{\protect\numberline{\thechapter}#1}{}{\thepage}}}%
\addtocontents{lot}{\protect\def\the@l@figure@leader{\protect\pagebreak[0]\protect\contentsline{chapter}{\protect\numberline{\thechapter}#1}{}{\thepage}}}%
}
\renewcommand*\l@figure{\the@l@figure@leader\let\the@l@figure@leader\relax\save@@l@figure}
\let\l@table\l@figure
\makeatother
% ---------------------- 
% Most Common Packages   
% ---------------------- 
\usepackage{latexsym}         
\usepackage{enumerate}         
\usepackage{fancybox}      
\usepackage{float}       
\usepackage{ragged2e}       
\usepackage{fancyvrb}         
\makeatletter\@namedef{FV@fontfamily@default}{\def\FV@FontScanPrep{}\def\FV@FontFamily{}}\makeatother
\fvset{obeytabs=true,tabsize=3}
\usepackage{parskip}         
\usepackage{rotating}         
\usepackage{subfigure}         
\usepackage{tabularx}         
\usepackage{url}         
% --------------------------------------------
% Math support                                
% --------------------------------------------
\usepackage{amsmath,amsthm, amsfonts, amssymb, amsxtra,amsopn}
%\newtheorem{thm}{Theorem}[section]
%\newtheorem{cor}[section]{Corollary}
%\newtheorem{lem}[section]{Lemma}
%\newtheorem{defn}[section]{Definition}
%\newtheorem{prop}[section]{Proposition}
%\newtheorem{ax}{Axiom}
%\newtheorem{theorem}[section]{Theorem}
%\newtheorem{corollary}{Corollary}
%\newtheorem{lemma}{Lemma}
%\newtheorem{proposition}{Proposition}
%\theoremstyle{definition}
%\newtheorem{definition}{Definition}
%\theoremstyle{remark}
%\newtheorem{rem}{Remark}
%\newtheorem*{notation}{Notation}
%\newcommand{\ntt}{\normalfont\ttfamily}
%\newcommand{\thmref}[1]{Theorem~\ref{#1}}
%\newcommand{\secref}[1]{\S\ref{#1}}
%\newcommand{\lemref}[1]{Lemma~\ref{#1}}
 \newcommand{\bysame}{\mbox{\rule{3em}{.4pt}}\,}
 \newcommand{\A}{\mathcal{A}}
 \newcommand{\B}{\mathcal{B}}
 \newcommand{\XcY}{{(X,Y)}}
 \newcommand{\SX}{{S_X}}
 \newcommand{\SY}{{S_Y}}
 \newcommand{\SXY}{{S_{X,Y}}}
 \newcommand{\SXgYy}{{S_{X|Y}(y)}}
 \newcommand{\Cw}[1]{{\hat C_#1(X|Y)}}
 \newcommand{\G}{{G(X|Y)}}
 \newcommand{\PY}{{P_{\mathcal{Y}}}}
 \newcommand{\X}{\mathcal{X}}
 \newcommand{\wt}{\widetilde}
 \newcommand{\wh}{\widehat}
 % --------------------------------------------
 %\DeclareMathOperator{\per}{per}
 \DeclareMathOperator{\cov}{cov}
 \DeclareMathOperator{\non}{non}
 \DeclareMathOperator{\cf}{cf}
 \DeclareMathOperator{\add}{add}
 \DeclareMathOperator{\Cham}{Cham}
 \DeclareMathOperator{\IM}{Im}
 \DeclareMathOperator{\esssup}{ess\,sup}
 \DeclareMathOperator{\meas}{meas}
 \DeclareMathOperator{\seg}{seg}
% --------------------------------------------
% --------------------------------------------
% Load hyperref package with pdf if needed 
% --------------------------------------------
\ifpdf
\usepackage[pdftex,bookmarksnumbered,colorlinks,backref,bookmarks,breaklinks,linktocpage,plainpages=false,hyperfigures,hyperindex,citecolor=black,urlcolor=black,filecolor=black,linkcolor=black,menucolor=red,pagecolor=black]{hyperref}
\else
\usepackage[bookmarksnumbered,colorlinks,backref,bookmarks,breaklinks,linktocpage,plainpages=false,]{hyperref}
\fi
% --------------------------------------------
% ----------------------------------------------
% Define a new LaTeX environment (adminipage)
% ----------------------------------------------
\newenvironment{admminipage}%
{ % this code corresponds to the \begin{adminipage} command
 \begin{Sbox}%
 \begin{minipage}%
} %done
{ % this code corresponds to the \end{adminipage} command
 \end{minipage}
 \end{Sbox}
 \fbox{\TheSbox}
} %done
% ----------------------------------------------
% Define a new LaTeX length (admlength)
% ----------------------------------------------
\newlength{\admlength}
% ----------------------------------------------
% Define a new LaTeX environment (admonition)
% With 2 parameters:
% #1 The file (e.g. note.pdf)
% #2 The caption
% ----------------------------------------------
\newenvironment{admonition}[2] 
{ % this code corresponds to the \begin{admonition} command
 \hspace{0mm}\newline\hspace*\fill\newline
 \noindent
 \setlength{\fboxsep}{5pt}
 \setlength{\admlength}{\linewidth}
 \addtolength{\admlength}{-10\fboxsep}
 \addtolength{\admlength}{-10\fboxrule}
 \admminipage{\admlength}
 {\bfseries \sc\large{#2}} \newline
 \\[1mm]
 \sffamily
 \includegraphics[width=1cm]{#1}
 \addtolength{\admlength}{-1cm}
 \addtolength{\admlength}{-20pt}
 \begin{minipage}[lt]{\admlength}
 \parskip=0.5\baselineskip \advance\parskip by 0pt plus 2pt
} %done
{ % this code corresponds to the \end{admonition} command
 \vspace{5mm} 
 \end{minipage}
 \endadmminipage
 \vspace{.5em}
 \par
}
% --------------------------------------------
% Commands to manage/style/create floats      
% figures, tables, algorithms, examples, eqn  
% --------------------------------------------
 \floatstyle{ruled}
 \restylefloat{figure}
 \floatstyle{ruled}
 \restylefloat{table}
 \floatstyle{ruled}
 \newfloat{program}{ht}{lop}[section]
 \floatstyle{ruled}
 \newfloat{example}{ht}{loe}[section]
 \floatname{example}{Example}
 \floatstyle{ruled}
 \newfloat{dbequation}{ht}{loe}[section]
 \makeatletter\def\toclevel@dbequation{0}\makeatother
 \floatname{dbequation}{Equation}
 \floatstyle{boxed}
 \newfloat{algorithm}{ht}{loa}[section]
 \floatname{algorithm}{Algorithm}
\ifpdf
\DeclareGraphicsExtensions{.pdf,.png,.jpg}
\else
\DeclareGraphicsExtensions{.eps}
\fi
% --------------------------------------------
% $latex.caption.swapskip enabled for $formal.title.placement support
\newlength{\docbooktolatextempskip}
\newcommand{\captionswapskip}{\setlength{\docbooktolatextempskip}{\abovecaptionskip}\setlength{\abovecaptionskip}{\belowcaptionskip}\setlength{\belowcaptionskip}{\docbooktolatextempskip}}
% --------------------------------------------
% Better linebreaks
\newcommand{\docbookhyphenatedot}[1]{{\hyphenchar\font=`\.\relax #1\hyphenchar\font=`\-}}
\newcommand{\docbookhyphenatefilename}[1]{{\hyphenchar\font=`\.\relax #1\hyphenchar\font=`\-}}
\newcommand{\docbookhyphenateurl}[1]{{\hyphenchar\font=`\/\relax #1\hyphenchar\font=`\-}}
\usepackage[english]{babel} 
% Guard against a problem with old package versions.
\makeatletter
\AtBeginDocument{
\DeclareRobustCommand\ref{\@refstar}
\DeclareRobustCommand\pageref{\@pagerefstar}
}
\makeatother
% --------------------------------------------
\makeatletter
\newcommand{\dbz}{\penalty \z@}
\newcommand{\docbooktolatexpipe}{\ensuremath{|}\dbz}
\newskip\docbooktolatexoldparskip
\newcommand{\docbooktolatexnoparskip}{\docbooktolatexoldparskip=\parskip\parskip=0pt plus 1pt}
\newcommand{\docbooktolatexrestoreparskip}{\parskip=\docbooktolatexoldparskip}
\def\cleardoublepage{\clearpage\if@twoside \ifodd\c@page\else\hbox{}\thispagestyle{empty}\newpage\if@twocolumn\hbox{}\newpage\fi\fi\fi}
\usepackage[]{ucs}
\usepackage[utf8]{inputenc}

\ifx\dblatex@chaptersmark\@undefined\def\dblatex@chaptersmark#1{\markboth{\MakeUppercase{#1}}{}}\fi
\let\save@makeschapterhead\@makeschapterhead
\def\dblatex@makeschapterhead#1{\vspace*{-80pt}\save@makeschapterhead{#1}}
\def\@makeschapterhead#1{\dblatex@makeschapterhead{#1}\dblatex@chaptersmark{#1}}

			
\AtBeginDocument{\ifx\refname\@undefined\let\docbooktolatexbibname\bibname\def\docbooktolatexbibnamex{\bibname}\else\let\docbooktolatexbibname\refname\def\docbooktolatexbibnamex{\refname}\fi}
% Facilitate use of \cite with \label
\newcommand{\docbooktolatexbibaux}[2]{%
  \protected@write\@auxout{}{\string\global\string\@namedef{docbooktolatexcite@#1}{#2}}
}
% Provide support for bibliography `subsection' environments with titles
\newenvironment{docbooktolatexbibliography}[3]{
   \begingroup
   \let\save@@chapter\chapter
   \let\save@@section\section
   \let\save@@@mkboth\@mkboth
   \let\save@@bibname\bibname
   \let\save@@refname\refname
   \let\@mkboth\@gobbletwo
   \def\@tempa{#3}
   \def\@tempb{}
   \ifx\@tempa\@tempb
      \let\chapter\@gobbletwo
      \let\section\@gobbletwo
      \let\bibname\relax
   \else
      \let\chapter#2
      \let\section#2
      \let\bibname\@tempa
   \fi
   \let\refname\bibname
   \begin{thebibliography}{#1}
}{
   \end{thebibliography}
   \let\chapter\save@@chapter
   \let\section\save@@section
   \let\@mkboth\save@@@mkboth
   \let\bibname\save@@bibname
   \let\refname\save@@refname
   \endgroup
}

		
			
%\usepackage{cite}
%\renewcommand\citeleft{(}  % parentheses around list
%\renewcommand\citeright{)} % parentheses around list
\newcommand{\docbooktolatexcite}[2]{%
  \@ifundefined{docbooktolatexcite@#1}%
  {\cite{#1}}%
  {\def\@docbooktolatextemp{#2}\ifx\@docbooktolatextemp\@empty%
   \cite{\@nameuse{docbooktolatexcite@#1}}%
   \else\cite[#2]{\@nameuse{docbooktolatexcite@#1}}%
   \fi%
  }%
}
\newcommand{\docbooktolatexbackcite}[1]{%
  \ifx\Hy@backout\@undefined\else%
    \@ifundefined{docbooktolatexcite@#1}{%
      % emit warning?
    }{%
      \ifBR@verbose%
        \PackageInfo{backref}{back cite \string`#1\string' as \string`\@nameuse{docbooktolatexcite@#1}\string'}%
      \fi%
      \Hy@backout{\@nameuse{docbooktolatexcite@#1}}%
    }%
  \fi%
}

		
			
% --------------------------------------------
% A way to honour <footnoteref>s
% Blame j-devenish (at) users.sourceforge.net
% In any other LaTeX context, this would probably go into a style file.
\newcommand{\docbooktolatexusefootnoteref}[1]{\@ifundefined{@fn@label@#1}%
  {\hbox{\@textsuperscript{\normalfont ?}}%
    \@latex@warning{Footnote label `#1' was not defined}}%
  {\@nameuse{@fn@label@#1}}}
\newcommand{\docbooktolatexmakefootnoteref}[1]{%
  \protected@write\@auxout{}%
    {\global\string\@namedef{@fn@label@#1}{\@makefnmark}}%
  \@namedef{@fn@label@#1}{\hbox{\@textsuperscript{\normalfont ?}}}%
  }

		
			
% index labeling helper
\newif\ifdocbooktolatexprintindex\docbooktolatexprintindextrue
\let\dbtolatex@@theindex\theindex
\let\dbtolatex@@endtheindex\endtheindex
\def\theindex{\relax}
\def\endtheindex{\relax}
\newenvironment{dbtolatexindex}[1]
   {
\if@openright\cleardoublepage\else\clearpage\fi
\let\dbtolatex@@indexname\indexname
\def\dbtolatex@indexlabel{%
 \ifnum \c@secnumdepth >\m@ne \refstepcounter{chapter}\fi%
 \label{#1}\hypertarget{#1}{\dbtolatex@@indexname}%
 \global\docbooktolatexprintindexfalse}
\def\indexname{\ifdocbooktolatexprintindex\dbtolatex@indexlabel\else\dbtolatex@@indexname\fi}
\dbtolatex@@theindex
   }
   {
\dbtolatex@@endtheindex\let\indexname\dbtolatex@@indexname
   }

\newlength\saveparskip \newlength\saveparindent
\newlength\tempparskip \newlength\tempparindent

		
\def\docbooktolatexgobble{\expandafter\@gobble}
% Prevent multiple openings of the same aux file
% (happens when backref is used with multiple bibliography environments)
\ifx\AfterBeginDocument\undefined\let\AfterBeginDocument\AtBeginDocument\fi
\AfterBeginDocument{
   \let\latex@@starttoc\@starttoc
   \def\@starttoc#1{%
      \@ifundefined{docbooktolatex@aux#1}{%
         \global\@namedef{docbooktolatex@aux#1}{}%
         \latex@@starttoc{#1}%
      }{}
   }
}
% --------------------------------------------
% Hacks for honouring row/entry/@align
% (\hspace not effective when in paragraph mode)
% Naming convention for these macros is:
% 'docbooktolatex' 'align' {alignment-type} {position-within-entry}
% where r = right, l = left, c = centre
\newcommand{\docbooktolatex@align}[2]{\protect\ifvmode#1\else\ifx\LT@@tabarray\@undefined#2\else#1\fi\fi}
\newcommand{\docbooktolatexalignll}{\docbooktolatex@align{\raggedright}{}}
\newcommand{\docbooktolatexalignlr}{\docbooktolatex@align{}{\hspace*\fill}}
\newcommand{\docbooktolatexaligncl}{\docbooktolatex@align{\centering}{\hfill}}
\newcommand{\docbooktolatexaligncr}{\docbooktolatex@align{}{\hspace*\fill}}
\newcommand{\docbooktolatexalignrl}{\protect\ifvmode\raggedleft\else\hfill\fi}
\newcommand{\docbooktolatexalignrr}{}
\ifx\captionswapskip\@undefined\newcommand{\captionswapskip}{}\fi
\makeatother
\title{\bfseries SAMBA Developers Guide}
\author{Jelmer R. Vernooij}
% --------------------------------------------
\makeindex
\makeglossary
% --------------------------------------------

\setcounter{tocdepth}{4}

\setcounter{secnumdepth}{4}
\begin{document}
{\maketitle\pagestyle{plain}
\thispagestyle{empty}}

% --------------------------------------------
% Abstract 
% --------------------------------------------
\begin{abstract}

{\em{Last Update}} : Fri Oct 10 00:59:58 CEST 2003

This book is a collection of documents that might be useful for people developing samba or those interested in doing so. It's nothing more than a collection of documents written by samba developers about the internals of various parts of samba and the SMB protocol. It's still (and will always be) incomplete. The most recent version of this document can be found at {\textless}\url{http://devel.samba.org/}{\textgreater}.

This documentation is distributed under the GNU General Public License (GPL) version 2. A copy of the license is included with the Samba source distribution. A copy can be found on-line at {\textless}\url{http://www.fsf.org/licenses/gpl.txt}{\textgreater}

\begin{admonition}{xslt/figures/warning}{Warning}% NOTICE: see the db2latex FAQ w.r.t db2latex variable $latex.admonition.path

This document is incomplete and unmaintained. It is merely a collection of development-related notes.
\end{admonition}

\end{abstract}

% -------------------------------------------------------------
% Preface 
% -------------------------------------------------------------
\chapter*{Attribution}%
\label{id2438505}\hypertarget{id2438505}{}%

\hyperlink{unix-smb}{Chapter {\ref{unix-smb}}, {``}NetBIOS in a Unix World{''}} 
\begin{itemize}
%--- Item
\item 
Andrew Tridgell

\end{itemize}


\hyperlink{ntdomain}{Chapter {\ref{ntdomain}}, {``}NT Domain RPC's{''}} 
\begin{itemize}
%--- Item
\item 
Luke Leighton{\textless}\url{mailto:lkcl@switchboard.net}{\textgreater}


%--- Item
\item 
Paul Ashton{\textless}\url{mailto:paul@argo.demon.co.uk}{\textgreater}


%--- Item
\item 
Duncan Stansfield{\textless}\url{mailto:duncans@sco.com}{\textgreater}

\end{itemize}


\hyperlink{architecture}{Chapter {\ref{architecture}}, {``}Samba Architecture{''}} 
\begin{itemize}
%--- Item
\item 
Dan Shearer

\end{itemize}


\hyperlink{debug}{Chapter {\ref{debug}}, {``}The samba DEBUG system{''}} 
\begin{itemize}
%--- Item
\item 
Chris Hertel

\end{itemize}


\hyperlink{internals}{Chapter {\ref{internals}}, {``}Samba Internals{''}} 
\begin{itemize}
%--- Item
\item 
David Chappell{\textless}\url{mailto:David.Chappell@mail.trincoll.edu}{\textgreater}

\end{itemize}


\hyperlink{CodingSuggestions}{Chapter {\ref{CodingSuggestions}}, {``}Coding Suggestions{''}} 
\begin{itemize}
%--- Item
\item 
Steve French


%--- Item
\item 
Simo Sorce


%--- Item
\item 
Andrew Bartlett


%--- Item
\item 
Tim Potter


%--- Item
\item 
Martin Pool

\end{itemize}


\hyperlink{contributing}{Chapter {\ref{contributing}}, {``}Contributing code{''}} 
\begin{itemize}
%--- Item
\item 
Jelmer R. Vernooij{\textless}\url{mailto:jelmer@samba.org}{\textgreater}

\end{itemize}


\hyperlink{modules}{Chapter {\ref{modules}}, {``}Modules{''}} 
\begin{itemize}
%--- Item
\item 
Jelmer Vernooij{\textless}\url{mailto:jelmer@samba.org}{\textgreater}

\end{itemize}


\hyperlink{rpc-plugin}{Chapter {\ref{rpc-plugin}}, {``}RPC Pluggable Modules{''}} 
\begin{itemize}
%--- Item
\item 
Anthony Liguori{\textless}\url{mailto:aliguor@us.ibm.com}{\textgreater}


%--- Item
\item 
Jelmer Vernooij{\textless}\url{mailto:jelmer@samba.org}{\textgreater}

\end{itemize}


\hyperlink{vfs}{Chapter {\ref{vfs}}, {``}VFS Modules{''}} 
\begin{itemize}
%--- Item
\item 
Alexander Bokovoy{\textless}\url{mailto:ab@samba.org}{\textgreater}


%--- Item
\item 
Stefan Metzmacher{\textless}\url{mailto:metze@samba.org}{\textgreater}

\end{itemize}


\hyperlink{parsing}{Chapter {\ref{parsing}}, {``}The smb.conf file{''}} 
\begin{itemize}
%--- Item
\item 
Chris Hertel

\end{itemize}


\hyperlink{wins}{Chapter {\ref{wins}}, {``}Samba WINS Internals{''}} 
\begin{itemize}
%--- Item
\item 
Gerald Carter

\end{itemize}


\hyperlink{pwencrypt}{Chapter {\ref{pwencrypt}}, {``}LanMan and NT Password Encryption{''}} 
\begin{itemize}
%--- Item
\item 
Jeremy Allison{\textless}\url{mailto:samba@samba.org}{\textgreater}

\end{itemize}


\hyperlink{tracing}{Chapter {\ref{tracing}}, {``}Tracing samba system calls{''}} 
\begin{itemize}
%--- Item
\item 
Andrew Tridgell

\end{itemize}


\hyperlink{devprinting}{Chapter {\ref{devprinting}}, {``}Samba Printing Internals{''}} 
\begin{itemize}
%--- Item
\item 
Gerald Carter

\end{itemize}


\hyperlink{Packaging}{Chapter {\ref{Packaging}}, {``}Notes to packagers{''}} 
\begin{itemize}
%--- Item
\item 
Jelmer Vernooij

\end{itemize}

\cleardoublepage 
\docbooktolatexnoparskip

\makeatletter
\def\dbtolatex@contentsid{id2407184}
\let\dbtolatex@@contentsname\contentsname
\newif\ifdocbooktolatexcontentsname\docbooktolatexcontentsnametrue
\def\dbtolatex@contentslabel{%
 \label{\dbtolatex@contentsid}\hypertarget{\dbtolatex@contentsid}{\dbtolatex@@contentsname}%
 \global\docbooktolatexcontentsnamefalse}
\def\contentsname{\ifdocbooktolatexcontentsname\dbtolatex@contentslabel\else\dbtolatex@@contentsname\fi}
\tableofcontents
\let\contentsname\dbtolatex@@contentsname
\Hy@writebookmark{}{\dbtolatex@@contentsname}{\dbtolatex@contentsid}{0}{toc}%
\makeatother
				\docbooktolatexrestoreparskip
\pagenumbering{arabic} % ------------------------------------------------------------- 
% 
% PART The protocol
% 
% ------------------------------------------------------------- 
\part{The protocol}
\label{id2407191}\hypertarget{id2407191}{}%

% -------------------------------------------------------------
% Chapter NetBIOS in a Unix World 
% ------------------------------------------------------------- 	
\chapter{NetBIOS in a Unix World}
\label{unix-smb}\hypertarget{unix-smb}{}%

% ------------------------   
% Section 
\section{Introduction}
\label{id2491544}\hypertarget{id2491544}{}%

This is a short document that describes some of the issues that confront a SMB implementation on unix, and how Samba copes with them. They may help people who are looking at unix\textless{}-\textgreater{}PC interoperability.

It was written to help out a person who was writing a paper on unix to PC connectivity.

% ------------------------   
% Section 
\section{Usernames}
\label{id2462775}\hypertarget{id2462775}{}%

The SMB protocol has only a loose username concept. Early SMB protocols (such as CORE and COREPLUS) have no username concept at all. Even in later protocols clients often attempt operations (particularly printer operations) without first validating a username on the server.

Unix security is based around username/password pairs. A unix box should not allow clients to do any substantive operation without some sort of validation.

The problem mostly manifests itself when the unix server is in "share level" security mode. This is the default mode as the alternative "user level" security mode usually forces a client to connect to the server as the same user for each connected share, which is inconvenient in many sites.

In "share level" security the client normally gives a username in the "session setup" protocol, but does not supply an accompanying password. The client then connects to resources using the "tree connect" protocol, and supplies a password. The problem is that the user on the PC types the username and the password in different contexts, unaware that they need to go together to give access to the server. The username is normally the one the user typed in when they "logged onto" the PC (this assumes Windows for Workgroups). The password is the one they chose when connecting to the disk or printer.

The user often chooses a totally different username for their login as for the drive connection. Often they also want to access different drives as different usernames. The unix server needs some way of divining the correct username to combine with each password.

Samba tries to avoid this problem using several methods. These succeed in the vast majority of cases. The methods include username maps, the service\%user syntax, the saving of session setup usernames for later validation and the derivation of the username from the service name (either directly or via the user= option).

% ------------------------   
% Section 
\section{File Ownership}
\label{id2474704}\hypertarget{id2474704}{}%

The commonly used SMB protocols have no way of saying "you can't do that because you don't own the file". They have, in fact, no concept of file ownership at all.

This brings up all sorts of interesting problems. For example, when you copy a file to a unix drive, and the file is world writeable but owned by another user the file will transfer correctly but will receive the wrong date. This is because the utime() call under unix only succeeds for the owner of the file, or root, even if the file is world writeable. For security reasons Samba does all file operations as the validated user, not root, so the utime() fails. This can stuff up shared development diectories as programs like "make" will not get file time comparisons right.

There are several possible solutions to this problem, including username mapping, and forcing a specific username for particular shares.

% ------------------------   
% Section 
\section{Passwords}
\label{id2474298}\hypertarget{id2474298}{}%

Many SMB clients uppercase passwords before sending them. I have no idea why they do this. Interestingly WfWg uppercases the password only if the server is running a protocol greater than COREPLUS, so obviously it isn't just the data entry routines that are to blame.

Unix passwords are case sensitive. So if users use mixed case passwords they are in trouble.

Samba can try to cope with this by either using the "password level" option which causes Samba to try the offered password with up to the specified number of case changes, or by using the "password server" option which allows Samba to do its validation via another machine (typically a WinNT server).

Samba supports the password encryption method used by SMB clients. Note that the use of password encryption in Microsoft networking leads to password hashes that are "plain text equivalent". This means that it is *VERY* important to ensure that the Samba smbpasswd file containing these password hashes is only readable by the root user. See the documentation ENCRYPTION.txt for more details.

% ------------------------   
% Section 
\section{Locking}
\label{id2430536}\hypertarget{id2430536}{}%

Since samba 2.2, samba supports other types of locking as well. This section is outdated.

The locking calls available under a DOS/Windows environment are much richer than those available in unix. This means a unix server (like Samba) choosing to use the standard fcntl() based unix locking calls to implement SMB locking has to improvise a bit.

One major problem is that dos locks can be in a 32 bit (unsigned) range. Unix locking calls are 32 bits, but are signed, giving only a 31 bit range. Unfortunately OLE2 clients use the top bit to select a locking range used for OLE semaphores.

To work around this problem Samba compresses the 32 bit range into 31 bits by appropriate bit shifting. This seems to work but is not ideal. In a future version a separate SMB lockd may be added to cope with the problem.

It also doesn't help that many unix lockd daemons are very buggy and crash at the slightest provocation. They normally go mostly unused in a unix environment because few unix programs use byte range locking. The stress of huge numbers of lock requests from dos/windows clients can kill the daemon on some systems.

The second major problem is the "opportunistic locking" requested by some clients. If a client requests opportunistic locking then it is asking the server to notify it if anyone else tries to do something on the same file, at which time the client will say if it is willing to give up its lock. Unix has no simple way of implementing opportunistic locking, and currently Samba has no support for it.

% ------------------------   
% Section 
\section{Deny Modes}
\label{id2430589}\hypertarget{id2430589}{}%

When a SMB client opens a file it asks for a particular "deny mode" to be placed on the file. These modes (DENY\_NONE, DENY\_READ, DENY\_WRITE, DENY\_ALL, DENY\_FCB and DENY\_DOS) specify what actions should be allowed by anyone else who tries to use the file at the same time. If DENY\_READ is placed on the file, for example, then any attempt to open the file for reading should fail.

Unix has no equivalent notion. To implement this Samba uses either lock files based on the files inode and placed in a separate lock directory or a shared memory implementation. The lock file method is clumsy and consumes processing and file resources, the shared memory implementation is vastly prefered and is turned on by default for those systems that support it.

% ------------------------   
% Section 
\section{Trapdoor UIDs}
\label{id2430616}\hypertarget{id2430616}{}%

A SMB session can run with several uids on the one socket. This happens when a user connects to two shares with different usernames. To cope with this the unix server needs to switch uids within the one process. On some unixes (such as SCO) this is not possible. This means that on those unixes the client is restricted to a single uid.

Note that you can also get the "trapdoor uid" message for other reasons. Please see the FAQ for details.

% ------------------------   
% Section 
\section{Port numbers}
\label{id2457963}\hypertarget{id2457963}{}%

There is a convention that clients on sockets use high "unprivileged" port numbers (\textgreater{}1000) and connect to servers on low "privilegedg" port numbers. This is enforced in Unix as non-root users can't open a socket for listening on port numbers less than 1000.

Most PC based SMB clients (such as WfWg and WinNT) don't follow this convention completely. The main culprit is the netbios nameserving on udp port 137. Name query requests come from a source port of 137. This is a problem when you combine it with the common firewalling technique of not allowing incoming packets on low port numbers. This means that these clients can't query a netbios nameserver on the other side of a low port based firewall.

The problem is more severe with netbios node status queries. I've found that WfWg, Win95 and WinNT3.5 all respond to netbios node status queries on port 137 no matter what the source port was in the request. This works between machines that are both using port 137, but it means it's not possible for a unix user to do a node status request to any of these OSes unless they are running as root. The answer comes back, but it goes to port 137 which the unix user can't listen on. Interestingly WinNT3.1 got this right - it sends node status responses back to the source port in the request.

% ------------------------   
% Section 
\section{Protocol Complexity}
\label{id2462089}\hypertarget{id2462089}{}%

There are many "protocol levels" in the SMB protocol. It seems that each time new functionality was added to a Microsoft operating system, they added the equivalent functions in a new protocol level of the SMB protocol to "externalise" the new capabilities.

This means the protocol is very "rich", offering many ways of doing each file operation. This means SMB servers need to be complex and large. It also means it is very difficult to make them bug free. It is not just Samba that suffers from this problem, other servers such as WinNT don't support every variation of every call and it has almost certainly been a headache for MS developers to support the myriad of SMB calls that are available.

There are about 65 "top level" operations in the SMB protocol (things like SMBread and SMBwrite). Some of these include hundreds of sub-functions (SMBtrans has at least 120 sub-functions, like DosPrintQAdd and NetSessionEnum). All of them take several options that can change the way they work. Many take dozens of possible "information levels" that change the structures that need to be returned. Samba supports all but 2 of the "top level" functions. It supports only 8 (so far) of the SMBtrans sub-functions. Even NT doesn't support them all.

Samba currently supports up to the "NT LM 0.12" protocol, which is the one preferred by Win95 and WinNT3.5. Luckily this protocol level has a "capabilities" field which specifies which super-duper new-fangled options the server suports. This helps to make the implementation of this protocol level much easier.

There is also a problem with the SMB specications. SMB is a X/Open spec, but the X/Open book is far from ideal, and fails to cover many important issues, leaving much to the imagination. Microsoft recently renamed the SMB protocol CIFS (Common Internet File System) and have published new specifications. These are far superior to the old X/Open documents but there are still undocumented calls and features. This specification is actively being worked on by a CIFS developers mailing list hosted by Microsft.

% -------------------------------------------------------------
% Chapter NT Domain RPC's 
% ------------------------------------------------------------- 	
\chapter{NT Domain RPC's}
\label{ntdomain}\hypertarget{ntdomain}{}%

% ------------------------   
% Section 
\section{Introduction}
\label{id2465771}\hypertarget{id2465771}{}%

This document contains information to provide an NT workstation with login services, without the need for an NT server. It is the sgml version of {\textless}\url{http://mailhost.cb1.com/~lkcl/cifsntdomain.txt}{\textgreater}, controlled by Luke.

It should be possible to select a domain instead of a workgroup (in the NT workstation's TCP/IP settings) and after the obligatory reboot, type in a username, password, select a domain and successfully log in. I would appreciate any feedback on your experiences with this process, and any comments, corrections and additions to this document.

The packets described here can be easily derived from (and are probably better understood using) Netmon.exe. You will need to use the version of Netmon that matches your system, in order to correctly decode the NETLOGON, lsarpc and srvsvc Transact pipes. This document is derived from NT Service Pack 1 and its corresponding version of Netmon. It is intended that an annotated packet trace be produced, which will likely be more instructive than this document.

Also needed, to fully implement NT Domain Login Services, is the document describing the cryptographic part of the NT authentication. This document is available from comp.protocols.smb; from the ntsecurity.net digest and from the samba digest, amongst other sources.

A copy is available from:

{\textless}\url{http://ntbugtraq.rc.on.ca/SCRIPTS/WA.EXE?A2=ind9708;L=ntbugtraq;O=A;P=2935}{\textgreater}

{\textless}\url{http://mailhost.cb1.com/~lkcl/crypt.html}{\textgreater}

A c-code implementation, provided by Linus Nordberg\label{id2459674}\begingroup\catcode`\#=12\footnote{ {\textless}\url{mailto:linus@incolumitas.se}{\textgreater}}\endgroup\docbooktolatexmakefootnoteref{id2459674} of this protocol is available from:

{\textless}\url{http://samba.org/cgi-bin/mfs/01/digest/1997/97aug/0391.html}{\textgreater}

{\textless}\url{http://mailhost.cb1.com/~lkcl/crypt.txt}{\textgreater}

Also used to provide debugging information is the Check Build version of NT workstation, and enabling full debugging in NETLOGON. This is achieved by setting the following REG\_SZ registry key to 0x1ffffff:

{\texttt{\docbookhyphenatefilename{HKLM\docbooktolatexgobble\string\\SYSTEM\docbooktolatexgobble\string\\CurrentControlSet\docbooktolatexgobble\string\\Services\docbooktolatexgobble\string\\Netlogon\docbooktolatexgobble\string\\Parameters}}}

{\em{Incorrect direct editing of the registry can cause your machine to fail. Then again, so can incorrect implementation of this protocol. See "Liability:" above.}}

Bear in mind that each packet over-the-wire will have its origin in an API call. Therefore, there are likely to be structures, enumerations and defines that are usefully documented elsewhere.

This document is by no means complete or authoritative. Missing sections include, but are not limited to:

\begin{enumerate}
%--- Item
\item 
Mappings of RIDs to usernames (and vice-versa).


%--- Item
\item 
What a User ID is and what a Group ID is.


%--- Item
\item 
The exact meaning/definition of various magic constants or enumerations.


%--- Item
\item 
The reply error code and use of that error code when a workstation becomes a member of a domain (to be described later). Failure to return this error code will make the workstation report that it is already a member of the domain.


%--- Item
\item 
the cryptographic side of the NetrServerPasswordSet command, which would allow the workstation to change its password. This password is used to generate the long-term session key. [It is possible to reject this command, and keep the default workstation password].

\end{enumerate}
\subsection{Sources}
\label{id2479508}\hypertarget{id2479508}{}%

\begin{tabular}{l}

cket Traces from Netmonitor (Service Pack 1 and above)  \\
ul Ashton and Luke Leighton's other "NT Domain" doc.  \\
FS documentation - cifs6.txt  \\
FS documentation - cifsrap2.txt  \\
\end{tabular}
\subsection{Credits}
\label{id2479537}\hypertarget{id2479537}{}%

\begin{tabular}{l}

Paul Ashton: loads of work with Net Monitor; understanding the NT authentication system; reference implementation of the NT domain support on which this document is originally based.  \\
Duncan Stansfield: low-level analysis of MSRPC Pipes.  \\
Linus Nordberg: producing c-code from Paul's crypto spec.  \\
Windows Sourcer development team  \\
\end{tabular}

% ------------------------   
% Section 
\section{Notes and Structures}
\label{id2479571}\hypertarget{id2479571}{}%
\subsection{Notes}
\label{id2479577}\hypertarget{id2479577}{}%

\begin{enumerate}
%--- Item
\item 
In the SMB Transact pipes, some "Structures", described here, appear to be 4-byte aligned with the SMB header, at their start. Exactly which "Structures" need aligning is not precisely known or documented.


%--- Item
\item 
In the UDP NTLOGON Mailslots, some "Structures", described here, appear to be 2-byte aligned with the start of the mailslot, at their start.


%--- Item
\item 
Domain SID is of the format S-revision-version-auth1-auth2...authN. e.g S-1-5-123-456-789-123-456. the 5 could be a sub-revision.


%--- Item
\item 
any undocumented buffer pointers must be non-zero if the string buffer it refers to contains characters. exactly what value they should be is unknown. 0x0000 0002 seems to do the trick to indicate that the buffer exists. a NULL buffer pointer indicates that the string buffer is of zero length. If the buffer pointer is NULL, then it is suspected that the structure it refers to is NOT put into (or taken out of) the SMB data stream. This is empirically derived from, for example, the LSA SAM Logon response packet, where if the buffer pointer is NULL, the user information is not inserted into the data stream. Exactly what happens with an array of buffer pointers is not known, although an educated guess can be made.


%--- Item
\item 
an array of structures (a container) appears to have a count and a pointer. if the count is zero, the pointer is also zero. no further data is put into or taken out of the SMB data stream. if the count is non-zero, then the pointer is also non-zero. immediately following the pointer is the count again, followed by an array of container sub-structures. the count appears a third time after the last sub-structure.

\end{enumerate}
\subsection{Enumerations}
\label{id2456987}\hypertarget{id2456987}{}%
\subsubsection{MSRPC Header type}
\label{id2457676}\hypertarget{id2457676}{}%

command number in the msrpc packet header

\begin{description}
% \null and \mbox are tricks to induce different typesetting decisions
\item[{MSRPC\_Request:}]\null{}
0x00
% \null and \mbox are tricks to induce different typesetting decisions
\item[{MSRPC\_Response:}]\null{}
0x02
% \null and \mbox are tricks to induce different typesetting decisions
\item[{MSRPC\_Bind:}]\null{}
0x0B
% \null and \mbox are tricks to induce different typesetting decisions
\item[{MSRPC\_BindAck:}]\null{}
0x0C
\end{description}
\subsubsection{MSRPC Packet info}
\label{id2479102}\hypertarget{id2479102}{}%

The meaning of these flags is undocumented

\begin{description}
% \null and \mbox are tricks to induce different typesetting decisions
\item[{FirstFrag:}]\null{}
0x01
% \null and \mbox are tricks to induce different typesetting decisions
\item[{LastFrag:}]\null{}
0x02
% \null and \mbox are tricks to induce different typesetting decisions
\item[{NotaFrag:}]\null{}
0x04
% \null and \mbox are tricks to induce different typesetting decisions
\item[{RecRespond:}]\null{}
0x08
% \null and \mbox are tricks to induce different typesetting decisions
\item[{NoMultiplex:}]\null{}
0x10
% \null and \mbox are tricks to induce different typesetting decisions
\item[{NotForIdemp:}]\null{}
0x20
% \null and \mbox are tricks to induce different typesetting decisions
\item[{NotforBcast:}]\null{}
0x40
% \null and \mbox are tricks to induce different typesetting decisions
\item[{NoUuid:}]\null{}
0x80
\end{description}
\subsection{Structures}
\label{id2479218}\hypertarget{id2479218}{}%
\subsubsection{VOID *}
\label{id2479224}\hypertarget{id2479224}{}%

sizeof VOID* is 32 bits.
\subsubsection{char}
\label{id2479234}\hypertarget{id2479234}{}%

sizeof char is 8 bits.
\subsubsection{UTIME}
\label{id2479243}\hypertarget{id2479243}{}%

UTIME is 32 bits, indicating time in seconds since 01jan1970. documented in cifs6.txt (section 3.5 page, page 30).
\subsubsection{NTTIME}
\label{id2479255}\hypertarget{id2479255}{}%

NTTIME is 64 bits. documented in cifs6.txt (section 3.5 page, page 30).
\subsubsection{DOM\_SID (domain SID structure)}
\label{id2479266}\hypertarget{id2479266}{}%

\begin{description}
% \null and \mbox are tricks to induce different typesetting decisions
\item[{UINT32}]\null{}
num of sub-authorities in domain SID
% \null and \mbox are tricks to induce different typesetting decisions
\item[{UINT8}]\null{}
SID revision number
% \null and \mbox are tricks to induce different typesetting decisions
\item[{UINT8}]\null{}
num of sub-authorities in domain SID
% \null and \mbox are tricks to induce different typesetting decisions
\item[{UINT8[6]}]\null{}
6 bytes for domain SID - Identifier Authority.
% \null and \mbox are tricks to induce different typesetting decisions
\item[{UINT16[n\_subauths]}]\null{}
domain SID sub-authorities
\end{description}

{\em{Note: the domain SID is documented elsewhere.}}
\subsubsection{STR (string)}
\label{id2470155}\hypertarget{id2470155}{}%

STR (string) is a char[] : a null-terminated string of ascii characters.
\subsubsection{UNIHDR (unicode string header)}
\label{id2470168}\hypertarget{id2470168}{}%

\begin{description}
% \null and \mbox are tricks to induce different typesetting decisions
\item[{UINT16}]\null{}
length of unicode string
% \null and \mbox are tricks to induce different typesetting decisions
\item[{UINT16}]\null{}
max length of unicode string
% \null and \mbox are tricks to induce different typesetting decisions
\item[{UINT32}]\null{}
4 - undocumented.
\end{description}
\subsubsection{UNIHDR2 (unicode string header plus buffer pointer)}
\label{id2470215}\hypertarget{id2470215}{}%

\begin{description}
% \null and \mbox are tricks to induce different typesetting decisions
\item[{UNIHDR}]\null{}
unicode string header
% \null and \mbox are tricks to induce different typesetting decisions
\item[{VOID*}]\null{}
undocumented buffer pointer
\end{description}
\subsubsection{UNISTR (unicode string)}
\label{id2470251}\hypertarget{id2470251}{}%

\begin{description}
% \null and \mbox are tricks to induce different typesetting decisions
\item[{UINT16[]}]\null{}
null-terminated string of unicode characters.
\end{description}
\subsubsection{NAME (length-indicated unicode string)}
\label{id2470275}\hypertarget{id2470275}{}%

\begin{description}
% \null and \mbox are tricks to induce different typesetting decisions
\item[{UINT32}]\null{}
length of unicode string
% \null and \mbox are tricks to induce different typesetting decisions
\item[{UINT16[]}]\null{}
null-terminated string of unicode characters.
\end{description}
\subsubsection{UNISTR2 (aligned unicode string)}
\label{id2470312}\hypertarget{id2470312}{}%

\begin{description}
% \null and \mbox are tricks to induce different typesetting decisions
\item[{UINT8[]}]\null{}
padding to get unicode string 4-byte aligned with the start of the SMB header.
% \null and \mbox are tricks to induce different typesetting decisions
\item[{UINT32}]\null{}
max length of unicode string
% \null and \mbox are tricks to induce different typesetting decisions
\item[{UINT32}]\null{}
0 - undocumented
% \null and \mbox are tricks to induce different typesetting decisions
\item[{UINT32}]\null{}
length of unicode string
% \null and \mbox are tricks to induce different typesetting decisions
\item[{UINT16[]}]\null{}
string of uncode characters
\end{description}
\subsubsection{OBJ\_ATTR (object attributes)}
\label{id2470386}\hypertarget{id2470386}{}%

\begin{description}
% \null and \mbox are tricks to induce different typesetting decisions
\item[{UINT32}]\null{}
0x18 - length (in bytes) including the length field.
% \null and \mbox are tricks to induce different typesetting decisions
\item[{VOID*}]\null{}
0 - root directory (pointer)
% \null and \mbox are tricks to induce different typesetting decisions
\item[{VOID*}]\null{}
0 - object name (pointer)
% \null and \mbox are tricks to induce different typesetting decisions
\item[{UINT32}]\null{}
0 - attributes (undocumented)
% \null and \mbox are tricks to induce different typesetting decisions
\item[{VOID*}]\null{}
0 - security descriptior (pointer)
% \null and \mbox are tricks to induce different typesetting decisions
\item[{UINT32}]\null{}
0 - security quality of service
\end{description}
\subsubsection{POL\_HND (LSA policy handle)}
\label{id2463576}\hypertarget{id2463576}{}%

\begin{description}
% \null and \mbox are tricks to induce different typesetting decisions
\item[{char[20]}]\null{}
policy handle
\end{description}
\subsubsection{DOM\_SID2 (domain SID structure, SIDS stored in unicode)}
\label{id2463599}\hypertarget{id2463599}{}%

\begin{description}
% \null and \mbox are tricks to induce different typesetting decisions
\item[{UINT32}]\null{}
5 - SID type
% \null and \mbox are tricks to induce different typesetting decisions
\item[{UINT32}]\null{}
0 - undocumented
% \null and \mbox are tricks to induce different typesetting decisions
\item[{UNIHDR2}]\null{}
domain SID unicode string header
% \null and \mbox are tricks to induce different typesetting decisions
\item[{UNISTR}]\null{}
domain SID unicode string
\end{description}

{\em{Note: there is a conflict between the unicode string header and the unicode string itself as to which to use to indicate string length. this will need to be resolved.}}

{\em{Note: the SID type indicates, for example, an alias; a well-known group etc. this is documented somewhere.}}
\subsubsection{DOM\_RID (domain RID structure)}
\label{id2463675}\hypertarget{id2463675}{}%

\begin{description}
% \null and \mbox are tricks to induce different typesetting decisions
\item[{UINT32}]\null{}
5 - well-known SID. 1 - user SID (see ShowACLs)
% \null and \mbox are tricks to induce different typesetting decisions
\item[{UINT32}]\null{}
5 - undocumented
% \null and \mbox are tricks to induce different typesetting decisions
\item[{UINT32}]\null{}
domain RID
% \null and \mbox are tricks to induce different typesetting decisions
\item[{UINT32}]\null{}
0 - domain index out of above reference domains
\end{description}
\subsubsection{LOG\_INFO (server, account, client structure)}
\label{id2463736}\hypertarget{id2463736}{}%

{\em{Note: logon server name starts with two '\textbackslash ' characters and is upper case.}}

{\em{Note: account name is the logon client name from the LSA Request Challenge, with a \$ on the end of it, in upper case.}}

\begin{description}
% \null and \mbox are tricks to induce different typesetting decisions
\item[{VOID*}]\null{}
undocumented buffer pointer
% \null and \mbox are tricks to induce different typesetting decisions
\item[{UNISTR2}]\null{}
logon server unicode string
% \null and \mbox are tricks to induce different typesetting decisions
\item[{UNISTR2}]\null{}
account name unicode string
% \null and \mbox are tricks to induce different typesetting decisions
\item[{UINT16}]\null{}
sec\_chan - security channel type
% \null and \mbox are tricks to induce different typesetting decisions
\item[{UNISTR2}]\null{}
logon client machine unicode string
\end{description}
\subsubsection{CLNT\_SRV (server, client names structure)}
\label{id2463821}\hypertarget{id2463821}{}%

{\em{Note: logon server name starts with two '\textbackslash ' characters and is upper case.}}

\begin{description}
% \null and \mbox are tricks to induce different typesetting decisions
\item[{VOID*}]\null{}
undocumented buffer pointer
% \null and \mbox are tricks to induce different typesetting decisions
\item[{UNISTR2}]\null{}
logon server unicode string
% \null and \mbox are tricks to induce different typesetting decisions
\item[{VOID*}]\null{}
undocumented buffer pointer
% \null and \mbox are tricks to induce different typesetting decisions
\item[{UNISTR2}]\null{}
logon client machine unicode string
\end{description}
\subsubsection{CREDS (credentials + time stamp)}
\label{id2507359}\hypertarget{id2507359}{}%

\begin{description}
% \null and \mbox are tricks to induce different typesetting decisions
\item[{char[8]}]\null{}
credentials
% \null and \mbox are tricks to induce different typesetting decisions
\item[{UTIME}]\null{}
time stamp
\end{description}
\subsubsection{CLNT\_INFO2 (server, client structure, client credentials)}
\label{id2507390}\hypertarget{id2507390}{}%

{\em{Note: whenever this structure appears in a request, you must take a copy of the client-calculated credentials received, because they will beused in subsequent credential checks. the presumed intention is to maintain an authenticated request/response trail.}}

\begin{description}
% \null and \mbox are tricks to induce different typesetting decisions
\item[{CLNT\_SRV}]\null{}
client and server names
% \null and \mbox are tricks to induce different typesetting decisions
\item[{UINT8[]}]\null{}
???? padding, for 4-byte alignment with SMB header.
% \null and \mbox are tricks to induce different typesetting decisions
\item[{VOID*}]\null{}
pointer to client credentials.
% \null and \mbox are tricks to induce different typesetting decisions
\item[{CREDS}]\null{}
client-calculated credentials + client time
\end{description}
\subsubsection{CLNT\_INFO (server, account, client structure, client credentials)}
\label{id2507455}\hypertarget{id2507455}{}%

{\em{Note: whenever this structure appears in a request, you must take a copy of the client-calculated credentials received, because they will be used in subsequent credential checks. the presumed intention is to maintain an authenticated request/response trail.}}

\begin{description}
% \null and \mbox are tricks to induce different typesetting decisions
\item[{LOG\_INFO}]\null{}
logon account info
% \null and \mbox are tricks to induce different typesetting decisions
\item[{CREDS}]\null{}
client-calculated credentials + client time
\end{description}
\subsubsection{ID\_INFO\_1 (id info structure, auth level 1)}
\label{id2507496}\hypertarget{id2507496}{}%

\begin{description}
% \null and \mbox are tricks to induce different typesetting decisions
\item[{VOID*}]\null{}
ptr\_id\_info\_1
% \null and \mbox are tricks to induce different typesetting decisions
\item[{UNIHDR}]\null{}
domain name unicode header
% \null and \mbox are tricks to induce different typesetting decisions
\item[{UINT32}]\null{}
param control
% \null and \mbox are tricks to induce different typesetting decisions
\item[{UINT64}]\null{}
logon ID
% \null and \mbox are tricks to induce different typesetting decisions
\item[{UNIHDR}]\null{}
user name unicode header
% \null and \mbox are tricks to induce different typesetting decisions
\item[{UNIHDR}]\null{}
workgroup name unicode header
% \null and \mbox are tricks to induce different typesetting decisions
\item[{char[16]}]\null{}
arc4 LM OWF Password
% \null and \mbox are tricks to induce different typesetting decisions
\item[{char[16]}]\null{}
arc4 NT OWF Password
% \null and \mbox are tricks to induce different typesetting decisions
\item[{UNISTR2}]\null{}
domain name unicode string
% \null and \mbox are tricks to induce different typesetting decisions
\item[{UNISTR2}]\null{}
user name unicode string
% \null and \mbox are tricks to induce different typesetting decisions
\item[{UNISTR2}]\null{}
workstation name unicode string
\end{description}
\subsubsection{SAM\_INFO (sam logon/logoff id info structure)}
\label{id2507626}\hypertarget{id2507626}{}%

{\em{Note: presumably, the return credentials is supposedly for the server to verify that the credential chain hasn't been compromised.}}

\begin{description}
% \null and \mbox are tricks to induce different typesetting decisions
\item[{CLNT\_INFO2}]\null{}
client identification/authentication info
% \null and \mbox are tricks to induce different typesetting decisions
\item[{VOID*}]\null{}
pointer to return credentials.
% \null and \mbox are tricks to induce different typesetting decisions
\item[{CRED}]\null{}
return credentials - ignored.
% \null and \mbox are tricks to induce different typesetting decisions
\item[{UINT16}]\null{}
logon level
% \null and \mbox are tricks to induce different typesetting decisions
\item[{UINT16}]\null{}
switch value
\end{description}


\begin{Verbatim}[]

        switch (switch_value)
        case 1:
        {
            ID_INFO_1     id_info_1;
        }

\end{Verbatim}

\subsubsection{GID (group id info)}
\label{id2507706}\hypertarget{id2507706}{}%

\begin{description}
% \null and \mbox are tricks to induce different typesetting decisions
\item[{UINT32}]\null{}
group id
% \null and \mbox are tricks to induce different typesetting decisions
\item[{UINT32}]\null{}
user attributes (only used by NT 3.1 and 3.51)
\end{description}
\subsubsection{DOM\_REF (domain reference info)}
\label{id2507736}\hypertarget{id2507736}{}%

\begin{description}
% \null and \mbox are tricks to induce different typesetting decisions
\item[{VOID*}]\null{}
undocumented buffer pointer.
% \null and \mbox are tricks to induce different typesetting decisions
\item[{UINT32}]\null{}
num referenced domains?
% \null and \mbox are tricks to induce different typesetting decisions
\item[{VOID*}]\null{}
undocumented domain name buffer pointer.
% \null and \mbox are tricks to induce different typesetting decisions
\item[{UINT32}]\null{}
32 - max number of entries
% \null and \mbox are tricks to induce different typesetting decisions
\item[{UINT32}]\null{}
4 - num referenced domains?
% \null and \mbox are tricks to induce different typesetting decisions
\item[{UNIHDR2}]\null{}
domain name unicode string header
% \null and \mbox are tricks to induce different typesetting decisions
\item[{UNIHDR2[num\_ref\_doms-1]}]\null{}
referenced domain unicode string headers
% \null and \mbox are tricks to induce different typesetting decisions
\item[{UNISTR}]\null{}
domain name unicode string
% \null and \mbox are tricks to induce different typesetting decisions
\item[{DOM\_SID[num\_ref\_doms]}]\null{}
referenced domain SIDs
\end{description}
\subsubsection{DOM\_INFO (domain info, levels 3 and 5 are the same))}
\label{id2507845}\hypertarget{id2507845}{}%

\begin{description}
% \null and \mbox are tricks to induce different typesetting decisions
\item[{UINT8[]}]\null{}
??? padding to get 4-byte alignment with start of SMB header
% \null and \mbox are tricks to induce different typesetting decisions
\item[{UINT16}]\null{}
domain name string length * 2
% \null and \mbox are tricks to induce different typesetting decisions
\item[{UINT16}]\null{}
domain name string length * 2
% \null and \mbox are tricks to induce different typesetting decisions
\item[{VOID*}]\null{}
undocumented domain name string buffer pointer
% \null and \mbox are tricks to induce different typesetting decisions
\item[{VOID*}]\null{}
undocumented domain SID string buffer pointer
% \null and \mbox are tricks to induce different typesetting decisions
\item[{UNISTR2}]\null{}
domain name (unicode string)
% \null and \mbox are tricks to induce different typesetting decisions
\item[{DOM\_SID}]\null{}
domain SID
\end{description}
\subsubsection{USER\_INFO (user logon info)}
\label{id2507933}\hypertarget{id2507933}{}%

{\em{Note: it would be nice to know what the 16 byte user session key is for.}}

\begin{description}
% \null and \mbox are tricks to induce different typesetting decisions
\item[{NTTIME}]\null{}
logon time
% \null and \mbox are tricks to induce different typesetting decisions
\item[{NTTIME}]\null{}
logoff time
% \null and \mbox are tricks to induce different typesetting decisions
\item[{NTTIME}]\null{}
kickoff time
% \null and \mbox are tricks to induce different typesetting decisions
\item[{NTTIME}]\null{}
password last set time
% \null and \mbox are tricks to induce different typesetting decisions
\item[{NTTIME}]\null{}
password can change time
% \null and \mbox are tricks to induce different typesetting decisions
\item[{NTTIME}]\null{}
password must change time
% \null and \mbox are tricks to induce different typesetting decisions
\item[{UNIHDR}]\null{}
username unicode string header
% \null and \mbox are tricks to induce different typesetting decisions
\item[{UNIHDR}]\null{}
user's full name unicode string header
% \null and \mbox are tricks to induce different typesetting decisions
\item[{UNIHDR}]\null{}
logon script unicode string header
% \null and \mbox are tricks to induce different typesetting decisions
\item[{UNIHDR}]\null{}
profile path unicode string header
% \null and \mbox are tricks to induce different typesetting decisions
\item[{UNIHDR}]\null{}
home directory unicode string header
% \null and \mbox are tricks to induce different typesetting decisions
\item[{UNIHDR}]\null{}
home directory drive unicode string header
% \null and \mbox are tricks to induce different typesetting decisions
\item[{UINT16}]\null{}
logon count
% \null and \mbox are tricks to induce different typesetting decisions
\item[{UINT16}]\null{}
bad password count
% \null and \mbox are tricks to induce different typesetting decisions
\item[{UINT32}]\null{}
User ID
% \null and \mbox are tricks to induce different typesetting decisions
\item[{UINT32}]\null{}
Group ID
% \null and \mbox are tricks to induce different typesetting decisions
\item[{UINT32}]\null{}
num groups
% \null and \mbox are tricks to induce different typesetting decisions
\item[{VOID*}]\null{}
undocumented buffer pointer to groups.
% \null and \mbox are tricks to induce different typesetting decisions
\item[{UINT32}]\null{}
user flags
% \null and \mbox are tricks to induce different typesetting decisions
\item[{char[16]}]\null{}
user session key
% \null and \mbox are tricks to induce different typesetting decisions
\item[{UNIHDR}]\null{}
logon server unicode string header
% \null and \mbox are tricks to induce different typesetting decisions
\item[{UNIHDR}]\null{}
logon domain unicode string header
% \null and \mbox are tricks to induce different typesetting decisions
\item[{VOID*}]\null{}
undocumented logon domain id pointer
% \null and \mbox are tricks to induce different typesetting decisions
\item[{char[40]}]\null{}
40 undocumented padding bytes. future expansion?
% \null and \mbox are tricks to induce different typesetting decisions
\item[{UINT32}]\null{}
0 - num\_other\_sids?
% \null and \mbox are tricks to induce different typesetting decisions
\item[{VOID*}]\null{}
NULL - undocumented pointer to other domain SIDs.
% \null and \mbox are tricks to induce different typesetting decisions
\item[{UNISTR2}]\null{}
username unicode string
% \null and \mbox are tricks to induce different typesetting decisions
\item[{UNISTR2}]\null{}
user's full name unicode string
% \null and \mbox are tricks to induce different typesetting decisions
\item[{UNISTR2}]\null{}
logon script unicode string
% \null and \mbox are tricks to induce different typesetting decisions
\item[{UNISTR2}]\null{}
profile path unicode string
% \null and \mbox are tricks to induce different typesetting decisions
\item[{UNISTR2}]\null{}
home directory unicode string
% \null and \mbox are tricks to induce different typesetting decisions
\item[{UNISTR2}]\null{}
home directory drive unicode string
% \null and \mbox are tricks to induce different typesetting decisions
\item[{UINT32}]\null{}
num groups
% \null and \mbox are tricks to induce different typesetting decisions
\item[{GID[num\_groups]}]\null{}
group info
% \null and \mbox are tricks to induce different typesetting decisions
\item[{UNISTR2}]\null{}
logon server unicode string
% \null and \mbox are tricks to induce different typesetting decisions
\item[{UNISTR2}]\null{}
logon domain unicode string
% \null and \mbox are tricks to induce different typesetting decisions
\item[{DOM\_SID}]\null{}
domain SID
% \null and \mbox are tricks to induce different typesetting decisions
\item[{DOM\_SID[num\_sids]}]\null{}
other domain SIDs?
\end{description}
\subsubsection{SH\_INFO\_1\_PTR (pointers to level 1 share info strings)}
\label{id2508365}\hypertarget{id2508365}{}%

{\em{Note: see cifsrap2.txt section5, page 10.}}

\begin{tabular}{l}

0 for shi1\_type indicates a Disk.  \\
1 for shi1\_type indicates a Print Queue.  \\
2 for shi1\_type indicates a Device.  \\
3 for shi1\_type indicates an IPC pipe.  \\
0x8000 0000 (top bit set in shi1\_type) indicates a hidden share.  \\
\end{tabular}

\begin{description}
% \null and \mbox are tricks to induce different typesetting decisions
\item[{VOID*}]\null{}
shi1\_netname - pointer to net name
% \null and \mbox are tricks to induce different typesetting decisions
\item[{UINT32}]\null{}
shi1\_type - type of share. 0 - undocumented.
% \null and \mbox are tricks to induce different typesetting decisions
\item[{VOID*}]\null{}
shi1\_remark - pointer to comment.
\end{description}
\subsubsection{SH\_INFO\_1\_STR (level 1 share info strings)}
\label{id2508441}\hypertarget{id2508441}{}%

\begin{description}
% \null and \mbox are tricks to induce different typesetting decisions
\item[{UNISTR2}]\null{}
shi1\_netname - unicode string of net name
% \null and \mbox are tricks to induce different typesetting decisions
\item[{UNISTR2}]\null{}
shi1\_remark - unicode string of comment.
\end{description}
\subsubsection{SHARE\_INFO\_1\_CTR}
\label{id2508474}\hypertarget{id2508474}{}%

share container with 0 entries:

\begin{description}
% \null and \mbox are tricks to induce different typesetting decisions
\item[{UINT32}]\null{}
0 - EntriesRead
% \null and \mbox are tricks to induce different typesetting decisions
\item[{UINT32}]\null{}
0 - Buffer
\end{description}

share container with \textgreater{} 0 entries:

\begin{description}
% \null and \mbox are tricks to induce different typesetting decisions
\item[{UINT32}]\null{}
EntriesRead
% \null and \mbox are tricks to induce different typesetting decisions
\item[{UINT32}]\null{}
non-zero - Buffer
% \null and \mbox are tricks to induce different typesetting decisions
\item[{UINT32}]\null{}
EntriesRead
% \null and \mbox are tricks to induce different typesetting decisions
\item[{SH\_INFO\_1\_PTR[EntriesRead]}]\null{}
share entry pointers
% \null and \mbox are tricks to induce different typesetting decisions
\item[{SH\_INFO\_1\_STR[EntriesRead]}]\null{}
share entry strings
% \null and \mbox are tricks to induce different typesetting decisions
\item[{UINT8[]}]\null{}
padding to get unicode string 4-byte aligned with start of the SMB header.
% \null and \mbox are tricks to induce different typesetting decisions
\item[{UINT32}]\null{}
EntriesRead
% \null and \mbox are tricks to induce different typesetting decisions
\item[{UINT32}]\null{}
0 - padding
\end{description}
\subsubsection{SERVER\_INFO\_101}
\label{id2508602}\hypertarget{id2508602}{}%

{\em{Note: see cifs6.txt section 6.4 - the fields described therein will be of assistance here. for example, the type listed below is the same as fServerType, which is described in 6.4.1.}}

\begin{description}
% \null and \mbox are tricks to induce different typesetting decisions
\item[{SV\_TYPE\_WORKSTATION}]\null{}
0x00000001 All workstations
% \null and \mbox are tricks to induce different typesetting decisions
\item[{SV\_TYPE\_SERVER}]\null{}
0x00000002 All servers
% \null and \mbox are tricks to induce different typesetting decisions
\item[{SV\_TYPE\_SQLSERVER}]\null{}
0x00000004 Any server running with SQL server
% \null and \mbox are tricks to induce different typesetting decisions
\item[{SV\_TYPE\_DOMAIN\_CTRL}]\null{}
0x00000008 Primary domain controller
% \null and \mbox are tricks to induce different typesetting decisions
\item[{SV\_TYPE\_DOMAIN\_BAKCTRL}]\null{}
0x00000010 Backup domain controller
% \null and \mbox are tricks to induce different typesetting decisions
\item[{SV\_TYPE\_TIME\_SOURCE}]\null{}
0x00000020 Server running the timesource service
% \null and \mbox are tricks to induce different typesetting decisions
\item[{SV\_TYPE\_AFP}]\null{}
0x00000040 Apple File Protocol servers
% \null and \mbox are tricks to induce different typesetting decisions
\item[{SV\_TYPE\_NOVELL}]\null{}
0x00000080 Novell servers
% \null and \mbox are tricks to induce different typesetting decisions
\item[{SV\_TYPE\_DOMAIN\_MEMBER}]\null{}
0x00000100 Domain Member
% \null and \mbox are tricks to induce different typesetting decisions
\item[{SV\_TYPE\_PRINTQ\_SERVER}]\null{}
0x00000200 Server sharing print queue
% \null and \mbox are tricks to induce different typesetting decisions
\item[{SV\_TYPE\_DIALIN\_SERVER}]\null{}
0x00000400 Server running dialin service.
% \null and \mbox are tricks to induce different typesetting decisions
\item[{SV\_TYPE\_XENIX\_SERVER}]\null{}
0x00000800 Xenix server
% \null and \mbox are tricks to induce different typesetting decisions
\item[{SV\_TYPE\_NT}]\null{}
0x00001000 NT server
% \null and \mbox are tricks to induce different typesetting decisions
\item[{SV\_TYPE\_WFW}]\null{}
0x00002000 Server running Windows for
% \null and \mbox are tricks to induce different typesetting decisions
\item[{SV\_TYPE\_SERVER\_NT}]\null{}
0x00008000 Windows NT non DC server
% \null and \mbox are tricks to induce different typesetting decisions
\item[{SV\_TYPE\_POTENTIAL\_BROWSER}]\null{}
0x00010000 Server that can run the browser service
% \null and \mbox are tricks to induce different typesetting decisions
\item[{SV\_TYPE\_BACKUP\_BROWSER}]\null{}
0x00020000 Backup browser server
% \null and \mbox are tricks to induce different typesetting decisions
\item[{SV\_TYPE\_MASTER\_BROWSER}]\null{}
0x00040000 Master browser server
% \null and \mbox are tricks to induce different typesetting decisions
\item[{SV\_TYPE\_DOMAIN\_MASTER}]\null{}
0x00080000 Domain Master Browser server
% \null and \mbox are tricks to induce different typesetting decisions
\item[{SV\_TYPE\_LOCAL\_LIST\_ONLY}]\null{}
0x40000000 Enumerate only entries marked "local"
% \null and \mbox are tricks to induce different typesetting decisions
\item[{SV\_TYPE\_DOMAIN\_ENUM}]\null{}
0x80000000 Enumerate Domains. The pszServer and pszDomain parameters must be NULL.
\end{description}

\begin{description}
% \null and \mbox are tricks to induce different typesetting decisions
\item[{UINT32}]\null{}
500 - platform\_id
% \null and \mbox are tricks to induce different typesetting decisions
\item[{VOID*}]\null{}
pointer to name
% \null and \mbox are tricks to induce different typesetting decisions
\item[{UINT32}]\null{}
5 - major version
% \null and \mbox are tricks to induce different typesetting decisions
\item[{UINT32}]\null{}
4 - minor version
% \null and \mbox are tricks to induce different typesetting decisions
\item[{UINT32}]\null{}
type (SV\_TYPE\_... bit field)
% \null and \mbox are tricks to induce different typesetting decisions
\item[{VOID*}]\null{}
pointer to comment
% \null and \mbox are tricks to induce different typesetting decisions
\item[{UNISTR2}]\null{}
sv101\_name - unicode string of server name
% \null and \mbox are tricks to induce different typesetting decisions
\item[{UNISTR2}]\null{}
sv\_101\_comment - unicode string of server comment.
% \null and \mbox are tricks to induce different typesetting decisions
\item[{UINT8[]}]\null{}
padding to get unicode string 4-byte aligned with start of the SMB header.
\end{description}

% ------------------------   
% Section 
\section{MSRPC over Transact Named Pipe}
\label{id2508960}\hypertarget{id2508960}{}%

For details on the SMB Transact Named Pipe, see cifs6.txt
\subsection{MSRPC Pipes}
\label{id2508970}\hypertarget{id2508970}{}%

The MSRPC is conducted over an SMB Transact Pipe with a name of {\texttt{\docbookhyphenatefilename{\docbooktolatexgobble\string\\PIPE\docbooktolatexgobble\string\\}}}. You must first obtain a 16 bit file handle, by sending a SMBopenX with the pipe name {\texttt{\docbookhyphenatefilename{\docbooktolatexgobble\string\\PIPE\docbooktolatexgobble\string\\srvsvc}}} for example. You can then perform an SMB Trans, and must carry out an SMBclose on the file handle once you are finished.

Trans Requests must be sent with two setup UINT16s, no UINT16 params (none known about), and UINT8 data parameters sufficient to contain the MSRPC header, and MSRPC data. The first UINT16 setup parameter must be either 0x0026 to indicate an RPC, or 0x0001 to indicate Set Named Pipe Handle state. The second UINT16 parameter must be the file handle for the pipe, obtained above.

The Data section for an API Command of 0x0026 (RPC pipe) in the Trans Request is the RPC Header, followed by the RPC Data. The Data section for an API Command of 0x0001 (Set Named Pipe Handle state) is two bytes. The only value seen for these two bytes is 0x00 0x43.

MSRPC Responses are sent as response data inside standard SMB Trans responses, with the MSRPC Header, MSRPC Data and MSRPC tail.

It is suspected that the Trans Requests will need to be at least 2-byte aligned (probably 4-byte). This is standard practice for SMBs. It is also independent of the observed 4-byte alignments with the start of the MSRPC header, including the 4-byte alignment between the MSRPC header and the MSRPC data.

First, an SMBtconX connection is made to the IPC\$ share. The connection must be made using encrypted passwords, not clear-text. Then, an SMBopenX is made on the pipe. Then, a Set Named Pipe Handle State must be sent, after which the pipe is ready to accept API commands. Lastly, and SMBclose is sent.

To be resolved:

lkcl/01nov97 there appear to be two additional bytes after the null-terminated \textbackslash PIPE\textbackslash \ name for the RPC pipe. Values seen so far are listed below:


\begin{Verbatim}[]

        initial SMBopenX request:         RPC API command 0x26 params:
        "\\PIPE\\lsarpc"                  0x65 0x63; 0x72 0x70; 0x44 0x65;
        "\\PIPE\\srvsvc"                  0x73 0x76; 0x4E 0x00; 0x5C 0x43;

\end{Verbatim}

\subsection{Header}
\label{id2509054}\hypertarget{id2509054}{}%

[section to be rewritten, following receipt of work by Duncan Stansfield]

Interesting note: if you set packed data representation to 0x0100 0000 then all 4-byte and 2-byte word ordering is turned around!

The start of each of the NTLSA and NETLOGON named pipes begins with:
 {\em{offset:}} 00 {\em{Variable type:}} UINT8 {\em{Variable data:}} 5 - RPC major version \\
 {\em{offset:}} 01 {\em{Variable type:}} UINT8 {\em{Variable data:}} 0 - RPC minor version \\
 {\em{offset:}} 02 {\em{Variable type:}} UINT8 {\em{Variable data:}} 2 - RPC response packet \\
 {\em{offset:}} 03 {\em{Variable type:}} UINT8 {\em{Variable data:}} 3 - (FirstFrag bit-wise or with LastFrag) \\
 {\em{offset:}} 04 {\em{Variable type:}} UINT32 {\em{Variable data:}} 0x1000 0000 - packed data representation \\
 {\em{offset:}} 08 {\em{Variable type:}} UINT16 {\em{Variable data:}} fragment length - data size (bytes) inc header and tail. \\
 {\em{offset:}} 0A {\em{Variable type:}} UINT16 {\em{Variable data:}} 0 - authentication length \\
 {\em{offset:}} 0C {\em{Variable type:}} UINT32 {\em{Variable data:}} call identifier. matches 12th UINT32 of incoming RPC data. \\
 {\em{offset:}} 10 {\em{Variable type:}} UINT32 {\em{Variable data:}} allocation hint - data size (bytes) minus header and tail. \\
 {\em{offset:}} 14 {\em{Variable type:}} UINT16 {\em{Variable data:}} 0 - presentation context identifier \\
 {\em{offset:}} 16 {\em{Variable type:}} UINT8 {\em{Variable data:}} 0 - cancel count \\
 {\em{offset:}} 17 {\em{Variable type:}} UINT8 {\em{Variable data:}} in replies: 0 - reserved; in requests: opnum - see \#defines. \\
 {\em{offset:}} 18 {\em{Variable type:}} ...... {\em{Variable data:}} start of data (goes on for allocation\_hint bytes)

\subsubsection{RPC\_Packet for request, response, bind and bind acknowledgement}
\label{id2509203}\hypertarget{id2509203}{}%

\begin{description}
% \null and \mbox are tricks to induce different typesetting decisions
\item[{UINT8 versionmaj}]\null{}
reply same as request (0x05)
% \null and \mbox are tricks to induce different typesetting decisions
\item[{UINT8 versionmin}]\null{}
reply same as request (0x00)
% \null and \mbox are tricks to induce different typesetting decisions
\item[{UINT8 type}]\null{}
one of the MSRPC\_Type enums
% \null and \mbox are tricks to induce different typesetting decisions
\item[{UINT8 flags}]\null{}
reply same as request (0x00 for Bind, 0x03 for Request)
% \null and \mbox are tricks to induce different typesetting decisions
\item[{UINT32 representation}]\null{}
reply same as request (0x00000010)
% \null and \mbox are tricks to induce different typesetting decisions
\item[{UINT16 fraglength}]\null{}
the length of the data section of the SMB trans packet
% \null and \mbox are tricks to induce different typesetting decisions
\item[{UINT16 authlength}]\null{}

% \null and \mbox are tricks to induce different typesetting decisions
\item[{UINT32 callid}]\null{}
call identifier. (e.g. 0x00149594)
% \null and \mbox are tricks to induce different typesetting decisions
\item[{* stub USE TvPacket}]\null{}
the remainder of the packet depending on the "type"
\end{description}
\subsubsection{Interface identification}
\label{id2509306}\hypertarget{id2509306}{}%

the interfaces are numbered. as yet I haven't seen more than one interface used on the same pipe name srvsvc


\begin{Verbatim}[]

abstract (0x4B324FC8, 0x01D31670, 0x475A7812, 0x88E16EBF, 0x00000003)
transfer (0x8A885D04, 0x11C91CEB, 0x0008E89F, 0x6048102B, 0x00000002)

\end{Verbatim}

\subsubsection{RPC\_Iface RW}
\label{id2509327}\hypertarget{id2509327}{}%

\begin{description}
% \null and \mbox are tricks to induce different typesetting decisions
\item[{UINT8 byte[16]}]\null{}
16 bytes of number
% \null and \mbox are tricks to induce different typesetting decisions
\item[{UINT32 version}]\null{}
the interface number
\end{description}
\subsubsection{RPC\_ReqBind RW}
\label{id2509355}\hypertarget{id2509355}{}%

the remainder of the packet after the header if "type" was Bind in the response header, "type" should be BindAck

\begin{description}
% \null and \mbox are tricks to induce different typesetting decisions
\item[{UINT16 maxtsize}]\null{}
maximum transmission fragment size (0x1630)
% \null and \mbox are tricks to induce different typesetting decisions
\item[{UINT16 maxrsize}]\null{}
max receive fragment size (0x1630)
% \null and \mbox are tricks to induce different typesetting decisions
\item[{UINT32 assocgid}]\null{}
associated group id (0x0)
% \null and \mbox are tricks to induce different typesetting decisions
\item[{UINT32 numelements}]\null{}
the number of elements (0x1)
% \null and \mbox are tricks to induce different typesetting decisions
\item[{UINT16 contextid}]\null{}
presentation context identifier (0x0)
% \null and \mbox are tricks to induce different typesetting decisions
\item[{UINT8 numsyntaxes}]\null{}
the number of syntaxes (has always been 1?)(0x1)
% \null and \mbox are tricks to induce different typesetting decisions
\item[{UINT8[]}]\null{}
4-byte alignment padding, against SMB header
% \null and \mbox are tricks to induce different typesetting decisions
\item[{* abstractint USE RPC\_Iface}]\null{}
num and vers. of interface client is using
% \null and \mbox are tricks to induce different typesetting decisions
\item[{* transferint USE RPC\_Iface}]\null{}
num and vers. of interface to use for replies
\end{description}
\subsubsection{RPC\_Address RW}
\label{id2509464}\hypertarget{id2509464}{}%

\begin{description}
% \null and \mbox are tricks to induce different typesetting decisions
\item[{UINT16 length}]\null{}
length of the string including null terminator
% \null and \mbox are tricks to induce different typesetting decisions
\item[{* port USE string}]\null{}
the string above in single byte, null terminated form
\end{description}
\subsubsection{RPC\_ResBind RW}
\label{id2509495}\hypertarget{id2509495}{}%

the response to place after the header in the reply packet

\begin{description}
% \null and \mbox are tricks to induce different typesetting decisions
\item[{UINT16 maxtsize}]\null{}
same as request
% \null and \mbox are tricks to induce different typesetting decisions
\item[{UINT16 maxrsize}]\null{}
same as request
% \null and \mbox are tricks to induce different typesetting decisions
\item[{UINT32 assocgid}]\null{}
zero
% \null and \mbox are tricks to induce different typesetting decisions
\item[{* secondaddr USE RPC\_Address}]\null{}
the address string, as described earlier
% \null and \mbox are tricks to induce different typesetting decisions
\item[{UINT8[]}]\null{}
4-byte alignment padding, against SMB header
% \null and \mbox are tricks to induce different typesetting decisions
\item[{UINT8 numresults}]\null{}
the number of results (0x01)
% \null and \mbox are tricks to induce different typesetting decisions
\item[{UINT8[]}]\null{}
4-byte alignment padding, against SMB header
% \null and \mbox are tricks to induce different typesetting decisions
\item[{UINT16 result}]\null{}
result (0x00 = accept)
% \null and \mbox are tricks to induce different typesetting decisions
\item[{UINT16 reason}]\null{}
reason (0x00 = no reason specified)
% \null and \mbox are tricks to induce different typesetting decisions
\item[{* transfersyntax USE RPC\_Iface}]\null{}
the transfer syntax from the request
\end{description}
\subsubsection{RPC\_ReqNorm RW}
\label{id2509611}\hypertarget{id2509611}{}%

the remainder of the packet after the header for every other other request

\begin{description}
% \null and \mbox are tricks to induce different typesetting decisions
\item[{UINT32 allochint}]\null{}
the size of the stub data in bytes
% \null and \mbox are tricks to induce different typesetting decisions
\item[{UINT16 prescontext}]\null{}
presentation context identifier (0x0)
% \null and \mbox are tricks to induce different typesetting decisions
\item[{UINT16 opnum}]\null{}
operation number (0x15)
% \null and \mbox are tricks to induce different typesetting decisions
\item[{* stub USE TvPacket}]\null{}
a packet dependent on the pipe name (probably the interface) and the op number)
\end{description}
\subsubsection{RPC\_ResNorm RW}
\label{id2509666}\hypertarget{id2509666}{}%

\begin{description}
% \null and \mbox are tricks to induce different typesetting decisions
\item[{UINT32 allochint}]\null{}
\# size of the stub data in bytes
% \null and \mbox are tricks to induce different typesetting decisions
\item[{UINT16 prescontext}]\null{}
\# presentation context identifier (same as request)
% \null and \mbox are tricks to induce different typesetting decisions
\item[{UINT8 cancelcount}]\null{}
\# cancel count? (0x0)
% \null and \mbox are tricks to induce different typesetting decisions
\item[{UINT8 reserved}]\null{}
\# 0 - one byte padding
% \null and \mbox are tricks to induce different typesetting decisions
\item[{* stub USE TvPacket}]\null{}
\# the remainder of the reply
\end{description}
\subsection{Tail}
\label{id2509727}\hypertarget{id2509727}{}%

The end of each of the NTLSA and NETLOGON named pipes ends with:

\begin{description}
% \null and \mbox are tricks to induce different typesetting decisions
\item[{......}]\null{}
end of data
% \null and \mbox are tricks to induce different typesetting decisions
\item[{UINT32}]\null{}
return code
\end{description}
\subsection{RPC Bind / Bind Ack}
\label{id2509763}\hypertarget{id2509763}{}%

RPC Binds are the process of associating an RPC pipe (e.g \textbackslash PIPE\textbackslash lsarpc) with a "transfer syntax" (see RPC\_Iface structure). The purpose for doing this is unknown.

{\em{Note: The RPC\_ResBind SMB Transact request is sent with two uint16 setup parameters. The first is 0x0026; the second is the file handle returned by the SMBopenX Transact response.}}

{\em{Note: The RPC\_ResBind members maxtsize, maxrsize and assocgid are the same in the response as the same members in the RPC\_ReqBind. The RPC\_ResBind member transfersyntax is the same in the response as the}}

{\em{Note: The RPC\_ResBind response member secondaddr contains the name of what is presumed to be the service behind the RPC pipe. The mapping identified so far is:}}

\begin{description}
% \null and \mbox are tricks to induce different typesetting decisions
\item[{initial SMBopenX request:}]\null{}
RPC\_ResBind response:
% \null and \mbox are tricks to induce different typesetting decisions
\item[{"\textbackslash \textbackslash PIPE\textbackslash \textbackslash srvsvc"}]\null{}
"\textbackslash \textbackslash PIPE\textbackslash \textbackslash ntsvcs"
% \null and \mbox are tricks to induce different typesetting decisions
\item[{"\textbackslash \textbackslash PIPE\textbackslash \textbackslash samr"}]\null{}
"\textbackslash \textbackslash PIPE\textbackslash \textbackslash lsass"
% \null and \mbox are tricks to induce different typesetting decisions
\item[{"\textbackslash \textbackslash PIPE\textbackslash \textbackslash lsarpc"}]\null{}
"\textbackslash \textbackslash PIPE\textbackslash \textbackslash lsass"
% \null and \mbox are tricks to induce different typesetting decisions
\item[{"\textbackslash \textbackslash PIPE\textbackslash \textbackslash wkssvc"}]\null{}
"\textbackslash \textbackslash PIPE\textbackslash \textbackslash wksvcs"
% \null and \mbox are tricks to induce different typesetting decisions
\item[{"\textbackslash \textbackslash PIPE\textbackslash \textbackslash NETLOGON"}]\null{}
"\textbackslash \textbackslash PIPE\textbackslash \textbackslash NETLOGON"
\end{description}

{\em{Note: The RPC\_Packet fraglength member in both the Bind Request and Bind Acknowledgment must contain the length of the entire RPC data, including the RPC\_Packet header.}}

Request:

\begin{tabular}{l}

RPC\_Packet  \\
RPC\_ReqBind  \\
\end{tabular}

Response:

\begin{tabular}{l}

RPC\_Packet  \\
RPC\_ResBind  \\
\end{tabular}
\subsection{NTLSA Transact Named Pipe}
\label{id2509902}\hypertarget{id2509902}{}%

The sequence of actions taken on this pipe are:

\begin{tabular}{l}

Establish a connection to the IPC\$ share (SMBtconX). use encrypted passwords.  \\
Open an RPC Pipe with the name "\textbackslash \textbackslash PIPE\textbackslash \textbackslash lsarpc". Store the file handle.  \\
Using the file handle, send a Set Named Pipe Handle state to 0x4300.  \\
Send an LSA Open Policy request. Store the Policy Handle.  \\
Using the Policy Handle, send LSA Query Info Policy requests, etc.  \\
Using the Policy Handle, send an LSA Close.  \\
Close the IPC\$ share.  \\
\end{tabular}

Defines for this pipe, identifying the query are:

\begin{description}
% \null and \mbox are tricks to induce different typesetting decisions
\item[{LSA Open Policy:}]\null{}
0x2c
% \null and \mbox are tricks to induce different typesetting decisions
\item[{LSA Query Info Policy:}]\null{}
0x07
% \null and \mbox are tricks to induce different typesetting decisions
\item[{LSA Enumerate Trusted Domains:}]\null{}
0x0d
% \null and \mbox are tricks to induce different typesetting decisions
\item[{LSA Open Secret:}]\null{}
0xff
% \null and \mbox are tricks to induce different typesetting decisions
\item[{LSA Lookup SIDs:}]\null{}
0xfe
% \null and \mbox are tricks to induce different typesetting decisions
\item[{LSA Lookup Names:}]\null{}
0xfd
% \null and \mbox are tricks to induce different typesetting decisions
\item[{LSA Close:}]\null{}
0x00
\end{description}
\subsection{LSA Open Policy}
\label{id2510030}\hypertarget{id2510030}{}%

{\em{Note: The policy handle can be anything you like.}}
\subsubsection{Request}
\label{id2510040}\hypertarget{id2510040}{}%

\begin{description}
% \null and \mbox are tricks to induce different typesetting decisions
\item[{VOID*}]\null{}
buffer pointer
% \null and \mbox are tricks to induce different typesetting decisions
\item[{UNISTR2}]\null{}
server name - unicode string starting with two '\textbackslash 's
% \null and \mbox are tricks to induce different typesetting decisions
\item[{OBJ\_ATTR}]\null{}
object attributes
% \null and \mbox are tricks to induce different typesetting decisions
\item[{UINT32}]\null{}
1 - desired access
\end{description}
\subsubsection{Response}
\label{id2510093}\hypertarget{id2510093}{}%

\begin{description}
% \null and \mbox are tricks to induce different typesetting decisions
\item[{POL\_HND}]\null{}
LSA policy handle
% \null and \mbox are tricks to induce different typesetting decisions
\item[{return}]\null{}
0 - indicates success
\end{description}
\subsection{LSA Query Info Policy}
\label{id2510125}\hypertarget{id2510125}{}%

{\em{Note: The info class in response must be the same as that in the request.}}
\subsubsection{Request}
\label{id2510136}\hypertarget{id2510136}{}%

\begin{description}
% \null and \mbox are tricks to induce different typesetting decisions
\item[{POL\_HND}]\null{}
LSA policy handle
% \null and \mbox are tricks to induce different typesetting decisions
\item[{UINT16}]\null{}
info class (also a policy handle?)
\end{description}
\subsubsection{Response}
\label{id2510168}\hypertarget{id2510168}{}%

\begin{description}
% \null and \mbox are tricks to induce different typesetting decisions
\item[{VOID*}]\null{}
undocumented buffer pointer
% \null and \mbox are tricks to induce different typesetting decisions
\item[{UINT16}]\null{}
info class (same as info class in request).
\end{description}


\begin{Verbatim}[]

switch (info class)
case 3:
case 5:
{
DOM_INFO domain info, levels 3 and 5 (are the same).
}

return    0 - indicates success

\end{Verbatim}

\subsection{LSA Enumerate Trusted Domains}
\label{id2510215}\hypertarget{id2510215}{}%
\subsubsection{Request}
\label{id2510221}\hypertarget{id2510221}{}%

no extra data
\subsubsection{Response}
\label{id2510233}\hypertarget{id2510233}{}%

\begin{description}
% \null and \mbox are tricks to induce different typesetting decisions
\item[{UINT32}]\null{}
0 - enumeration context
% \null and \mbox are tricks to induce different typesetting decisions
\item[{UINT32}]\null{}
0 - entries read
% \null and \mbox are tricks to induce different typesetting decisions
\item[{UINT32}]\null{}
0 - trust information
% \null and \mbox are tricks to induce different typesetting decisions
\item[{return}]\null{}
0x8000 001a - "no trusted domains" success code
\end{description}
\subsection{LSA Open Secret}
\label{id2510296}\hypertarget{id2510296}{}%
\subsubsection{Request}
\label{id2510302}\hypertarget{id2510302}{}%

no extra data
\subsubsection{Response}
\label{id2510313}\hypertarget{id2510313}{}%

\begin{description}
% \null and \mbox are tricks to induce different typesetting decisions
\item[{UINT32}]\null{}
0 - undocumented
% \null and \mbox are tricks to induce different typesetting decisions
\item[{UINT32}]\null{}
0 - undocumented
% \null and \mbox are tricks to induce different typesetting decisions
\item[{UINT32}]\null{}
0 - undocumented
% \null and \mbox are tricks to induce different typesetting decisions
\item[{UINT32}]\null{}
0 - undocumented
% \null and \mbox are tricks to induce different typesetting decisions
\item[{UINT32}]\null{}
0 - undocumented
\end{description}

return 0x0C00 0034 - "no such secret" success code
\subsection{LSA Close}
\label{id2510392}\hypertarget{id2510392}{}%
\subsubsection{Request}
\label{id2510398}\hypertarget{id2510398}{}%

\begin{description}
% \null and \mbox are tricks to induce different typesetting decisions
\item[{POL\_HND}]\null{}
policy handle to be closed
\end{description}
\subsubsection{Response}
\label{id2510422}\hypertarget{id2510422}{}%

\begin{description}
% \null and \mbox are tricks to induce different typesetting decisions
\item[{POL\_HND}]\null{}
0s - closed policy handle (all zeros)
\end{description}

return 0 - indicates success
\subsection{LSA Lookup SIDS}
\label{id2510450}\hypertarget{id2510450}{}%

{\em{Note: num\_entries in response must be same as num\_entries in request.}}
\subsubsection{Request}
\label{id2510462}\hypertarget{id2510462}{}%

\begin{description}
% \null and \mbox are tricks to induce different typesetting decisions
\item[{POL\_HND}]\null{}
LSA policy handle
% \null and \mbox are tricks to induce different typesetting decisions
\item[{UINT32}]\null{}
num\_entries
% \null and \mbox are tricks to induce different typesetting decisions
\item[{VOID*}]\null{}
undocumented domain SID buffer pointer
% \null and \mbox are tricks to induce different typesetting decisions
\item[{VOID*}]\null{}
undocumented domain name buffer pointer
% \null and \mbox are tricks to induce different typesetting decisions
\item[{VOID*[num\_entries] undocumented domain SID pointers to be looked up.}]\null{}
DOM\_SID[num\_entries] domain SIDs to be looked up.
% \null and \mbox are tricks to induce different typesetting decisions
\item[{char[16]}]\null{}
completely undocumented 16 bytes.
\end{description}
\subsubsection{Response}
\label{id2510549}\hypertarget{id2510549}{}%

\begin{description}
% \null and \mbox are tricks to induce different typesetting decisions
\item[{DOM\_REF}]\null{}
domain reference response
% \null and \mbox are tricks to induce different typesetting decisions
\item[{UINT32}]\null{}
num\_entries (listed above)
% \null and \mbox are tricks to induce different typesetting decisions
\item[{VOID*}]\null{}
undocumented buffer pointer
% \null and \mbox are tricks to induce different typesetting decisions
\item[{UINT32}]\null{}
num\_entries (listed above)
% \null and \mbox are tricks to induce different typesetting decisions
\item[{DOM\_SID2[num\_entries]}]\null{}
domain SIDs (from Request, listed above).
% \null and \mbox are tricks to induce different typesetting decisions
\item[{UINT32}]\null{}
num\_entries (listed above)
\end{description}

return 0 - indicates success
\subsection{LSA Lookup Names}
\label{id2510634}\hypertarget{id2510634}{}%

{\em{Note: num\_entries in response must be same as num\_entries in request.}}
\subsubsection{Request}
\label{id2510646}\hypertarget{id2510646}{}%

\begin{description}
% \null and \mbox are tricks to induce different typesetting decisions
\item[{POL\_HND}]\null{}
LSA policy handle
% \null and \mbox are tricks to induce different typesetting decisions
\item[{UINT32}]\null{}
num\_entries
% \null and \mbox are tricks to induce different typesetting decisions
\item[{UINT32}]\null{}
num\_entries
% \null and \mbox are tricks to induce different typesetting decisions
\item[{VOID*}]\null{}
undocumented domain SID buffer pointer
% \null and \mbox are tricks to induce different typesetting decisions
\item[{VOID*}]\null{}
undocumented domain name buffer pointer
% \null and \mbox are tricks to induce different typesetting decisions
\item[{NAME[num\_entries]}]\null{}
names to be looked up.
% \null and \mbox are tricks to induce different typesetting decisions
\item[{char[]}]\null{}
undocumented bytes - falsely translated SID structure?
\end{description}
\subsubsection{Response}
\label{id2510745}\hypertarget{id2510745}{}%

\begin{description}
% \null and \mbox are tricks to induce different typesetting decisions
\item[{DOM\_REF}]\null{}
domain reference response
% \null and \mbox are tricks to induce different typesetting decisions
\item[{UINT32}]\null{}
num\_entries (listed above)
% \null and \mbox are tricks to induce different typesetting decisions
\item[{VOID*}]\null{}
undocumented buffer pointer
% \null and \mbox are tricks to induce different typesetting decisions
\item[{UINT32}]\null{}
num\_entries (listed above)
% \null and \mbox are tricks to induce different typesetting decisions
\item[{DOM\_RID[num\_entries]}]\null{}
domain SIDs (from Request, listed above).
% \null and \mbox are tricks to induce different typesetting decisions
\item[{UINT32}]\null{}
num\_entries (listed above)
\end{description}

return 0 - indicates success

% ------------------------   
% Section 
\section{NETLOGON rpc Transact Named Pipe}
\label{id2510831}\hypertarget{id2510831}{}%

The sequence of actions taken on this pipe are:

\begin{tabular}{l}

tablish a connection to the IPC\$ share (SMBtconX). use encrypted passwords.  \\
en an RPC Pipe with the name "\textbackslash \textbackslash PIPE\textbackslash \textbackslash NETLOGON". Store the file handle.  \\
ing the file handle, send a Set Named Pipe Handle state to 0x4300.  \\
eate Client Challenge. Send LSA Request Challenge. Store Server Challenge.  \\
lculate Session Key. Send an LSA Auth 2 Challenge. Store Auth2 Challenge.  \\
lc/Verify Client Creds. Send LSA Srv PW Set. Calc/Verify Server Creds.  \\
lc/Verify Client Creds. Send LSA SAM Logon . Calc/Verify Server Creds.  \\
lc/Verify Client Creds. Send LSA SAM Logoff. Calc/Verify Server Creds.  \\
ose the IPC\$ share.  \\
\end{tabular}

Defines for this pipe, identifying the query are

\begin{description}
% \null and \mbox are tricks to induce different typesetting decisions
\item[{LSA Request Challenge:}]\null{}
0x04
% \null and \mbox are tricks to induce different typesetting decisions
\item[{LSA Server Password Set:}]\null{}
0x06
% \null and \mbox are tricks to induce different typesetting decisions
\item[{LSA SAM Logon:}]\null{}
0x02
% \null and \mbox are tricks to induce different typesetting decisions
\item[{LSA SAM Logoff:}]\null{}
0x03
% \null and \mbox are tricks to induce different typesetting decisions
\item[{LSA Auth 2:}]\null{}
0x0f
% \null and \mbox are tricks to induce different typesetting decisions
\item[{LSA Logon Control:}]\null{}
0x0e
\end{description}
\subsection{LSA Request Challenge}
\label{id2510973}\hypertarget{id2510973}{}%

{\em{Note: logon server name starts with two '\textbackslash ' characters and is upper case.}}

{\em{Note: logon client is the machine, not the user.}}

{\em{Note: the initial LanManager password hash, against which the challenge is issued, is the machine name itself (lower case). there will becalls issued (LSA Server Password Set) which will change this, later. refusing these calls allows you to always deal with the same password (i.e the LM\# of the machine name in lower case).}}
\subsubsection{Request}
\label{id2511001}\hypertarget{id2511001}{}%

\begin{description}
% \null and \mbox are tricks to induce different typesetting decisions
\item[{VOID*}]\null{}
undocumented buffer pointer
% \null and \mbox are tricks to induce different typesetting decisions
\item[{UNISTR2}]\null{}
logon server unicode string
% \null and \mbox are tricks to induce different typesetting decisions
\item[{UNISTR2}]\null{}
logon client unicode string
% \null and \mbox are tricks to induce different typesetting decisions
\item[{char[8]}]\null{}
client challenge
\end{description}
\subsubsection{Response}
\label{id2511062}\hypertarget{id2511062}{}%

\begin{description}
% \null and \mbox are tricks to induce different typesetting decisions
\item[{char[8]}]\null{}
server challenge
\end{description}

return 0 - indicates success
\subsection{LSA Authenticate 2}
\label{id2511091}\hypertarget{id2511091}{}%

{\em{Note: in between request and response, calculate the client credentials, and check them against the client-calculated credentials (this process uses the previously received client credentials).}}

{\em{Note: neg\_flags in the response is the same as that in the request.}}

{\em{Note: you must take a copy of the client-calculated credentials received here, because they will be used in subsequent authentication packets.}}
\subsubsection{Request}
\label{id2511118}\hypertarget{id2511118}{}%

\begin{description}
% \null and \mbox are tricks to induce different typesetting decisions
\item[{LOG\_INFO}]\null{}
client identification info
% \null and \mbox are tricks to induce different typesetting decisions
\item[{char[8]}]\null{}
client-calculated credentials
% \null and \mbox are tricks to induce different typesetting decisions
\item[{UINT8[]}]\null{}
padding to 4-byte align with start of SMB header.
% \null and \mbox are tricks to induce different typesetting decisions
\item[{UINT32}]\null{}
neg\_flags - negotiated flags (usual value is 0x0000 01ff)
\end{description}
\subsubsection{Response}
\label{id2511178}\hypertarget{id2511178}{}%

\begin{description}
% \null and \mbox are tricks to induce different typesetting decisions
\item[{char[8]}]\null{}
server credentials.
% \null and \mbox are tricks to induce different typesetting decisions
\item[{UINT32}]\null{}
neg\_flags - same as neg\_flags in request.
\end{description}

return 0 - indicates success. failure value unknown.
\subsection{LSA Server Password Set}
\label{id2511220}\hypertarget{id2511220}{}%

{\em{Note: the new password is suspected to be a DES encryption using the old password to generate the key.}}

{\em{Note: in between request and response, calculate the client credentials, and check them against the client-calculated credentials (this process uses the previously received client credentials).}}

{\em{Note: the server credentials are constructed from the client-calculated credentials and the client time + 1 second.}}

{\em{Note: you must take a copy of the client-calculated credentials received here, because they will be used in subsequent authentication packets.}}
\subsubsection{Request}
\label{id2511254}\hypertarget{id2511254}{}%

\begin{description}
% \null and \mbox are tricks to induce different typesetting decisions
\item[{CLNT\_INFO}]\null{}
client identification/authentication info
% \null and \mbox are tricks to induce different typesetting decisions
\item[{char[]}]\null{}
new password - undocumented.
\end{description}
\subsubsection{Response}
\label{id2511291}\hypertarget{id2511291}{}%

\begin{description}
% \null and \mbox are tricks to induce different typesetting decisions
\item[{CREDS}]\null{}
server credentials. server time stamp appears to be ignored.
\end{description}

return 0 - indicates success; 0xC000 006a indicates failure
\subsection{LSA SAM Logon}
\label{id2511321}\hypertarget{id2511321}{}%

{\em{Note: valid\_user is True iff the username and password hash are valid for the requested domain.}}
\subsubsection{Request}
\label{id2511334}\hypertarget{id2511334}{}%

\begin{description}
% \null and \mbox are tricks to induce different typesetting decisions
\item[{SAM\_INFO}]\null{}
sam\_id structure
\end{description}
\subsubsection{Response}
\label{id2511358}\hypertarget{id2511358}{}%

\begin{description}
% \null and \mbox are tricks to induce different typesetting decisions
\item[{VOID*}]\null{}
undocumented buffer pointer
% \null and \mbox are tricks to induce different typesetting decisions
\item[{CREDS}]\null{}
server credentials. server time stamp appears to be ignored.
\end{description}


\begin{Verbatim}[]

if (valid_user)
{
	UINT16      3 - switch value indicating USER_INFO structure.
    VOID*     non-zero - pointer to USER_INFO structure
    USER_INFO user logon information

    UINT32    1 - Authoritative response; 0 - Non-Auth?

    return    0 - indicates success
}
else
{
	UINT16    0 - switch value.  value to indicate no user presumed.
    VOID*     0x0000 0000 - indicates no USER_INFO structure.

    UINT32    1 - Authoritative response; 0 - Non-Auth?

    return    0xC000 0064 - NT_STATUS_NO_SUCH_USER.
}

\end{Verbatim}

\subsection{LSA SAM Logoff}
\label{id2511412}\hypertarget{id2511412}{}%

{\em{Note: presumably, the SAM\_INFO structure is validated, and a (currently undocumented) error code returned if the Logoff is invalid.}}
\subsubsection{Request}
\label{id2511425}\hypertarget{id2511425}{}%

\begin{description}
% \null and \mbox are tricks to induce different typesetting decisions
\item[{SAM\_INFO}]\null{}
sam\_id structure
\end{description}
\subsubsection{Response}
\label{id2511448}\hypertarget{id2511448}{}%

\begin{description}
% \null and \mbox are tricks to induce different typesetting decisions
\item[{VOID*}]\null{}
undocumented buffer pointer
% \null and \mbox are tricks to induce different typesetting decisions
\item[{CREDS}]\null{}
server credentials. server time stamp appears to be ignored.
\end{description}

return 0 - indicates success. undocumented failure indication.

% ------------------------   
% Section 
\section{\textbackslash \textbackslash MAILSLOT\textbackslash NET\textbackslash NTLOGON}
\label{id2511493}\hypertarget{id2511493}{}%

{\em{Note: mailslots will contain a response mailslot, to which the response should be sent. the target NetBIOS name is REQUEST\_NAME\textless{}20\textgreater{}, where REQUEST\_NAME is the name of the machine that sent the request.}}
\subsection{Query for PDC}
\label{id2511507}\hypertarget{id2511507}{}%

{\em{Note: NTversion, LMNTtoken, LM20token in response are the same as those given in the request.}}
\subsubsection{Request}
\label{id2511520}\hypertarget{id2511520}{}%

\begin{description}
% \null and \mbox are tricks to induce different typesetting decisions
\item[{UINT16}]\null{}
0x0007 - Query for PDC
% \null and \mbox are tricks to induce different typesetting decisions
\item[{STR}]\null{}
machine name
% \null and \mbox are tricks to induce different typesetting decisions
\item[{STR}]\null{}
response mailslot
% \null and \mbox are tricks to induce different typesetting decisions
\item[{UINT8[]}]\null{}
padding to 2-byte align with start of mailslot.
% \null and \mbox are tricks to induce different typesetting decisions
\item[{UNISTR}]\null{}
machine name
% \null and \mbox are tricks to induce different typesetting decisions
\item[{UINT32}]\null{}
NTversion
% \null and \mbox are tricks to induce different typesetting decisions
\item[{UINT16}]\null{}
LMNTtoken
% \null and \mbox are tricks to induce different typesetting decisions
\item[{UINT16}]\null{}
LM20token
\end{description}
\subsubsection{Response}
\label{id2511630}\hypertarget{id2511630}{}%

\begin{description}
% \null and \mbox are tricks to induce different typesetting decisions
\item[{UINT16}]\null{}
0x000A - Respose to Query for PDC
% \null and \mbox are tricks to induce different typesetting decisions
\item[{STR}]\null{}
machine name (in uppercase)
% \null and \mbox are tricks to induce different typesetting decisions
\item[{UINT8[]}]\null{}
padding to 2-byte align with start of mailslot.
% \null and \mbox are tricks to induce different typesetting decisions
\item[{UNISTR}]\null{}
machine name
% \null and \mbox are tricks to induce different typesetting decisions
\item[{UNISTR}]\null{}
domain name
% \null and \mbox are tricks to induce different typesetting decisions
\item[{UINT32}]\null{}
NTversion (same as received in request)
% \null and \mbox are tricks to induce different typesetting decisions
\item[{UINT16}]\null{}
LMNTtoken (same as received in request)
% \null and \mbox are tricks to induce different typesetting decisions
\item[{UINT16}]\null{}
LM20token (same as received in request)
\end{description}
\subsection{SAM Logon}
\label{id2511736}\hypertarget{id2511736}{}%

{\em{Note: machine name in response is preceded by two '\textbackslash ' characters.}}

{\em{Note: NTversion, LMNTtoken, LM20token in response are the same as those given in the request.}}

{\em{Note: user name in the response is presumably the same as that in the request.}}
\subsubsection{Request}
\label{id2511760}\hypertarget{id2511760}{}%

\begin{description}
% \null and \mbox are tricks to induce different typesetting decisions
\item[{UINT16}]\null{}
0x0012 - SAM Logon
% \null and \mbox are tricks to induce different typesetting decisions
\item[{UINT16}]\null{}
request count
% \null and \mbox are tricks to induce different typesetting decisions
\item[{UNISTR}]\null{}
machine name
% \null and \mbox are tricks to induce different typesetting decisions
\item[{UNISTR}]\null{}
user name
% \null and \mbox are tricks to induce different typesetting decisions
\item[{STR}]\null{}
response mailslot
% \null and \mbox are tricks to induce different typesetting decisions
\item[{UINT32}]\null{}
alloweable account
% \null and \mbox are tricks to induce different typesetting decisions
\item[{UINT32}]\null{}
domain SID size
% \null and \mbox are tricks to induce different typesetting decisions
\item[{char[sid\_size]}]\null{}
domain SID, of sid\_size bytes.
% \null and \mbox are tricks to induce different typesetting decisions
\item[{UINT8[]}]\null{}
???? padding to 4? 2? -byte align with start of mailslot.
% \null and \mbox are tricks to induce different typesetting decisions
\item[{UINT32}]\null{}
NTversion
% \null and \mbox are tricks to induce different typesetting decisions
\item[{UINT16}]\null{}
LMNTtoken
% \null and \mbox are tricks to induce different typesetting decisions
\item[{UINT16}]\null{}
LM20token
\end{description}
\subsubsection{Response}
\label{id2511921}\hypertarget{id2511921}{}%

\begin{description}
% \null and \mbox are tricks to induce different typesetting decisions
\item[{UINT16}]\null{}
0x0013 - Response to SAM Logon
% \null and \mbox are tricks to induce different typesetting decisions
\item[{UNISTR}]\null{}
machine name
% \null and \mbox are tricks to induce different typesetting decisions
\item[{UNISTR}]\null{}
user name - workstation trust account
% \null and \mbox are tricks to induce different typesetting decisions
\item[{UNISTR}]\null{}
domain name
% \null and \mbox are tricks to induce different typesetting decisions
\item[{UINT32}]\null{}
NTversion
% \null and \mbox are tricks to induce different typesetting decisions
\item[{UINT16}]\null{}
LMNTtoken
% \null and \mbox are tricks to induce different typesetting decisions
\item[{UINT16}]\null{}
LM20token
\end{description}

% ------------------------   
% Section 
\section{SRVSVC Transact Named Pipe}
\label{id2512022}\hypertarget{id2512022}{}%

Defines for this pipe, identifying the query are:

\begin{description}
% \null and \mbox are tricks to induce different typesetting decisions
\item[{Net Share Enum}]\null{}
0x0f
% \null and \mbox are tricks to induce different typesetting decisions
\item[{Net Server Get Info}]\null{}
0x15
\end{description}
\subsection{Net Share Enum}
\label{id2512061}\hypertarget{id2512061}{}%

{\em{Note: share level and switch value in the response are presumably the same as those in the request.}}

{\em{Note: cifsrap2.txt (section 5) may be of limited assistance here.}}
\subsubsection{Request}
\label{id2512079}\hypertarget{id2512079}{}%

\begin{description}
% \null and \mbox are tricks to induce different typesetting decisions
\item[{VOID*}]\null{}
pointer (to server name?)
% \null and \mbox are tricks to induce different typesetting decisions
\item[{UNISTR2}]\null{}
server name
% \null and \mbox are tricks to induce different typesetting decisions
\item[{UINT8[]}]\null{}
padding to get unicode string 4-byte aligned with the start of the SMB header.
% \null and \mbox are tricks to induce different typesetting decisions
\item[{UINT32}]\null{}
share level
% \null and \mbox are tricks to induce different typesetting decisions
\item[{UINT32}]\null{}
switch value
% \null and \mbox are tricks to induce different typesetting decisions
\item[{VOID*}]\null{}
pointer to SHARE\_INFO\_1\_CTR
% \null and \mbox are tricks to induce different typesetting decisions
\item[{SHARE\_INFO\_1\_CTR}]\null{}
share info with 0 entries
% \null and \mbox are tricks to induce different typesetting decisions
\item[{UINT32}]\null{}
preferred maximum length (0xffff ffff)
\end{description}
\subsubsection{Response}
\label{id2512189}\hypertarget{id2512189}{}%

\begin{description}
% \null and \mbox are tricks to induce different typesetting decisions
\item[{UINT32}]\null{}
share level
% \null and \mbox are tricks to induce different typesetting decisions
\item[{UINT32}]\null{}
switch value
% \null and \mbox are tricks to induce different typesetting decisions
\item[{VOID*}]\null{}
pointer to SHARE\_INFO\_1\_CTR
% \null and \mbox are tricks to induce different typesetting decisions
\item[{SHARE\_INFO\_1\_CTR}]\null{}
share info (only added if share info ptr is non-zero)
\end{description}

return 0 - indicates success
\subsection{Net Server Get Info}
\label{id2512253}\hypertarget{id2512253}{}%

{\em{Note: level is the same value as in the request.}}
\subsubsection{Request}
\label{id2512265}\hypertarget{id2512265}{}%

\begin{description}
% \null and \mbox are tricks to induce different typesetting decisions
\item[{UNISTR2}]\null{}
server name
% \null and \mbox are tricks to induce different typesetting decisions
\item[{UINT32}]\null{}
switch level
\end{description}
\subsubsection{Response}
\label{id2512301}\hypertarget{id2512301}{}%

\begin{description}
% \null and \mbox are tricks to induce different typesetting decisions
\item[{UINT32}]\null{}
switch level
% \null and \mbox are tricks to induce different typesetting decisions
\item[{VOID*}]\null{}
pointer to SERVER\_INFO\_101
% \null and \mbox are tricks to induce different typesetting decisions
\item[{SERVER\_INFO\_101}]\null{}
server info (only added if server info ptr is non-zero)
\end{description}

return 0 - indicates success

% ------------------------   
% Section 
\section{Cryptographic side of NT Domain Authentication}
\label{id2512356}\hypertarget{id2512356}{}%
\subsection{Definitions}
\label{id2512363}\hypertarget{id2512363}{}%

\begin{description}
% \null and \mbox are tricks to induce different typesetting decisions
\item[{Add(A1,A2)}]\null{}
Intel byte ordered addition of corresponding 4 byte words in arrays A1 and A2
% \null and \mbox are tricks to induce different typesetting decisions
\item[{E(K,D)}]\null{}
DES ECB encryption of 8 byte data D using 7 byte key K
% \null and \mbox are tricks to induce different typesetting decisions
\item[{lmowf()}]\null{}
Lan man hash
% \null and \mbox are tricks to induce different typesetting decisions
\item[{ntowf()}]\null{}
NT hash
% \null and \mbox are tricks to induce different typesetting decisions
\item[{PW}]\null{}
md4(machine\_password) == md4(lsadump \$machine.acc) == pwdump(machine\$) (initially) == md4(lmowf(unicode(machine)))
% \null and \mbox are tricks to induce different typesetting decisions
\item[{ARC4(K,Lk,D,Ld)}]\null{}
ARC4 encryption of data D of length Ld with key K of length Lk
% \null and \mbox are tricks to induce different typesetting decisions
\item[{v[m..n(,l)]}]\null{}
subset of v from bytes m to n, optionally padded with zeroes to length l
% \null and \mbox are tricks to induce different typesetting decisions
\item[{Cred(K,D)}]\null{}
E(K[7..7,7],E(K[0..6],D)) computes a credential
% \null and \mbox are tricks to induce different typesetting decisions
\item[{Time()}]\null{}
4 byte current time
% \null and \mbox are tricks to induce different typesetting decisions
\item[{Cc,Cs}]\null{}
8 byte client and server challenges Rc,Rs: 8 byte client and server credentials
\end{description}
\subsection{Protocol}
\label{id2512505}\hypertarget{id2512505}{}%

\begin{Verbatim}[]

C->S ReqChal,Cc
S->C Cs

\end{Verbatim}

\begin{Verbatim}[]

C & S compute session key Ks = E(PW[9..15],E(PW[0..6],Add(Cc,Cs)))

\end{Verbatim}

\begin{Verbatim}[]

C: Rc = Cred(Ks,Cc)
C->S Authenticate,Rc
S: Rs = Cred(Ks,Cs), assert(Rc == Cred(Ks,Cc))
S->C Rs
C: assert(Rs == Cred(Ks,Cs))

\end{Verbatim}

On joining the domain the client will optionally attempt to change its password and the domain controller may refuse to update it depending on registry settings. This will also occur weekly afterwards.

\begin{Verbatim}[]

C: Tc = Time(), Rc' = Cred(Ks,Rc+Tc)
C->S ServerPasswordSet,Rc',Tc,arc4(Ks[0..7,16],lmowf(randompassword())
C: Rc = Cred(Ks,Rc+Tc+1)
S: assert(Rc' == Cred(Ks,Rc+Tc)), Ts = Time()
S: Rs' = Cred(Ks,Rs+Tc+1)
S->C Rs',Ts
C: assert(Rs' == Cred(Ks,Rs+Tc+1))
S: Rs = Rs'

\end{Verbatim}

User: U with password P wishes to login to the domain (incidental data such as workstation and domain omitted)

\begin{Verbatim}[]

C: Tc = Time(), Rc' = Cred(Ks,Rc+Tc)
C->S NetLogonSamLogon,Rc',Tc,U,arc4(Ks[0..7,16],16,ntowf(P),16), arc4(Ks[0..7,16],16,lmowf(P),16)
S: assert(Rc' == Cred(Ks,Rc+Tc)) assert(passwords match those in SAM)
S: Ts = Time()

\end{Verbatim}

\begin{Verbatim}[]

S->C Cred(Ks,Cred(Ks,Rc+Tc+1)),userinfo(logon script,UID,SIDs,etc)
C: assert(Rs == Cred(Ks,Cred(Rc+Tc+1))
C: Rc = Cred(Ks,Rc+Tc+1)

\end{Verbatim}
\subsection{Comments}
\label{id2512575}\hypertarget{id2512575}{}%

On first joining the domain the session key could be computed by anyone listening in on the network as the machine password has a well known value. Until the machine is rebooted it will use this session key to encrypt NT and LM one way functions of passwords which are password equivalents. Any user who logs in before the machine has been rebooted a second time will have their password equivalent exposed. Of course the new machine password is exposed at this time anyway.

None of the returned user info such as logon script, profile path and SIDs *appear* to be protected by anything other than the TCP checksum.

The server time stamps appear to be ignored.

The client sends a ReturnAuthenticator in the SamLogon request which I can't find a use for. However its time is used as the timestamp returned by the server.

The password OWFs should NOT be sent over the network reversibly encrypted. They should be sent using ARC4(Ks,md4(owf)) with the server computing the same function using the owf values in the SAM.

% ------------------------   
% Section 
\section{SIDs and RIDs}
\label{id2512619}\hypertarget{id2512619}{}%

SIDs and RIDs are well documented elsewhere.

A SID is an NT Security ID (see DOM\_SID structure). They are of the form:

\begin{tabular}{l}

revision-NN-SubAuth1-SubAuth2-SubAuth3...  \\
revision-0xNNNNNNNNNNNN-SubAuth1-SubAuth2-SubAuth3...  \\
\end{tabular}

currently, the SID revision is 1. The Sub-Authorities are known as Relative IDs (RIDs).
\subsection{Well-known SIDs}
\label{id2512654}\hypertarget{id2512654}{}%
\subsubsection{Universal well-known SIDs}
\label{id2512660}\hypertarget{id2512660}{}%

\begin{description}
% \null and \mbox are tricks to induce different typesetting decisions
\item[{Null SID}]\null{}
S-1-0-0
% \null and \mbox are tricks to induce different typesetting decisions
\item[{World}]\null{}
S-1-1-0
% \null and \mbox are tricks to induce different typesetting decisions
\item[{Local}]\null{}
S-1-2-0
% \null and \mbox are tricks to induce different typesetting decisions
\item[{Creator Owner ID}]\null{}
S-1-3-0
% \null and \mbox are tricks to induce different typesetting decisions
\item[{Creator Group ID}]\null{}
S-1-3-1
% \null and \mbox are tricks to induce different typesetting decisions
\item[{Creator Owner Server ID}]\null{}
S-1-3-2
% \null and \mbox are tricks to induce different typesetting decisions
\item[{Creator Group Server ID}]\null{}
S-1-3-3
% \null and \mbox are tricks to induce different typesetting decisions
\item[{(Non-unique IDs)}]\null{}
S-1-4
\end{description}
\subsubsection{NT well-known SIDs}
\label{id2512771}\hypertarget{id2512771}{}%

\begin{description}
% \null and \mbox are tricks to induce different typesetting decisions
\item[{NT Authority}]\null{}
S-1-5
% \null and \mbox are tricks to induce different typesetting decisions
\item[{Dialup}]\null{}
S-1-5-1
% \null and \mbox are tricks to induce different typesetting decisions
\item[{Network}]\null{}
S-1-5-2
% \null and \mbox are tricks to induce different typesetting decisions
\item[{Batch}]\null{}
S-1-5-3
% \null and \mbox are tricks to induce different typesetting decisions
\item[{Interactive}]\null{}
S-1-5-4
% \null and \mbox are tricks to induce different typesetting decisions
\item[{Service}]\null{}
S-1-5-6
% \null and \mbox are tricks to induce different typesetting decisions
\item[{AnonymousLogon(aka null logon session)}]\null{}
S-1-5-7
% \null and \mbox are tricks to induce different typesetting decisions
\item[{Proxy}]\null{}
S-1-5-8
% \null and \mbox are tricks to induce different typesetting decisions
\item[{ServerLogon(aka domain controller account)}]\null{}
S-1-5-8
% \null and \mbox are tricks to induce different typesetting decisions
\item[{(Logon IDs)}]\null{}
S-1-5-5-X-Y
% \null and \mbox are tricks to induce different typesetting decisions
\item[{(NT non-unique IDs)}]\null{}
S-1-5-0x15-...
% \null and \mbox are tricks to induce different typesetting decisions
\item[{(Built-in domain)}]\null{}
s-1-5-0x20
\end{description}
\subsection{Well-known RIDS}
\label{id2512930}\hypertarget{id2512930}{}%

A RID is a sub-authority value, as part of either a SID, or in the case of Group RIDs, part of the DOM\_GID structure, in the USER\_INFO\_1 structure, in the LSA SAM Logon response.
\subsubsection{Well-known RID users}
\label{id2512943}\hypertarget{id2512943}{}%
 {\em{Groupname:}} DOMAIN\_USER\_RID\_ADMIN {\em{????:}} 0x0000 {\em{RID:}} 01F4 \\
 {\em{Groupname:}} DOMAIN\_USER\_RID\_GUEST {\em{????:}} 0x0000 {\em{RID:}} 01F5

\subsubsection{Well-known RID groups}
\label{id2512985}\hypertarget{id2512985}{}%
 {\em{Groupname:}} DOMAIN\_GROUP\_RID\_ADMINS {\em{????:}} 0x0000 {\em{RID:}} 0200 \\
 {\em{Groupname:}} DOMAIN\_GROUP\_RID\_USERS {\em{????:}} 0x0000 {\em{RID:}} 0201 \\
 {\em{Groupname:}} DOMAIN\_GROUP\_RID\_GUESTS {\em{????:}} 0x0000 {\em{RID:}} 0202

\subsubsection{Well-known RID aliases}
\label{id2513037}\hypertarget{id2513037}{}%
 {\em{Groupname:}} DOMAIN\_ALIAS\_RID\_ADMINS {\em{????:}} 0x0000 {\em{RID:}} 0220 \\
 {\em{Groupname:}} DOMAIN\_ALIAS\_RID\_USERS {\em{????:}} 0x0000 {\em{RID:}} 0221 \\
 {\em{Groupname:}} DOMAIN\_ALIAS\_RID\_GUESTS {\em{????:}} 0x0000 {\em{RID:}} 0222 \\
 {\em{Groupname:}} DOMAIN\_ALIAS\_RID\_POWER\_USERS {\em{????:}} 0x0000 {\em{RID:}} 0223 \\
 {\em{Groupname:}} DOMAIN\_ALIAS\_RID\_ACCOUNT\_OPS {\em{????:}} 0x0000 {\em{RID:}} 0224 \\
 {\em{Groupname:}} DOMAIN\_ALIAS\_RID\_SYSTEM\_OPS {\em{????:}} 0x0000 {\em{RID:}} 0225 \\
 {\em{Groupname:}} DOMAIN\_ALIAS\_RID\_PRINT\_OPS {\em{????:}} 0x0000 {\em{RID:}} 0226 \\
 {\em{Groupname:}} DOMAIN\_ALIAS\_RID\_BACKUP\_OPS {\em{????:}} 0x0000 {\em{RID:}} 0227 \\
 {\em{Groupname:}} DOMAIN\_ALIAS\_RID\_REPLICATOR {\em{????:}} 0x0000 {\em{RID:}} 0228

% ------------------------------------------------------------- 
% 
% PART Samba Basics
% 
% ------------------------------------------------------------- 
\part{Samba Basics}
\label{id2407207}\hypertarget{id2407207}{}%

% -------------------------------------------------------------
% Chapter Samba Architecture 
% ------------------------------------------------------------- 	
\chapter{Samba Architecture}
\label{architecture}\hypertarget{architecture}{}%

% ------------------------   
% Section 
\section{Introduction}
\label{id2489320}\hypertarget{id2489320}{}%

This document gives a general overview of how Samba works internally. The Samba Team has tried to come up with a model which is the best possible compromise between elegance, portability, security and the constraints imposed by the very messy SMB and CIFS protocol.

It also tries to answer some of the frequently asked questions such as:

\begin{enumerate}
%--- Item
\item 
Is Samba secure when running on Unix? The xyz platform? What about the root priveliges issue?


%--- Item
\item 
Pros and cons of multithreading in various parts of Samba


%--- Item
\item 
Why not have a separate process for name resolution, WINS, and browsing?

\end{enumerate}

% ------------------------   
% Section 
\section{Multithreading and Samba}
\label{id2461270}\hypertarget{id2461270}{}%

People sometimes tout threads as a uniformly good thing. They are very nice in their place but are quite inappropriate for smbd. nmbd is another matter, and multi-threading it would be very nice.

The short version is that smbd is not multithreaded, and alternative servers that take this approach under Unix (such as Syntax, at the time of writing) suffer tremendous performance penalties and are less robust. nmbd is not threaded either, but this is because it is not possible to do it while keeping code consistent and portable across 35 or more platforms. (This drawback also applies to threading smbd.)

The longer versions is that there are very good reasons for not making smbd multi-threaded. Multi-threading would actually make Samba much slower, less scalable, less portable and much less robust. The fact that we use a separate process for each connection is one of Samba's biggest advantages.

% ------------------------   
% Section 
\section{Threading smbd}
\label{id2499011}\hypertarget{id2499011}{}%

A few problems that would arise from a threaded smbd are:

\begin{enumerate}
%--- Item
\item 
It's not only to create threads instead of processes, but you must care about all variables if they have to be thread specific (currently they would be global).


%--- Item
\item 
if one thread dies (eg. a seg fault) then all threads die. We can immediately throw robustness out the window.


%--- Item
\item 
many of the system calls we make are blocking. Non-blocking equivalents of many calls are either not available or are awkward (and slow) to use. So while we block in one thread all clients are waiting. Imagine if one share is a slow NFS filesystem and the others are fast, we will end up slowing all clients to the speed of NFS.


%--- Item
\item 
you can't run as a different uid in different threads. This means we would have to switch uid/gid on \_every\_ SMB packet. It would be horrendously slow.


%--- Item
\item 
the per process file descriptor limit would mean that we could only support a limited number of clients.


%--- Item
\item 
we couldn't use the system locking calls as the locking context of fcntl() is a process, not a thread.

\end{enumerate}

% ------------------------   
% Section 
\section{Threading nmbd}
\label{id2499075}\hypertarget{id2499075}{}%

This would be ideal, but gets sunk by portability requirements.

Andrew tried to write a test threads library for nmbd that used only ansi-C constructs (using setjmp and longjmp). Unfortunately some OSes defeat this by restricting longjmp to calling addresses that are shallower than the current address on the stack (apparently AIX does this). This makes a truly portable threads library impossible. So to support all our current platforms we would have to code nmbd both with and without threads, and as the real aim of threads is to make the code clearer we would not have gained anything. (it is a myth that threads make things faster. threading is like recursion, it can make things clear but the same thing can always be done faster by some other method)

Chris tried to spec out a general design that would abstract threading vs separate processes (vs other methods?) and make them accessible through some general API. This doesn't work because of the data sharing requirements of the protocol (packets in the future depending on packets now, etc.) At least, the code would work but would be very clumsy, and besides the fork() type model would never work on Unix. (Is there an OS that it would work on, for nmbd?)

A fork() is cheap, but not nearly cheap enough to do on every UDP packet that arrives. Having a pool of processes is possible but is nasty to program cleanly due to the enormous amount of shared data (in complex structures) between the processes. We can't rely on each platform having a shared memory system.

% ------------------------   
% Section 
\section{nbmd Design}
\label{id2483067}\hypertarget{id2483067}{}%

Originally Andrew used recursion to simulate a multi-threaded environment, which use the stack enormously and made for really confusing debugging sessions. Luke Leighton rewrote it to use a queuing system that keeps state information on each packet. The first version used a single structure which was used by all the pending states. As the initialisation of this structure was done by adding arguments, as the functionality developed, it got pretty messy. So, it was replaced with a higher-order function and a pointer to a user-defined memory block. This suddenly made things much simpler: large numbers of functions could be made static, and modularised. This is the same principle as used in NT's kernel, and achieves the same effect as threads, but in a single process.

Then Jeremy rewrote nmbd. The packet data in nmbd isn't what's on the wire. It's a nice format that is very amenable to processing but still keeps the idea of a distinct packet. See "struct packet\_struct" in nameserv.h. It has all the detail but none of the on-the-wire mess. This makes it ideal for using in disk or memory-based databases for browsing and WINS support.

% -------------------------------------------------------------
% Chapter The samba DEBUG system 
% ------------------------------------------------------------- 	
\chapter{The samba DEBUG system}
\label{debug}\hypertarget{debug}{}%

% ------------------------   
% Section 
\section{New Output Syntax}
\label{id2482016}\hypertarget{id2482016}{}%

The syntax of a debugging log file is represented as:


\begin{Verbatim}[]

  >debugfile< :== { >debugmsg< }

  >debugmsg<  :== >debughdr< '\n' >debugtext<

  >debughdr<  :== '[' TIME ',' LEVEL ']' FILE ':' [FUNCTION] '(' LINE ')'

  >debugtext< :== { >debugline< }

  >debugline< :== TEXT '\n'

\end{Verbatim}


TEXT is a string of characters excluding the newline character.

LEVEL is the DEBUG level of the message (an integer in the range 0..10).

TIME is a timestamp.

FILE is the name of the file from which the debug message was generated.

FUNCTION is the function from which the debug message was generated.

LINE is the line number of the debug statement that generated the message.

Basically, what that all means is:

\begin{enumerate}
%--- Item
\item 
A debugging log file is made up of debug messages.


%--- Item
\item 
Each debug message is made up of a header and text. The header is separated from the text by a newline.


%--- Item
\item 
The header begins with the timestamp and debug level of the message enclosed in brackets. The filename, function, and line number at which the message was generated follow. The filename is terminated by a colon, and the function name is terminated by the parenthesis which contain the line number. Depending upon the compiler, the function name may be missing (it is generated by the \_\_FUNCTION\_\_ macro, which is not universally implemented, dangit).


%--- Item
\item 
The message text is made up of zero or more lines, each terminated by a newline.

\end{enumerate}

Here's some example output:


\begin{Verbatim}[]

    [1998/08/03 12:55:25, 1] nmbd.c:(659)
      Netbios nameserver version 1.9.19-prealpha started.
      Copyright Andrew Tridgell 1994-1997
    [1998/08/03 12:55:25, 3] loadparm.c:(763)
      Initializing global parameters

\end{Verbatim}


Note that in the above example the function names are not listed on the header line. That's because the example above was generated on an SGI Indy, and the SGI compiler doesn't support the \_\_FUNCTION\_\_ macro.

% ------------------------   
% Section 
\section{The DEBUG() Macro}
\label{id2481069}\hypertarget{id2481069}{}%

Use of the DEBUG() macro is unchanged. DEBUG() takes two parameters. The first is the message level, the second is the body of a function call to the Debug1() function.

That's confusing.

Here's an example which may help a bit. If you would write


\begin{Verbatim}[]

printf( "This is a %s message.\n", "debug" );

\end{Verbatim}


to send the output to stdout, then you would write


\begin{Verbatim}[]

DEBUG( 0, ( "This is a %s message.\n", "debug" ) );

\end{Verbatim}


to send the output to the debug file. All of the normal printf() formatting escapes work.

Note that in the above example the DEBUG message level is set to 0. Messages at level 0 always print. Basically, if the message level is less than or equal to the global value DEBUGLEVEL, then the DEBUG statement is processed.

The output of the above example would be something like:


\begin{Verbatim}[]

    [1998/07/30 16:00:51, 0] file.c:function(128)
      This is a debug message.

\end{Verbatim}


Each call to DEBUG() creates a new header *unless* the output produced by the previous call to DEBUG() did not end with a '\textbackslash n'. Output to the debug file is passed through a formatting buffer which is flushed every time a newline is encountered. If the buffer is not empty when DEBUG() is called, the new input is simply appended.

...but that's really just a Kludge. It was put in place because DEBUG() has been used to write partial lines. Here's a simple (dumb) example of the kind of thing I'm talking about:


\begin{Verbatim}[]

    DEBUG( 0, ("The test returned " ) );
    if( test() )
      DEBUG(0, ("True") );
    else
      DEBUG(0, ("False") );
    DEBUG(0, (".\n") );

\end{Verbatim}


Without the format buffer, the output (assuming test() returned true) would look like this:


\begin{Verbatim}[]

    [1998/07/30 16:00:51, 0] file.c:function(256)
      The test returned
    [1998/07/30 16:00:51, 0] file.c:function(258)
      True
    [1998/07/30 16:00:51, 0] file.c:function(261)
      .

\end{Verbatim}


Which isn't much use. The format buffer kludge fixes this problem.

% ------------------------   
% Section 
\section{The DEBUGADD() Macro}
\label{id2465283}\hypertarget{id2465283}{}%

In addition to the kludgey solution to the broken line problem described above, there is a clean solution. The DEBUGADD() macro never generates a header. It will append new text to the current debug message even if the format buffer is empty. The syntax of the DEBUGADD() macro is the same as that of the DEBUG() macro.


\begin{Verbatim}[]

    DEBUG( 0, ("This is the first line.\n" ) );
    DEBUGADD( 0, ("This is the second line.\nThis is the third line.\n" ) );

\end{Verbatim}


Produces


\begin{Verbatim}[]

    [1998/07/30 16:00:51, 0] file.c:function(512)
      This is the first line.
      This is the second line.
      This is the third line.

\end{Verbatim}


% ------------------------   
% Section 
\section{The DEBUGLVL() Macro}
\label{id2465321}\hypertarget{id2465321}{}%

One of the problems with the DEBUG() macro was that DEBUG() lines tended to get a bit long. Consider this example from nmbd\_sendannounce.c:


\begin{Verbatim}[]

  DEBUG(3,("send_local_master_announcement: type %x for name %s on subnet %s for workgroup %s\n",
            type, global_myname, subrec->subnet_name, work->work_group));

\end{Verbatim}


One solution to this is to break it down using DEBUG() and DEBUGADD(), as follows:


\begin{Verbatim}[]

  DEBUG( 3, ( "send_local_master_announcement: " ) );
  DEBUGADD( 3, ( "type %x for name %s ", type, global_myname ) );
  DEBUGADD( 3, ( "on subnet %s ", subrec->subnet_name ) );
  DEBUGADD( 3, ( "for workgroup %s\n", work->work_group ) );

\end{Verbatim}


A similar, but arguably nicer approach is to use the DEBUGLVL() macro. This macro returns True if the message level is less than or equal to the global DEBUGLEVEL value, so:


\begin{Verbatim}[]

  if( DEBUGLVL( 3 ) )
    {
    dbgtext( "send_local_master_announcement: " );
    dbgtext( "type %x for name %s ", type, global_myname );
    dbgtext( "on subnet %s ", subrec->subnet_name );
    dbgtext( "for workgroup %s\n", work->work_group );
    }

\end{Verbatim}


(The dbgtext() function is explained below.)

There are a few advantages to this scheme:

\begin{enumerate}
%--- Item
\item 
The test is performed only once.


%--- Item
\item 
You can allocate variables off of the stack that will only be used within the DEBUGLVL() block.


%--- Item
\item 
Processing that is only relevant to debug output can be contained within the DEBUGLVL() block.

\end{enumerate}

% ------------------------   
% Section 
\section{New Functions}
\label{id2465410}\hypertarget{id2465410}{}%
\subsection{dbgtext()}
\label{id2408596}\hypertarget{id2408596}{}%

This function prints debug message text to the debug file (and possibly to syslog) via the format buffer. The function uses a variable argument list just like printf() or Debug1(). The input is printed into a buffer using the vslprintf() function, and then passed to format\_debug\_text(). If you use DEBUGLVL() you will probably print the body of the message using dbgtext().
\subsection{dbghdr()}
\label{id2408614}\hypertarget{id2408614}{}%

This is the function that writes a debug message header. Headers are not processed via the format buffer. Also note that if the format buffer is not empty, a call to dbghdr() will not produce any output. See the comments in dbghdr() for more info.

It is not likely that this function will be called directly. It is used by DEBUG() and DEBUGADD().
\subsection{format\_debug\_text()}
\label{id2408634}\hypertarget{id2408634}{}%

This is a static function in debug.c. It stores the output text for the body of the message in a buffer until it encounters a newline. When the newline character is found, the buffer is written to the debug file via the Debug1() function, and the buffer is reset. This allows us to add the indentation at the beginning of each line of the message body, and also ensures that the output is written a line at a time (which cleans up syslog output).

% -------------------------------------------------------------
% Chapter Samba Internals 
% ------------------------------------------------------------- 	
\chapter{Samba Internals}
\label{internals}\hypertarget{internals}{}%

% ------------------------   
% Section 
\section{Character Handling}
\label{id2460192}\hypertarget{id2460192}{}%

This section describes character set handling in Samba, as implemented in Samba 3.0 and above

In the past Samba had very ad-hoc character set handling. Scattered throughout the code were numerous calls which converted particular strings to/from DOS codepages. The problem is that there was no way of telling if a particular char* is in dos codepage or unix codepage. This led to a nightmare of code that tried to cope with particular cases without handlingt the general case.

% ------------------------   
% Section 
\section{The new functions}
\label{id2490718}\hypertarget{id2490718}{}%

The new system works like this:

\begin{enumerate}
%--- Item
\item 
all char* strings inside Samba are "unix" strings. These are multi-byte strings that are in the charset defined by the "unix charset" option in smb.conf.


%--- Item
\item 
there is no single fixed character set for unix strings, but any character set that is used does need the following properties:

\begin{enumerate}
%--- Item
\item 
must not contain NULLs except for termination


%--- Item
\item 
must be 7-bit compatible with C strings, so that a constant string or character in C will be byte-for-byte identical to the equivalent string in the chosen character set.


%--- Item
\item 
when you uppercase or lowercase a string it does not become longer than the original string


%--- Item
\item 
must be able to correctly hold all characters that your client will throw at it

\end{enumerate}

For example, UTF-8 is fine, and most multi-byte asian character sets are fine, but UCS2 could not be used for unix strings as they contain nulls.


%--- Item
\item 
when you need to put a string into a buffer that will be sent on the wire, or you need a string in a character set format that is compatible with the clients character set then you need to use a pull\_ or push\_ function. The pull\_ functions pull a string from a wire buffer into a (multi-byte) unix string. The push\_ functions push a string out to a wire buffer.


%--- Item
\item 
the two main pull\_ and push\_ functions you need to understand are pull\_string and push\_string. These functions take a base pointer that should point at the start of the SMB packet that the string is in. The functions will check the flags field in this packet to automatically determine if the packet is marked as a unicode packet, and they will choose whether to use unicode for this string based on that flag. You may also force this decision using the STR\_UNICODE or STR\_ASCII flags. For use in smbd/ and libsmb/ there are wrapper functions clistr\_ and srvstr\_ that call the pull\_/push\_ functions with the appropriate first argument.

You may also call the pull\_ascii/pull\_ucs2 or push\_ascii/push\_ucs2 functions if you know that a particular string is ascii or unicode. There are also a number of other convenience functions in charcnv.c that call the pull\_/push\_ functions with particularly common arguments, such as pull\_ascii\_pstring()


%--- Item
\item 
The biggest thing to remember is that internal (unix) strings in Samba may now contain multi-byte characters. This means you cannot assume that characters are always 1 byte long. Often this means that you will have to convert strings to ucs2 and back again in order to do some (seemingly) simple task. For examples of how to do this see functions like strchr\_m(). I know this is very slow, and we will eventually speed it up but right now we want this stuff correct not fast.


%--- Item
\item 
all lp\_ functions now return unix strings. The magic "DOS" flag on parameters is gone.


%--- Item
\item 
all vfs functions take unix strings. Don't convert when passing to them

\end{enumerate}

% ------------------------   
% Section 
\section{Macros in byteorder.h}
\label{id2500670}\hypertarget{id2500670}{}%

This section describes the macros defined in byteorder.h. These macros are used extensively in the Samba code.
\subsection{CVAL(buf,pos)}
\label{id2500681}\hypertarget{id2500681}{}%

returns the byte at offset pos within buffer buf as an unsigned character.
\subsection{PVAL(buf,pos)}
\label{id2500693}\hypertarget{id2500693}{}%

returns the value of CVAL(buf,pos) cast to type unsigned integer.
\subsection{SCVAL(buf,pos,val)}
\label{id2500704}\hypertarget{id2500704}{}%

sets the byte at offset pos within buffer buf to value val.
\subsection{SVAL(buf,pos)}
\label{id2500715}\hypertarget{id2500715}{}%

returns the value of the unsigned short (16 bit) little-endian integer at offset pos within buffer buf. An integer of this type is sometimes refered to as "USHORT".
\subsection{IVAL(buf,pos)}
\label{id2500729}\hypertarget{id2500729}{}%

returns the value of the unsigned 32 bit little-endian integer at offset pos within buffer buf.
\subsection{SVALS(buf,pos)}
\label{id2499618}\hypertarget{id2499618}{}%

returns the value of the signed short (16 bit) little-endian integer at offset pos within buffer buf.
\subsection{IVALS(buf,pos)}
\label{id2486240}\hypertarget{id2486240}{}%

returns the value of the signed 32 bit little-endian integer at offset pos within buffer buf.
\subsection{SSVAL(buf,pos,val)}
\label{id2482354}\hypertarget{id2482354}{}%

sets the unsigned short (16 bit) little-endian integer at offset pos within buffer buf to value val.
\subsection{SIVAL(buf,pos,val)}
\label{id2482366}\hypertarget{id2482366}{}%

sets the unsigned 32 bit little-endian integer at offset pos within buffer buf to the value val.
\subsection{SSVALS(buf,pos,val)}
\label{id2482379}\hypertarget{id2482379}{}%

sets the short (16 bit) signed little-endian integer at offset pos within buffer buf to the value val.
\subsection{SIVALS(buf,pos,val)}
\label{id2482391}\hypertarget{id2482391}{}%

sets the signed 32 bit little-endian integer at offset pos withing buffer buf to the value val.
\subsection{RSVAL(buf,pos)}
\label{id2482404}\hypertarget{id2482404}{}%

returns the value of the unsigned short (16 bit) big-endian integer at offset pos within buffer buf.
\subsection{RIVAL(buf,pos)}
\label{id2482416}\hypertarget{id2482416}{}%

returns the value of the unsigned 32 bit big-endian integer at offset pos within buffer buf.
\subsection{RSSVAL(buf,pos,val)}
\label{id2482429}\hypertarget{id2482429}{}%

sets the value of the unsigned short (16 bit) big-endian integer at offset pos within buffer buf to value val. refered to as "USHORT".
\subsection{RSIVAL(buf,pos,val)}
\label{id2482442}\hypertarget{id2482442}{}%

sets the value of the unsigned 32 bit big-endian integer at offset pos within buffer buf to value val.

% ------------------------   
% Section 
\section{LAN Manager Samba API}
\label{id2482456}\hypertarget{id2482456}{}%

This section describes the functions need to make a LAN Manager RPC call. This information had been obtained by examining the Samba code and the LAN Manager 2.0 API documentation. It should not be considered entirely reliable.


\begin{Verbatim}[]

call_api(int prcnt, int drcnt, int mprcnt, int mdrcnt, 
	char *param, char *data, char **rparam, char **rdata);

\end{Verbatim}


This function is defined in client.c. It uses an SMB transaction to call a remote api.
\subsection{Parameters}
\label{id2482483}\hypertarget{id2482483}{}%

The parameters are as follows:

\begin{enumerate}
%--- Item
\item 
prcnt: the number of bytes of parameters begin sent.


%--- Item
\item 
drcnt: the number of bytes of data begin sent.


%--- Item
\item 
mprcnt: the maximum number of bytes of parameters which should be returned


%--- Item
\item 
mdrcnt: the maximum number of bytes of data which should be returned


%--- Item
\item 
param: a pointer to the parameters to be sent.


%--- Item
\item 
data: a pointer to the data to be sent.


%--- Item
\item 
rparam: a pointer to a pointer which will be set to point to the returned parameters. The caller of call\_api() must deallocate this memory.


%--- Item
\item 
rdata: a pointer to a pointer which will be set to point to the returned data. The caller of call\_api() must deallocate this memory.

\end{enumerate}

These are the parameters which you ought to send, in the order of their appearance in the parameter block:

\begin{enumerate}
%--- Item
\item 
An unsigned 16 bit integer API number. You should set this value with SSVAL(). I do not know where these numbers are described.


%--- Item
\item 
An ASCIIZ string describing the parameters to the API function as defined in the LAN Manager documentation. The first parameter, which is the server name, is ommited. This string is based uppon the API function as described in the manual, not the data which is actually passed.


%--- Item
\item 
An ASCIIZ string describing the data structure which ought to be returned.


%--- Item
\item 
Any parameters which appear in the function call, as defined in the LAN Manager API documentation, after the "Server" and up to and including the "uLevel" parameters.


%--- Item
\item 
An unsigned 16 bit integer which gives the size in bytes of the buffer we will use to receive the returned array of data structures. Presumably this should be the same as mdrcnt. This value should be set with SSVAL().


%--- Item
\item 
An ASCIIZ string describing substructures which should be returned. If no substructures apply, this string is of zero length.

\end{enumerate}

The code in client.c always calls call\_api() with no data. It is unclear when a non-zero length data buffer would be sent.
\subsection{Return value}
\label{id2458342}\hypertarget{id2458342}{}%

The returned parameters (pointed to by rparam), in their order of appearance are:

\begin{enumerate}
%--- Item
\item 
An unsigned 16 bit integer which contains the API function's return code. This value should be read with SVAL().


%--- Item
\item 
An adjustment which tells the amount by which pointers in the returned data should be adjusted. This value should be read with SVAL(). Basically, the address of the start of the returned data buffer should have the returned pointer value added to it and then have this value subtracted from it in order to obtain the currect offset into the returned data buffer.


%--- Item
\item 
A count of the number of elements in the array of structures returned. It is also possible that this may sometimes be the number of bytes returned.

\end{enumerate}

When call\_api() returns, rparam points to the returned parameters. The first if these is the result code. It will be zero if the API call suceeded. This value by be read with "SVAL(rparam,0)".

The second parameter may be read as "SVAL(rparam,2)". It is a 16 bit offset which indicates what the base address of the returned data buffer was when it was built on the server. It should be used to correct pointer before use.

The returned data buffer contains the array of returned data structures. Note that all pointers must be adjusted before use. The function fix\_char\_ptr() in client.c can be used for this purpose.

The third parameter (which may be read as "SVAL(rparam,4)") has something to do with indicating the amount of data returned or possibly the amount of data which can be returned if enough buffer space is allowed.

% ------------------------   
% Section 
\section{Code character table}
\label{id2458417}\hypertarget{id2458417}{}%

Certain data structures are described by means of ASCIIz strings containing code characters. These are the code characters:

\begin{enumerate}
%--- Item
\item 
W a type byte little-endian unsigned integer


%--- Item
\item 
N a count of substructures which follow


%--- Item
\item 
D a four byte little-endian unsigned integer


%--- Item
\item 
B a byte (with optional count expressed as trailing ASCII digits)


%--- Item
\item 
z a four byte offset to a NULL terminated string


%--- Item
\item 
l a four byte offset to non-string user data


%--- Item
\item 
b an offset to data (with count expressed as trailing ASCII digits)


%--- Item
\item 
r pointer to returned data buffer???


%--- Item
\item 
L length in bytes of returned data buffer???


%--- Item
\item 
h number of bytes of information available???

\end{enumerate}

% -------------------------------------------------------------
% Chapter Coding Suggestions 
% ------------------------------------------------------------- 	
\chapter{Coding Suggestions}
\label{CodingSuggestions}\hypertarget{CodingSuggestions}{}%

So you want to add code to Samba ...

One of the daunting tasks facing a programmer attempting to write code for Samba is understanding the various coding conventions used by those most active in the project. These conventions were mostly unwritten and helped improve either the portability, stability or consistency of the code. This document will attempt to document a few of the more important coding practices used at this time on the Samba project. The coding practices are expected to change slightly over time, and even to grow as more is learned about obscure portability considerations. Two existing documents {\texttt{\docbookhyphenatefilename{samba/\dbz{}source/\dbz{}internals.\dbz{}doc}}} and {\texttt{\docbookhyphenatefilename{samba/\dbz{}source/\dbz{}architecture.\dbz{}doc}}} provide additional information.

The loosely related question of coding style is very personal and this document does not attempt to address that subject, except to say that I have observed that eight character tabs seem to be preferred in Samba source. If you are interested in the topic of coding style, two oft-quoted documents are:

{\textless}\url{http://lxr.linux.no/source/Documentation/CodingStyle}{\textgreater}

{\textless}\url{http://www.fsf.org/prep/standards_toc.html}{\textgreater}

But note that coding style in Samba varies due to the many different programmers who have contributed.

Following are some considerations you should use when adding new code to Samba. First and foremost remember that:

Portability is a primary consideration in adding function, as is network compatability with de facto, existing, real world CIFS/SMB implementations. There are lots of platforms that Samba builds on so use caution when adding a call to a library function that is not invoked in existing Samba code. Also note that there are many quite different SMB/CIFS clients that Samba tries to support, not all of which follow the SNIA CIFS Technical Reference (or the earlier Microsoft reference documents or the X/Open book on the SMB Standard) perfectly.

Here are some other suggestions:

\begin{enumerate}
%--- Item
\item 
use d\_printf instead of printf for display text reason: enable auto-substitution of translated language text


%--- Item
\item 
use SAFE\_FREE instead of free reason: reduce traps due to null pointers


%--- Item
\item 
don't use bzero use memset, or ZERO\_STRUCT and ZERO\_STRUCTP macros reason: not POSIX


%--- Item
\item 
don't use strcpy and strlen (use safe\_* equivalents) reason: to avoid traps due to buffer overruns


%--- Item
\item 
don't use getopt\_long, use popt functions instead reason: portability


%--- Item
\item 
explicitly add const qualifiers on parm passing in functions where parm is input only (somewhat controversial but const can be \#defined away)


%--- Item
\item 
when passing a va\_list as an arg, or assigning one to another please use the VA\_COPY() macro reason: on some platforms, va\_list is a struct that must be initialized in each function...can SEGV if you don't.


%--- Item
\item 
discourage use of threads reason: portability (also see architecture.doc)


%--- Item
\item 
don't explicitly include new header files in C files - new h files should be included by adding them once to includes.h reason: consistency


%--- Item
\item 
don't explicitly extern functions (they are autogenerated by "make proto" into proto.h) reason: consistency


%--- Item
\item 
use endian safe macros when unpacking SMBs (see byteorder.h and internals.doc) reason: not everyone uses Intel


%--- Item
\item 
Note Unicode implications of charset handling (see internals.doc). See pull\_* and push\_* and convert\_string functions. reason: Internationalization


%--- Item
\item 
Don't assume English only reason: See above


%--- Item
\item 
Try to avoid using in/out parameters (functions that return data which overwrites input parameters) reason: Can cause stability problems


%--- Item
\item 
Ensure copyright notices are correct, don't append Tridge's name to code that he didn't write. If you did not write the code, make sure that it can coexist with the rest of the Samba GPLed code.


%--- Item
\item 
Consider usage of DATA\_BLOBs for length specified byte-data. reason: stability


%--- Item
\item 
Take advantage of tdbs for database like function reason: consistency


%--- Item
\item 
Don't access the SAM\_ACCOUNT structure directly, they should be accessed via pdb\_get...() and pdb\_set...() functions. reason: stability, consistency


%--- Item
\item 
Don't check a password directly against the passdb, always use the check\_password() interface. reason: long term pluggability


%--- Item
\item 
Try to use asprintf rather than pstrings and fstrings where possible


%--- Item
\item 
Use normal C comments / * instead of C++ comments // like this. Although the C++ comment format is part of the C99 standard, some older vendor C compilers do not accept it.


%--- Item
\item 
Try to write documentation for API functions and structures explaining the point of the code, the way it should be used, and any special conditions or results. Mark these with a double-star comment start / ** so that they can be picked up by Doxygen, as in this file.


%--- Item
\item 
Keep the scope narrow. This means making functions/variables static whenever possible. We don't want our namespace polluted. Each module should have a minimal number of externally visible functions or variables.


%--- Item
\item 
Use function pointers to keep knowledge about particular pieces of code isolated in one place. We don't want a particular piece of functionality to be spread out across lots of places - that makes for fragile, hand to maintain code. Instead, design an interface and use tables containing function pointers to implement specific functionality. This is particularly important for command interpreters.


%--- Item
\item 
Think carefully about what it will be like for someone else to add to and maintain your code. If it would be hard for someone else to maintain then do it another way.

\end{enumerate}

The suggestions above are simply that, suggestions, but the information may help in reducing the routine rework done on new code. The preceeding list is expected to change routinely as new support routines and macros are added.

% -------------------------------------------------------------
% Chapter Contributing code 
% ------------------------------------------------------------- 	
\chapter{Contributing code}
\label{contributing}\hypertarget{contributing}{}%

Here are a few tips and notes that might be useful if you are interested in modifying samba source code and getting it into samba's main branch.

\begin{description}
% \null and \mbox are tricks to induce different typesetting decisions
\item[{Retrieving the source}]\null{}
In order to contribute code to samba, make sure you have the latest source. Retrieving the samba source code from CVS is documented in the appendix of the Samba HOWTO Collection.
% \null and \mbox are tricks to induce different typesetting decisions
\item[{Discuss large modifications with team members}]\null{}
Please discuss large modifications you are going to make with members of the samba team. Some parts of the samba code have one or more 'owners' - samba developers who wrote most of the code and maintain it.

This way you can avoid spending your time and effort on something that is not going to make it into the main samba branch because someone else was working on the same thing or because your implementation is not the correct one.
% \null and \mbox are tricks to induce different typesetting decisions
\item[{Patch format}]\null{}
Patches to the samba tree should be in unified diff format, e.g. files generated by {\ttfamily\bfseries{\docbookhyphenatedot{diff -\dbz{}u}}}.

If you are modifying a copy of samba you retrieved from CVS, you can easily generate a diff file of these changes by running {\ttfamily\bfseries{\docbookhyphenatedot{cvs diff -\dbz{}u}}}.
% \null and \mbox are tricks to induce different typesetting decisions
\item[{Points of attention when modifying samba source code}]\null{}

\begin{itemize}
%--- Item
\item 
Don't simply copy code from other places and modify it until it works. Code needs to be clean and logical. Duplicate code is to be avoided.


%--- Item
\item 
Test your patch. It might take a while before one of us looks at your patch so it will take longer before your patch when your patch needs to go thru the review cycle again.


%--- Item
\item 
Don't put separate patches in one large diff file. This makes it harder to read, understand and test the patch. You might also risk not getting a good patch committed because you mixed it with one that had issues.


%--- Item
\item 
Make sure your patch complies to the samba coding style as suggested in the coding-suggestions chapter.

\end{itemize}

% \null and \mbox are tricks to induce different typesetting decisions
\item[{Sending in bugfixes}]\null{}
Bugfixes to bugs in samba should be submitted to samba's bugzilla system\label{id2466649}\begingroup\catcode`\#=12\footnote{ {\textless}\url{https://bugzilla.samba.org/}{\textgreater}}\endgroup\docbooktolatexmakefootnoteref{id2466649}, along with a description of the bug.
% \null and \mbox are tricks to induce different typesetting decisions
\item[{Sending in feature patches}]\null{}
Send feature patches along with a description of what the patch is supposed to do to the Samba-technical mailinglist\label{id2466671}\begingroup\catcode`\#=12\footnote{ {\textless}\url{mailto:samba-technical@samba.org}{\textgreater}}\endgroup\docbooktolatexmakefootnoteref{id2466671} and possibly to a samba team member who is (one of the) 'owners' of the code you made modifications to. We are all busy people so everybody tends to 'let one of the others handle it'. If nobody responded to your patch for a week, try to send it again until you get a response from one of us.
% \null and \mbox are tricks to induce different typesetting decisions
\item[{Feedback on your patch}]\null{}
One of the team members will look at your patch and either commit your patch or give comments why he won't apply it. In the latter case you can fix your patch and re-send it until your patch is approved.
\end{description}

% -------------------------------------------------------------
% Chapter Modules 
% ------------------------------------------------------------- 	
\chapter{Modules}
\label{modules}\hypertarget{modules}{}%

% ------------------------   
% Section 
\section{Advantages}
\label{id2475932}\hypertarget{id2475932}{}%

The new modules system has the following advantages:

\begin{tabular}{l}

Transparent loading of static and shared modules (no need for a subsystem to know about modules)  \\
Simple selection between shared and static modules at configure time  \\
"preload modules" option for increasing performance for stable modules  \\
No nasty \#define stuff anymore  \\
All backends are available as plugin now (including pdb\_ldap and pdb\_tdb)  \\
\end{tabular}

% ------------------------   
% Section 
\section{Loading modules}
\label{id2501186}\hypertarget{id2501186}{}%

Some subsystems in samba use different backends. These backends can be either statically linked in to samba or available as a plugin. A subsystem should have a function that allows a module to register itself. For example, the passdb subsystem has:


\begin{Verbatim}[]

NTSTATUS smb_register_passdb(int version, const char *name, pdb_init_function init);

\end{Verbatim}


This function will be called by the initialisation function of the module to register itself.
\subsection{Static modules}
\label{id2481484}\hypertarget{id2481484}{}%

The modules system compiles a list of initialisation functions for the static modules of each subsystem. This is a define. For example, it is here currently (from {\texttt{\docbookhyphenatefilename{include/\dbz{}config.\dbz{}h}}}):


\begin{Verbatim}[]

/* Static init functions */
#define static_init_pdb { pdb_mysql_init(); pdb_ldap_init(); pdb_smbpasswd_init(); pdb_tdbsam_init(); pdb_guest_init();}

\end{Verbatim}


These functions should be called before the subsystem is used. That should be done when the subsystem is initialised or first used.
\subsection{Shared modules}
\label{id2481519}\hypertarget{id2481519}{}%

If a subsystem needs a certain backend, it should check if it has already been registered. If the backend hasn't been registered already, the subsystem should call smb\_probe\_module(char *subsystem, char *backend). This function tries to load the correct module from a certain path (\$LIBDIR/subsystem/backend.so). If the first character in 'backend' is a slash, smb\_probe\_module() tries to load the module from the absolute path specified in 'backend'.

After smb\_probe\_module() has been executed, the subsystem should check again if the module has been registered.

% ------------------------   
% Section 
\section{Writing modules}
\label{id2482605}\hypertarget{id2482605}{}%

Each module has an initialisation function. For modules that are included with samba this name is '{\ttfamily\itshape{\docbookhyphenatedot{subsystem}}}\_{\ttfamily\itshape{\docbookhyphenatedot{backend}}}\_init'. For external modules (that will never be built-in, but only available as a module) this name is always 'init\_module'. (In the case of modules included with samba, the configure system will add a \#define subsystem\_backend\_init() init\_module()). The prototype for these functions is:


\begin{Verbatim}[]

NTSTATUS init_module(void);

\end{Verbatim}


This function should call one or more registration functions. The function should return NT\_STATUS\_OK on success and NT\_STATUS\_UNSUCCESSFUL or a more useful nt error code on failure.

For example, pdb\_ldap\_init() contains:


\begin{Verbatim}[]

NTSTATUS pdb_ldap_init(void)
{
smb_register_passdb(PASSDB_INTERFACE_VERSION, "ldapsam", pdb_init_ldapsam);
smb_register_passdb(PASSDB_INTERFACE_VERSION, "ldapsam_nua", pdb_init_ldapsam_nua);
	return NT_STATUS_OK;
}

\end{Verbatim}

\subsection{Static/Shared selection in configure.in}
\label{id2482655}\hypertarget{id2482655}{}%

Some macros in configure.in generate the various defines and substs that are necessary for the system to work correct. All modules that should be built by default have to be added to the variable 'default\_modules'. For example, if ldap is found, pdb\_ldap is added to this variable.

On the bottom of configure.in, SMB\_MODULE() should be called for each module and SMB\_SUBSYSTEM() for each subsystem.

Syntax:


\begin{Verbatim}[]

SMB_MODULE(subsystem_backend, object files, plugin name, subsystem name, static_action, shared_action)
SMB_SUBSYSTEM(subsystem,depfile)

\end{Verbatim}


The depfile for a certain subsystem is the file that calls the initialisation functions for the statically built in modules.

{\ttfamily\itshape{\docbookhyphenatedot{@SUBSYSTEM\_MODULES@}}} in Makefile.in will be replaced with the names of the plugins to build.

You must make sure all .c files that contain defines that can be changed by ./configure are rebuilded in the 'modules\_clean' make target. Practically, this means all c files that contain {\bfseries{static\_init\_subsystem;}} calls need to be rebuilded.

\begin{admonition}{xslt/figures/note}{Note}% NOTICE: see the db2latex FAQ w.r.t db2latex variable $latex.admonition.path

There currently also is a configure.in command called SMB\_MODULE\_PROVIVES(). This is used for modules that register multiple things. It should not be used as probing will most likely disappear in the future.
\end{admonition}

% ------------------------------------------------------------- 
% 
% PART Samba Subsystems
% 
% ------------------------------------------------------------- 
\part{Samba Subsystems}
\label{id2407240}\hypertarget{id2407240}{}%

% -------------------------------------------------------------
% Chapter RPC Pluggable Modules 
% ------------------------------------------------------------- 	
\chapter{RPC Pluggable Modules}
\label{rpc-plugin}\hypertarget{rpc-plugin}{}%

% ------------------------   
% Section 
\section{About}
\label{id2467197}\hypertarget{id2467197}{}%

This document describes how to make use the new RPC Pluggable Modules features of Samba 3.0. This architecture was added to increase the maintainability of Samba allowing RPC Pipes to be worked on separately from the main CVS branch. The RPM architecture will also allow third-party vendors to add functionality to Samba through plug-ins.

% ------------------------   
% Section 
\section{General Overview}
\label{id2468764}\hypertarget{id2468764}{}%

When an RPC call is sent to smbd, smbd tries to load a shared library by the name {\texttt{\docbookhyphenatefilename{librpc\_\dbz{}<pipename>.\dbz{}so}}} to handle the call if it doesn't know how to handle the call internally. For instance, LSA calls are handled by {\texttt{\docbookhyphenatefilename{librpc\_\dbz{}lsass.\dbz{}so}}}.. These shared libraries should be located in the {\texttt{\docbookhyphenatefilename{<sambaroot>/\dbz{}lib/\dbz{}rpc}}}. smbd then attempts to call the init\_module function within the shared library. Check the chapter on modules for more information.

In the init\_module function, the library should call rpc\_pipe\_register\_commands(). This function takes the following arguments:


\begin{Verbatim}[]

NTSTATUS rpc_pipe_register_commands(int version, const char *clnt, const char *srv,
                               const struct api_struct *cmds, int size);

\end{Verbatim}


\begin{description}
% \null and \mbox are tricks to induce different typesetting decisions
\item[{version}]\null{}
Version number of the RPC interface. Use the define {\em{SMB\_RPC\_INTERFACE\_VERSION}} for this argument.
% \null and \mbox are tricks to induce different typesetting decisions
\item[{clnt}]\null{}
the Client name of the named pipe
% \null and \mbox are tricks to induce different typesetting decisions
\item[{srv}]\null{}
the Server name of the named pipe
% \null and \mbox are tricks to induce different typesetting decisions
\item[{cmds}]\null{}
a list of api\_structs that map RPC ordinal numbers to function calls
% \null and \mbox are tricks to induce different typesetting decisions
\item[{size}]\null{}
the number of api\_structs contained in cmds
\end{description}

See rpc\_server/srv\_reg.c and rpc\_server/srv\_reg\_nt.c for a small example of how to use this library.

% -------------------------------------------------------------
% Chapter VFS Modules 
% ------------------------------------------------------------- 	
\chapter{VFS Modules}
\label{vfs}\hypertarget{vfs}{}%

% ------------------------   
% Section 
\section{The Samba (Posix) VFS layer}
\label{id2464697}\hypertarget{id2464697}{}%

While most of Samba deployments are done using POSIX-compatible operating systems, there is clearly more to a file system than what is required by POSIX when it comes to adopting semantics of NT file system. Since Samba 2.2 all file-system related operations go through an abstraction layer for virtual file system (VFS) that is modelled after both POSIX and additional functions needed to transform NTFS semantics.

This abstraction layer now provides more features than a regular POSIX file system could fill in. It is not required that all of them should be implemented by your particular file system. However, when those features are available, Samba would advertize them to a CIFS client and they might be used by an application and in case of Windows client that might mean a client expects even more additional functionality when it encounters those features. There is a practical reason to allow handling of this snowfall without modifying the Samba core and it is fulfilled by providing an infrastructure to dynamically load VFS modules at run time.

Each VFS module could implement a number of VFS operations. The way it does it is irrelevant, only two things actually matter: whether specific implementation wants to cooperate with other modules' implementations or not, and whether module needs to store additional information that is specific to a context it is operating in. Multiple VFS modules could be loaded at the same time and it is even possible to load several instances of the same VFS module with different parameters.
\subsection{The general interface}
\label{id2477574}\hypertarget{id2477574}{}%

A VFS module has three major components: 
\begin{itemize}
%--- Item
\item {\em{An initialization function}} that is called during the module load to register implemented operations.

%--- Item
\item {\em{An operations table}} representing a mapping between statically defined module functions and VFS layer operations.

%--- Item
\item {\em{Module functions}} that do actual work.
\end{itemize}


While this structure has been first applied to the VFS subsystem, it is now commonly used across all Samba 3 subsystems that support loadable modules. In fact, one module could provide a number of interfaces to different subsystems by exposing different {\em{operation tables}} through separate {\em{initialization functions}}.

{\em{An initialization function}} is used to register module with Samba run-time. As Samba internal structures and API are changed over lifetime, each released version has a VFS interface version that is increased as VFS development progresses or any of underlying Samba structures are changed in binary-incompatible way. When VFS module is compiled in, VFS interface version of that Samba environment is embedded into the module's binary object and is checked by the Samba core upon module load. If VFS interface number reported by the module isn't the same Samba core knows about, version conflict is detected and module dropped to avoid any potential memory corruption when accessing (changed) Samba structures.

Therefore, initialization function passes three parameters to the VFS registration function, {\texttt{\docbookhyphenatedot{smb\_\dbz{}register\_\dbz{}vfs(\dbz{})\dbz{}}}} 
\begin{itemize}
%--- Item
\item {\em{interface version number}}, as constant {\texttt{\docbookhyphenatedot{SMB\_\dbz{}VFS\_\dbz{}INTERFACE\_\dbz{}VERSION}}},

%--- Item
\item {\em{module name}}, under which Samba core will know it, and

%--- Item
\item {\em{an operations' table}}.
\end{itemize}


The {\em{operations' table}} defines which functions in the module would correspond to specific VFS operations and how those functions would co-operate with the rest of VFS subsystem. Each operation could perform in a following ways: 
\begin{itemize}
%--- Item
\item {\em{transparent}}, meaning that while operation is overriden, the module will still call a previous implementation, before or after its own action. This mode is indicated by the constant {\texttt{\docbookhyphenatedot{SMB\_\dbz{}VFS\_\dbz{}LAYER\_\dbz{}TRANSPARENT}}};

%--- Item
\item {\em{opaque}}, for the implementations that are terminating sequence of actions. For example, it is used to implement POSIX operation on top of non-POSIX file system or even not a file system at all, like a database for a personal audio collection. Use constant {\texttt{\docbookhyphenatedot{SMB\_\dbz{}VFS\_\dbz{}LAYER\_\dbz{}OPAQUE}}} for this mode;

%--- Item
\item {\em{splitter}}, a way when some file system activity is done in addition to the transparently calling previous implentation. This usually involves mangling the result of that call before returning it back to the caller. This mode is selected by {\texttt{\docbookhyphenatedot{SMB\_\dbz{}VFS\_\dbz{}LAYER\_\dbz{}SPLITTER}}} constant;

%--- Item
\item {\em{logger}} does not change anything or performs any additional VFS operations. When {\em{logger}} module acts, information about operations is logged somewhere using an external facility (or Samba's own debugging tools) but not the VFS layer. In order to describe this type of activity use constant {\texttt{\docbookhyphenatedot{SMB\_\dbz{}VFS\_\dbz{}LAYER\_\dbz{}LOGGER}}};

%--- Item
\item On contrary, {\em{scanner}} module does call other VFS operations while processing the data that goes through the system. This type of operation is indicated by the {\texttt{\docbookhyphenatedot{SMB\_\dbz{}VFS\_\dbz{}LAYER\_\dbz{}SCANNER}}} constant.
\end{itemize}


Fundamentally, there are three types: {\em{transparent}}, {\em{opaque}}, and {\em{logger}}. {\em{Splitter}} and {\em{scanner}} may confuse developers (and indeed they are confused as our experience has shown) but this separation is to better expose the nature of a module's actions. Most of modules developed so far are either one of those three fundamental types with transparent and opaque being prevalent.

Each VFS operation has a vfs\_op\_type, a function pointer and a handle pointer in the struct vfs\_ops and tree macros to make it easier to call the operations. (Take a look at {\texttt{\docbookhyphenatefilename{include/\dbz{}vfs.\dbz{}h}}} and {\texttt{\docbookhyphenatefilename{include/\dbz{}vfs\_\dbz{}macros.\dbz{}h}}}.)


\begin{Verbatim}[]

typedef enum _vfs_op_type {
	SMB_VFS_OP_NOOP = -1,

	...

	/* File operations */

	SMB_VFS_OP_OPEN,
	SMB_VFS_OP_CLOSE,
	SMB_VFS_OP_READ,
	SMB_VFS_OP_WRITE,
	SMB_VFS_OP_LSEEK,
	SMB_VFS_OP_SENDFILE,

	...

	SMB_VFS_OP_LAST
} vfs_op_type;

\end{Verbatim}


This struct contains the function and handle pointers for all operations.
\begin{Verbatim}[]

struct vfs_ops {
	struct vfs_fn_pointers {
		...
		
		/* File operations */
		
		int (*open)(struct vfs_handle_struct *handle,
			struct connection_struct *conn,
			const char *fname, int flags, mode_t mode);
		int (*close)(struct vfs_handle_struct *handle,
			struct files_struct *fsp, int fd);
		ssize_t (*read)(struct vfs_handle_struct *handle, 
			struct files_struct *fsp, int fd, void *data, size_t n);
		ssize_t (*write)(struct vfs_handle_struct *handle, 
			struct files_struct *fsp, int fd, 
			const void *data, size_t n);
		SMB_OFF_T (*lseek)(struct vfs_handle_struct *handle, 
			struct files_struct *fsp, int fd, 
			SMB_OFF_T offset, int whence);
		ssize_t (*sendfile)(struct vfs_handle_struct *handle, 
			int tofd, files_struct *fsp, int fromfd, 
			const DATA_BLOB *header, SMB_OFF_T offset, size_t count);

		...
	} ops;
	
	struct vfs_handles_pointers {
		...
		
		/* File operations */
		
		struct vfs_handle_struct *open;
		struct vfs_handle_struct *close;
		struct vfs_handle_struct *read;
		struct vfs_handle_struct *write;
		struct vfs_handle_struct *lseek;
		struct vfs_handle_struct *sendfile;
		
		...
	} handles;
};

\end{Verbatim}


This macros SHOULD be used to call any vfs operation. DO NOT ACCESS conn-\textgreater{}vfs.ops.* directly !!! 
\begin{Verbatim}[]

...
	
/* File operations */
#define SMB_VFS_OPEN(conn, fname, flags, mode) \
	((conn)->vfs.ops.open((conn)->vfs.handles.open,\
	 (conn), (fname), (flags), (mode)))
#define SMB_VFS_CLOSE(fsp, fd) \
	((fsp)->conn->vfs.ops.close(\
	(fsp)->conn->vfs.handles.close, (fsp), (fd)))
#define SMB_VFS_READ(fsp, fd, data, n) \
	((fsp)->conn->vfs.ops.read(\
	(fsp)->conn->vfs.handles.read,\
	 (fsp), (fd), (data), (n)))
#define SMB_VFS_WRITE(fsp, fd, data, n) \
	((fsp)->conn->vfs.ops.write(\
	(fsp)->conn->vfs.handles.write,\
	 (fsp), (fd), (data), (n)))
#define SMB_VFS_LSEEK(fsp, fd, offset, whence) \
	((fsp)->conn->vfs.ops.lseek(\
	(fsp)->conn->vfs.handles.lseek,\
	 (fsp), (fd), (offset), (whence)))
#define SMB_VFS_SENDFILE(tofd, fsp, fromfd, header, offset, count) \
	((fsp)->conn->vfs.ops.sendfile(\
	(fsp)->conn->vfs.handles.sendfile,\
	 (tofd), (fsp), (fromfd), (header), (offset), (count)))

...

\end{Verbatim}

\subsection{Possible VFS operation layers}
\label{id2503002}\hypertarget{id2503002}{}%

These values are used by the VFS subsystem when building the conn-\textgreater{}vfs and conn-\textgreater{}vfs\_opaque structs for a connection with multiple VFS modules. Internally, Samba differentiates only opaque and transparent layers at this process. Other types are used for providing better diagnosing facilities.

Most modules will provide transparent layers. Opaque layer is for modules which implement actual file system calls (like DB-based VFS). For example, default POSIX VFS which is built in into Samba is an opaque VFS module.

Other layer types (logger, splitter, scanner) were designed to provide different degree of transparency and for diagnosing VFS module behaviour.

Each module can implement several layers at the same time provided that only one layer is used per each operation.


\begin{Verbatim}[]

typedef enum _vfs_op_layer {
	SMB_VFS_LAYER_NOOP = -1,	/* - For using in VFS module to indicate end of array */
					/*   of operations description */
	SMB_VFS_LAYER_OPAQUE = 0,	/* - Final level, does not call anything beyond itself */
	SMB_VFS_LAYER_TRANSPARENT,	/* - Normal operation, calls underlying layer after */
					/*   possibly changing passed data */
	SMB_VFS_LAYER_LOGGER,		/* - Logs data, calls underlying layer, logging may not */
					/*   use Samba VFS */
	SMB_VFS_LAYER_SPLITTER,		/* - Splits operation, calls underlying layer _and_ own facility, */
					/*   then combines result */
	SMB_VFS_LAYER_SCANNER		/* - Checks data and possibly initiates additional */
					/*   file activity like logging to files _inside_ samba VFS */
} vfs_op_layer;

\end{Verbatim}


% ------------------------   
% Section 
\section{The Interaction between the Samba VFS subsystem and the modules}
\label{id2502696}\hypertarget{id2502696}{}%
\subsection{Initialization and registration}
\label{id2502704}\hypertarget{id2502704}{}%

As each Samba module a VFS module should have a 
\begin{Verbatim}[]
NTSTATUS vfs_example_init(void);
\end{Verbatim}
 function if it's staticly linked to samba or 
\begin{Verbatim}[]
NTSTATUS init_module(void);
\end{Verbatim}
 function if it's a shared module.

This should be the only non static function inside the module. Global variables should also be static!

The module should register its functions via the 
\begin{Verbatim}[]

NTSTATUS smb_register_vfs(int version, const char *name, vfs_op_tuple *vfs_op_tuples);

\end{Verbatim}
 function.

\begin{description}
% \null and \mbox are tricks to induce different typesetting decisions
\item[{version}]\null{}
should be filled with SMB\_VFS\_INTERFACE\_VERSION
% \null and \mbox are tricks to induce different typesetting decisions
\item[{name}]\null{}
this is the name witch can be listed in the {\bfseries{vfs objects}} parameter to use this module.
% \null and \mbox are tricks to induce different typesetting decisions
\item[{vfs\_op\_tuples}]\null{}
this is an array of vfs\_op\_tuple's. (vfs\_op\_tuples is descripted in details below.)
\end{description}

For each operation the module wants to provide it has a entry in the vfs\_op\_tuple array.

\begin{Verbatim}[]

typedef struct _vfs_op_tuple {
	void* op;
	vfs_op_type type;
	vfs_op_layer layer;
} vfs_op_tuple;

\end{Verbatim}

\begin{description}
% \null and \mbox are tricks to induce different typesetting decisions
\item[{op}]\null{}
the function pointer to the specified function.
% \null and \mbox are tricks to induce different typesetting decisions
\item[{type}]\null{}
the vfs\_op\_type of the function to specified witch operation the function provides.
% \null and \mbox are tricks to induce different typesetting decisions
\item[{layer}]\null{}
the vfs\_op\_layer in whitch the function operates.
\end{description}

A simple example:

\begin{Verbatim}[]

static vfs_op_tuple example_op_tuples[] = {	
	{SMB_VFS_OP(example_connect),	SMB_VFS_OP_CONNECT,	SMB_VFS_LAYER_TRANSPARENT},
	{SMB_VFS_OP(example_disconnect),	SMB_VFS_OP_DISCONNECT,	SMB_VFS_LAYER_TRANSPARENT},

	{SMB_VFS_OP(example_rename),	SMB_VFS_OP_RENAME,	SMB_VFS_LAYER_OPAQUE},

	/* This indicates the end of the array */
	{SMB_VFS_OP(NULL),				SMB_VFS_OP_NOOP,	SMB_VFS_LAYER_NOOP}
};

NTSTATUS init_module(void)
{
	return smb_register_vfs(SMB_VFS_INTERFACE_VERSION, "example", example_op_tuples);
}

\end{Verbatim}
\subsection{How the Modules handle per connection data}
\label{id2502860}\hypertarget{id2502860}{}%

Each VFS function has as first parameter a pointer to the modules vfs\_handle\_struct.

\begin{Verbatim}[]

typedef struct vfs_handle_struct {
	struct vfs_handle_struct  *next, *prev;
	const char *param;
	struct vfs_ops vfs_next;
	struct connection_struct *conn;
	void *data;
	void (*free_data)(void **data);
} vfs_handle_struct;

\end{Verbatim}

\begin{description}
% \null and \mbox are tricks to induce different typesetting decisions
\item[{param}]\null{}
this is the module parameter specified in the {\bfseries{vfs objects}} parameter.

e.g. for 'vfs objects = example:test' param would be "test".
% \null and \mbox are tricks to induce different typesetting decisions
\item[{vfs\_next}]\null{}
This vfs\_ops struct contains the information for calling the next module operations. Use the SMB\_VFS\_NEXT\_* macros to call a next module operations and don't access handle-\textgreater{}vfs\_next.ops.* directly!
% \null and \mbox are tricks to induce different typesetting decisions
\item[{conn}]\null{}
This is a pointer back to the connection\_struct to witch the handle belongs.
% \null and \mbox are tricks to induce different typesetting decisions
\item[{data}]\null{}
This is a pointer for holding module private data. You can alloc data with connection life time on the handle-\textgreater{}conn-\textgreater{}mem\_ctx TALLOC\_CTX. But you can also manage the memory allocation yourself.
% \null and \mbox are tricks to induce different typesetting decisions
\item[{free\_data}]\null{}
This is a function pointer to a function that free's the module private data. If you talloc your private data on the TALLOC\_CTX handle-\textgreater{}conn-\textgreater{}mem\_ctx, you can set this function pointer to NULL.
\end{description}

Some useful MACROS for handle private data.

\begin{Verbatim}[]

#define SMB_VFS_HANDLE_GET_DATA(handle, datap, type, ret) { \
	if (!(handle)||((datap=(type *)(handle)->data)==NULL)) { \
		DEBUG(0,("%s() failed to get vfs_handle->data!\n",FUNCTION_MACRO)); \
		ret; \
	} \
}

#define SMB_VFS_HANDLE_SET_DATA(handle, datap, free_fn, type, ret) { \
	if (!(handle)) { \
		DEBUG(0,("%s() failed to set handle->data!\n",FUNCTION_MACRO)); \
		ret; \
	} else { \
		if ((handle)->free_data) { \
			(handle)->free_data(&(handle)->data); \
		} \
		(handle)->data = (void *)datap; \
		(handle)->free_data = free_fn; \
	} \
}

#define SMB_VFS_HANDLE_FREE_DATA(handle) { \
	if ((handle) && (handle)->free_data) { \
		(handle)->free_data(&(handle)->data); \
	} \
}

\end{Verbatim}

How SMB\_VFS\_LAYER\_TRANSPARENT functions can call the SMB\_VFS\_LAYER\_OPAQUE functions.

The easiest way to do this is to use the SMB\_VFS\_OPAQUE\_* macros.

\begin{Verbatim}[]

...
/* File operations */
#define SMB_VFS_OPAQUE_OPEN(conn, fname, flags, mode) \
	((conn)->vfs_opaque.ops.open(\
	(conn)->vfs_opaque.handles.open,\
	 (conn), (fname), (flags), (mode)))
#define SMB_VFS_OPAQUE_CLOSE(fsp, fd) \
	((fsp)->conn->vfs_opaque.ops.close(\
	(fsp)->conn->vfs_opaque.handles.close,\
	 (fsp), (fd)))
#define SMB_VFS_OPAQUE_READ(fsp, fd, data, n) \
	((fsp)->conn->vfs_opaque.ops.read(\
	(fsp)->conn->vfs_opaque.handles.read,\
	 (fsp), (fd), (data), (n)))
#define SMB_VFS_OPAQUE_WRITE(fsp, fd, data, n) \
	((fsp)->conn->vfs_opaque.ops.write(\
	(fsp)->conn->vfs_opaque.handles.write,\
	 (fsp), (fd), (data), (n)))
#define SMB_VFS_OPAQUE_LSEEK(fsp, fd, offset, whence) \
	((fsp)->conn->vfs_opaque.ops.lseek(\
	(fsp)->conn->vfs_opaque.handles.lseek,\
	 (fsp), (fd), (offset), (whence)))
#define SMB_VFS_OPAQUE_SENDFILE(tofd, fsp, fromfd, header, offset, count) \
	((fsp)->conn->vfs_opaque.ops.sendfile(\
	(fsp)->conn->vfs_opaque.handles.sendfile,\
	 (tofd), (fsp), (fromfd), (header), (offset), (count)))
...

\end{Verbatim}

How SMB\_VFS\_LAYER\_TRANSPARENT functions can call the next modules functions.

The easiest way to do this is to use the SMB\_VFS\_NEXT\_* macros.

\begin{Verbatim}[]

...
/* File operations */
#define SMB_VFS_NEXT_OPEN(handle, conn, fname, flags, mode) \
	((handle)->vfs_next.ops.open(\
	(handle)->vfs_next.handles.open,\
	 (conn), (fname), (flags), (mode)))
#define SMB_VFS_NEXT_CLOSE(handle, fsp, fd) \
	((handle)->vfs_next.ops.close(\
	(handle)->vfs_next.handles.close,\
	 (fsp), (fd)))
#define SMB_VFS_NEXT_READ(handle, fsp, fd, data, n) \
	((handle)->vfs_next.ops.read(\
	(handle)->vfs_next.handles.read,\
	 (fsp), (fd), (data), (n)))
#define SMB_VFS_NEXT_WRITE(handle, fsp, fd, data, n) \
	((handle)->vfs_next.ops.write(\
	(handle)->vfs_next.handles.write,\
	 (fsp), (fd), (data), (n)))
#define SMB_VFS_NEXT_LSEEK(handle, fsp, fd, offset, whence) \
	((handle)->vfs_next.ops.lseek(\
	(handle)->vfs_next.handles.lseek,\
	 (fsp), (fd), (offset), (whence)))
#define SMB_VFS_NEXT_SENDFILE(handle, tofd, fsp, fromfd, header, offset, count) \
	((handle)->vfs_next.ops.sendfile(\
	(handle)->vfs_next.handles.sendfile,\
	 (tofd), (fsp), (fromfd), (header), (offset), (count)))
...

\end{Verbatim}

% ------------------------   
% Section 
\section{Upgrading to the New VFS Interface}
\label{id2504179}\hypertarget{id2504179}{}%
\subsection{Upgrading from 2.2.* and 3.0alpha modules}
\label{id2504185}\hypertarget{id2504185}{}%

\begin{enumerate}
%--- Item
\item 
Add "vfs\_handle\_struct *handle, " as first parameter to all vfs operation functions. e.g. example\_connect(connection\_struct *conn, const char *service, const char *user); -\textgreater{} example\_connect(vfs\_handle\_struct *handle, connection\_struct *conn, const char *service, const char *user);


%--- Item
\item 
Replace "default\_vfs\_ops." with "smb\_vfs\_next\_". e.g. default\_vfs\_ops.connect(conn, service, user); -\textgreater{} smb\_vfs\_next\_connect(conn, service, user);


%--- Item
\item 
Uppercase all "smb\_vfs\_next\_*" functions. e.g. smb\_vfs\_next\_connect(conn, service, user); -\textgreater{} SMB\_VFS\_NEXT\_CONNECT(conn, service, user);


%--- Item
\item 
Add "handle, " as first parameter to all SMB\_VFS\_NEXT\_*() calls. e.g. SMB\_VFS\_NEXT\_CONNECT(conn, service, user); -\textgreater{} SMB\_VFS\_NEXT\_CONNECT(handle, conn, service, user);


%--- Item
\item 
(Only for 2.2.* modules) Convert the old struct vfs\_ops example\_ops to a vfs\_op\_tuple example\_op\_tuples[] array. e.g. 
\begin{Verbatim}[]

struct vfs_ops example_ops = {
	/* Disk operations */
	example_connect,		/* connect */
	example_disconnect,		/* disconnect */
	NULL,				/* disk free *
	/* Directory operations */
	NULL,				/* opendir */
	NULL,				/* readdir */
	NULL,				/* mkdir */
	NULL,				/* rmdir */
	NULL,				/* closedir */
	/* File operations */
	NULL,				/* open */
	NULL,				/* close */
	NULL,				/* read  */
	NULL,				/* write */
	NULL,				/* lseek */
	NULL,				/* sendfile */
	NULL,				/* rename */
	NULL,				/* fsync */
	example_stat,			/* stat  */
	example_fstat,			/* fstat */
	example_lstat,			/* lstat */
	NULL,				/* unlink */
	NULL,				/* chmod */
	NULL,				/* fchmod */
	NULL,				/* chown */
	NULL,				/* fchown */
	NULL,				/* chdir */
	NULL,				/* getwd */
	NULL,				/* utime */
	NULL,				/* ftruncate */
	NULL,				/* lock */
	NULL,				/* symlink */
	NULL,				/* readlink */
	NULL,				/* link */
	NULL,				/* mknod */
	NULL,				/* realpath */
	NULL,				/* fget_nt_acl */
	NULL,				/* get_nt_acl */
	NULL,				/* fset_nt_acl */
	NULL,				/* set_nt_acl */

	NULL,				/* chmod_acl */
	NULL,				/* fchmod_acl */

	NULL,				/* sys_acl_get_entry */
	NULL,				/* sys_acl_get_tag_type */
	NULL,				/* sys_acl_get_permset */
	NULL,				/* sys_acl_get_qualifier */
	NULL,				/* sys_acl_get_file */
	NULL,				/* sys_acl_get_fd */
	NULL,				/* sys_acl_clear_perms */
	NULL,				/* sys_acl_add_perm */
	NULL,				/* sys_acl_to_text */
	NULL,				/* sys_acl_init */
	NULL,				/* sys_acl_create_entry */
	NULL,				/* sys_acl_set_tag_type */
	NULL,				/* sys_acl_set_qualifier */
	NULL,				/* sys_acl_set_permset */
	NULL,				/* sys_acl_valid */
	NULL,				/* sys_acl_set_file */
	NULL,				/* sys_acl_set_fd */
	NULL,				/* sys_acl_delete_def_file */
	NULL,				/* sys_acl_get_perm */
	NULL,				/* sys_acl_free_text */
	NULL,				/* sys_acl_free_acl */
	NULL				/* sys_acl_free_qualifier */
};

\end{Verbatim}
 -\textgreater{} 
\begin{Verbatim}[]
 
static vfs_op_tuple example_op_tuples[] = {
	{SMB_VFS_OP(example_connect),	SMB_VFS_OP_CONNECT,	SMB_VFS_LAYER_TRANSPARENT},
	{SMB_VFS_OP(example_disconnect),	SMB_VFS_OP_DISCONNECT,	SMB_VFS_LAYER_TRANSPARENT},
	
	{SMB_VFS_OP(example_fstat), 	SMB_VFS_OP_FSTAT,	SMB_VFS_LAYER_TRANSPARENT},
	{SMB_VFS_OP(example_stat),		SMB_VFS_OP_STAT,	SMB_VFS_LAYER_TRANSPARENT},
	{SMB_VFS_OP(example_lstat), 	SMB_VFS_OP_LSTAT,	SMB_VFS_LAYER_TRANSPARENT},

	{SMB_VFS_OP(NULL),				SMB_VFS_OP_NOOP,	SMB_VFS_LAYER_NOOP}
};

\end{Verbatim}



%--- Item
\item 
Move the example\_op\_tuples[] array to the end of the file.


%--- Item
\item 
Add the init\_module() function at the end of the file. e.g. 
\begin{Verbatim}[]

NTSTATUS init_module(void)
{
	return smb_register_vfs(SMB_VFS_INTERFACE_VERSION,"example",example_op_tuples);
}

\end{Verbatim}



%--- Item
\item 
Check if your vfs\_init() function does more then just prepare the vfs\_ops structs or remember the struct smb\_vfs\_handle\_struct. 
\begin{tabular}{l}

If NOT you can remove the vfs\_init() function.  \\
If YES decide if you want to move the code to the example\_connect() operation or to the init\_module(). And then remove vfs\_init(). e.g. a debug class registration should go into init\_module() and the allocation of private data should go to example\_connect().  \\
\end{tabular}



%--- Item
\item 
(Only for 3.0alpha* modules) Check if your vfs\_done() function contains needed code. 
\begin{tabular}{l}

If NOT you can remove the vfs\_done() function.  \\
If YES decide if you can move the code to the example\_disconnect() operation. Otherwise register a SMB\_EXIT\_EVENT with smb\_register\_exit\_event(); (Described in the \hyperlink{modules}{Chapter {\ref{modules}}, {``}Modules{''}}) And then remove vfs\_done(). e.g. the freeing of private data should go to example\_disconnect().  \\
\end{tabular}



%--- Item
\item 
Check if you have any global variables left. Decide if it wouldn't be better to have this data on a connection basis. 
\begin{tabular}{l}

If NOT leave them as they are. (e.g. this could be the variable for the private debug class.)  \\
If YES pack all this data into a struct. You can use handle-\textgreater{}data to point to such a struct on a per connection basis.  \\
\end{tabular}
 e.g. if you have such a struct: 
\begin{Verbatim}[]
    
struct example_privates {
	char *some_string;
	int db_connection;
};

\end{Verbatim}
 first way of doing it: 
\begin{Verbatim}[]

static int example_connect(vfs_handle_struct *handle,
	connection_struct *conn, const char *service, 
	const char* user)
{
	struct example_privates *data = NULL;

	/* alloc our private data */
	data = (struct example_privates *)talloc_zero(conn->mem_ctx, sizeof(struct example_privates));
	if (!data) {
		DEBUG(0,("talloc_zero() failed\n"));
		return -1;
	}

	/* init out private data */
	data->some_string = talloc_strdup(conn->mem_ctx,"test");
	if (!data->some_string) {
		DEBUG(0,("talloc_strdup() failed\n"));
		return -1;
	}

	data->db_connection = open_db_conn();

	/* and now store the private data pointer in handle->data
	 * we don't need to specify a free_function here because
	 * we use the connection TALLOC context.
	 * (return -1 if something failed.)
	 */
	VFS_HANDLE_SET_DATA(handle, data, NULL, struct example_privates, return -1);

	return SMB_VFS_NEXT_CONNECT(handle,conn,service,user);
}

static int example_close(vfs_handle_struct *handle, files_struct *fsp, int fd)
{
	struct example_privates *data = NULL;
	
	/* get the pointer to our private data
	 * return -1 if something failed
	 */
	SMB_VFS_HANDLE_GET_DATA(handle, data, struct example_privates, return -1);
	
	/* do something here...*/
	DEBUG(0,("some_string: %s\n",data->some_string));
	
	return SMB_VFS_NEXT_CLOSE(handle, fsp, fd);
}

\end{Verbatim}
 second way of doing it: 
\begin{Verbatim}[]

static void free_example_privates(void **datap)
{
	struct example_privates *data = (struct example_privates *)*datap;
	
	SAFE_FREE(data->some_string);
	SAFE_FREE(data);
	
	*datap = NULL;
	
	return;
}

static int example_connect(vfs_handle_struct *handle, 
	connection_struct *conn, const char *service, 
	const char* user)
{
	struct example_privates *data = NULL;

	/* alloc our private data */
	data = (struct example_privates *)malloc(sizeof(struct example_privates));
	if (!data) {
		DEBUG(0,("malloc() failed\n"));
		return -1;
	}

	/* init out private data */
	data->some_string = strdup("test");
	if (!data->some_string) {
		DEBUG(0,("strdup() failed\n"));
		return -1;
	}

	data->db_connection = open_db_conn();

	/* and now store the private data pointer in handle->data
	 * we need to specify a free_function because we used malloc() and strdup().
	 * (return -1 if something failed.)
	 */
	SMB_VFS_HANDLE_SET_DATA(handle, data, free_example_privates, struct example_privates, return -1);

	return SMB_VFS_NEXT_CONNECT(handle,conn,service,user);
}

static int example_close(vfs_handle_struct *handle, files_struct *fsp, int fd)
{
	struct example_privates *data = NULL;
	
	/* get the pointer to our private data
	 * return -1 if something failed
	 */
	SMB_VFS_HANDLE_GET_DATA(handle, data, struct example_privates, return -1);
	
	/* do something here...*/
	DEBUG(0,("some_string: %s\n",data->some_string));
	
	return SMB_VFS_NEXT_CLOSE(handle, fsp, fd);
}

\end{Verbatim}



%--- Item
\item 
To make it easy to build 3rd party modules it would be useful to provide configure.in, (configure), install.sh and Makefile.in with the module. (Take a look at the example in {\texttt{\docbookhyphenatefilename{examples/\dbz{}VFS}}}.)

The configure script accepts {\texttt{\docbookhyphenatedot{--with-samba-source}}} to specify the path to the samba source tree. It also accept {\texttt{\docbookhyphenatedot{--enable-developer}}} which lets the compiler give you more warnings.

The idea is that you can extend this {\texttt{\docbookhyphenatefilename{configure.\dbz{}in}}} and {\texttt{\docbookhyphenatefilename{Makefile.\dbz{}in}}} scripts for your module.


%--- Item
\item 
Compiling \& Testing... 
\begin{tabular}{l}

{\ttfamily\bfseries{\docbookhyphenatedot{.\dbz{}/\dbz{}configure {\texttt{\docbookhyphenatedot{-\dbz{}-\dbz{}enable-\dbz{}developer}}}}}} ...  \\
{\ttfamily\bfseries{\docbookhyphenatedot{make}}}  \\
Try to fix all compiler warnings  \\
{\ttfamily\bfseries{\docbookhyphenatedot{make}}}  \\
Testing, Testing, Testing ...  \\
\end{tabular}


\end{enumerate}

% ------------------------   
% Section 
\section{Some Notes}
\label{id2501988}\hypertarget{id2501988}{}%
\subsection{Implement TRANSPARENT functions}
\label{id2501994}\hypertarget{id2501994}{}%

Avoid writing functions like this: 
\begin{Verbatim}[]

static int example_close(vfs_handle_struct *handle, files_struct *fsp, int fd)
{
	return SMB_VFS_NEXT_CLOSE(handle, fsp, fd);
}

\end{Verbatim}
 Overload only the functions you really need to!
\subsection{Implement OPAQUE functions}
\label{id2502014}\hypertarget{id2502014}{}%

If you want to just implement a better version of a default samba opaque function (e.g. like a disk\_free() function for a special filesystem) it's ok to just overload that specific function.

If you want to implement a database filesystem or something different from a posix filesystem. Make sure that you overload every vfs operation!!!

Functions your FS does not support should be overloaded by something like this: e.g. for a readonly filesystem.

\begin{Verbatim}[]

static int example_rename(vfs_handle_struct *handle, connection_struct *conn,
			char *oldname, char *newname)
{
	DEBUG(10,("function rename() not allowed on vfs 'example'\n"));
	errno = ENOSYS;
	return -1;
}

\end{Verbatim}

% -------------------------------------------------------------
% Chapter The smb.conf file 
% ------------------------------------------------------------- 	
\chapter{The smb.conf file}
\label{parsing}\hypertarget{parsing}{}%

% ------------------------   
% Section 
\section{Lexical Analysis}
\label{id2474442}\hypertarget{id2474442}{}%

Basically, the file is processed on a line by line basis. There are four types of lines that are recognized by the lexical analyzer (params.c):

\begin{enumerate}
%--- Item
\item 
Blank lines - Lines containing only whitespace.


%--- Item
\item 
Comment lines - Lines beginning with either a semi-colon or a pound sign (';' or '\#').


%--- Item
\item 
Section header lines - Lines beginning with an open square bracket ('[').


%--- Item
\item 
Parameter lines - Lines beginning with any other character. (The default line type.)

\end{enumerate}

The first two are handled exclusively by the lexical analyzer, which ignores them. The latter two line types are scanned for

\begin{enumerate}
%--- Item
\item 
- Section names


%--- Item
\item 
- Parameter names


%--- Item
\item 
- Parameter values

\end{enumerate}

These are the only tokens passed to the parameter loader (loadparm.c). Parameter names and values are divided from one another by an equal sign: '='.
\subsection{Handling of Whitespace}
\label{id2500834}\hypertarget{id2500834}{}%

Whitespace is defined as all characters recognized by the isspace() function (see ctype(3C)) except for the newline character ('\textbackslash n') The newline is excluded because it identifies the end of the line.

\begin{enumerate}
%--- Item
\item 
The lexical analyzer scans past white space at the beginning of a line.


%--- Item
\item 
Section and parameter names may contain internal white space. All whitespace within a name is compressed to a single space character.


%--- Item
\item 
Internal whitespace within a parameter value is kept verbatim with the exception of carriage return characters ('\textbackslash r'), all of which are removed.


%--- Item
\item 
Leading and trailing whitespace is removed from names and values.

\end{enumerate}
\subsection{Handling of Line Continuation}
\label{id2500880}\hypertarget{id2500880}{}%

Long section header and parameter lines may be extended across multiple lines by use of the backslash character ('\textbackslash \textbackslash '). Line continuation is ignored for blank and comment lines.

If the last (non-whitespace) character within a section header or on a parameter line is a backslash, then the next line will be (logically) concatonated with the current line by the lexical analyzer. For example:


\begin{Verbatim}[]

	param name = parameter value string \
	with line continuation.

\end{Verbatim}


Would be read as


\begin{Verbatim}[]

    param name = parameter value string     with line continuation.

\end{Verbatim}


Note that there are five spaces following the word 'string', representing the one space between 'string' and '\textbackslash \textbackslash ' in the top line, plus the four preceeding the word 'with' in the second line. (Yes, I'm counting the indentation.)

Line continuation characters are ignored on blank lines and at the end of comments. They are *only* recognized within section and parameter lines.
\subsection{Line Continuation Quirks}
\label{id2502424}\hypertarget{id2502424}{}%

Note the following example:


\begin{Verbatim}[]

	param name = parameter value string \
    \
    with line continuation.

\end{Verbatim}


The middle line is *not* parsed as a blank line because it is first concatonated with the top line. The result is


\begin{Verbatim}[]

param name = parameter value string         with line continuation.

\end{Verbatim}


The same is true for comment lines.


\begin{Verbatim}[]

	param name = parameter value string \
	; comment \
    with a comment.

\end{Verbatim}


This becomes:


\begin{Verbatim}[]

param name = parameter value string     ; comment     with a comment.

\end{Verbatim}


On a section header line, the closing bracket (']') is considered a terminating character, and the rest of the line is ignored. The lines


\begin{Verbatim}[]

	[ section   name ] garbage \
    param  name  = value

\end{Verbatim}


are read as


\begin{Verbatim}[]

	[section name]
    param name = value

\end{Verbatim}


% ------------------------   
% Section 
\section{Syntax}
\label{id2502504}\hypertarget{id2502504}{}%

The syntax of the smb.conf file is as follows:


\begin{Verbatim}[]

  <file>            :==  { <section> } EOF
  <section>         :==  <section header> { <parameter line> }
  <section header>  :==  '[' NAME ']'
  <parameter line>  :==  NAME '=' VALUE NL

\end{Verbatim}


Basically, this means that

\begin{enumerate}
%--- Item
\item 
a file is made up of zero or more sections, and is terminated by an EOF (we knew that).


%--- Item
\item 
A section is made up of a section header followed by zero or more parameter lines.


%--- Item
\item 
A section header is identified by an opening bracket and terminated by the closing bracket. The enclosed NAME identifies the section.


%--- Item
\item 
A parameter line is divided into a NAME and a VALUE. The *first* equal sign on the line separates the NAME from the VALUE. The VALUE is terminated by a newline character (NL = '\textbackslash n').

\end{enumerate}
\subsection{About params.c}
\label{id2408675}\hypertarget{id2408675}{}%

The parsing of the config file is a bit unusual if you are used to lex, yacc, bison, etc. Both lexical analysis (scanning) and parsing are performed by params.c. Values are loaded via callbacks to loadparm.c.

% -------------------------------------------------------------
% Chapter Samba WINS Internals 
% ------------------------------------------------------------- 	
\chapter{Samba WINS Internals}
\label{wins}\hypertarget{wins}{}%

% ------------------------   
% Section 
\section{WINS Failover}
\label{id2486163}\hypertarget{id2486163}{}%

The current Samba codebase possesses the capability to use groups of WINS servers that share a common namespace for NetBIOS name registration and resolution. The formal parameter syntax is


\begin{Verbatim}[]

	WINS_SERVER_PARAM 	= SERVER [ SEPARATOR SERVER_LIST ]
	WINS_SERVER_PARAM 	= "wins server"
	SERVER 			= ADDR[:TAG]
	ADDR 			= ip_addr | fqdn
	TAG 			= string
	SEPARATOR		= comma | \s+
	SERVER_LIST		= SERVER [ SEPARATOR SERVER_LIST ]

\end{Verbatim}


A simple example of a valid wins server setting is


\begin{Verbatim}[]

[global]
	wins server = 192.168.1.2 192.168.1.3

\end{Verbatim}


In the event that no TAG is defined in for a SERVER in the list, smbd assigns a default TAG of "*". A TAG is used to group servers of a shared NetBIOS namespace together. Upon startup, nmbd will attempt to register the netbios name value with one server in each tagged group.

An example using tags to group WINS servers together is show here. Note that the use of interface names in the tags is only by convention and is not a technical requirement.


\begin{Verbatim}[]

[global]
	wins server = 192.168.1.2:eth0 192.168.1.3:eth0 192.168.2.2:eth1

\end{Verbatim}


Using this configuration, nmbd would attempt to register the server's NetBIOS name with one WINS server in each group. Because the "eth0" group has two servers, the second server would only be used when a registration (or resolution) request to the first server in that group timed out.

NetBIOS name resolution follows a similar pattern as name registration. When resolving a NetBIOS name via WINS, smbd and other Samba programs will attempt to query a single WINS server in a tagged group until either a positive response is obtained at least once or until a server from every tagged group has responded negatively to the name query request. If a timeout occurs when querying a specific WINS server, that server is marked as down to prevent further timeouts and the next server in the WINS group is contacted. Once marked as dead, Samba will not attempt to contact that server for name registration/resolution queries for a period of 10 minutes.

% -------------------------------------------------------------
% Chapter LanMan and NT Password Encryption 
% ------------------------------------------------------------- 	
\chapter{LanMan and NT Password Encryption}
\label{pwencrypt}\hypertarget{pwencrypt}{}%

% ------------------------   
% Section 
\section{Introduction}
\label{id2485399}\hypertarget{id2485399}{}%

With the development of LanManager and Windows NT compatible password encryption for Samba, it is now able to validate user connections in exactly the same way as a LanManager or Windows NT server.

This document describes how the SMB password encryption algorithm works and what issues there are in choosing whether you want to use it. You should read it carefully, especially the part about security and the "PROS and CONS" section.

% ------------------------   
% Section 
\section{How does it work?}
\label{id2469753}\hypertarget{id2469753}{}%

LanManager encryption is somewhat similar to UNIX password encryption. The server uses a file containing a hashed value of a user's password. This is created by taking the user's plaintext password, capitalising it, and either truncating to 14 bytes or padding to 14 bytes with null bytes. This 14 byte value is used as two 56 bit DES keys to encrypt a 'magic' eight byte value, forming a 16 byte value which is stored by the server and client. Let this value be known as the "hashed password".

Windows NT encryption is a higher quality mechanism, consisting of doing an MD4 hash on a Unicode version of the user's password. This also produces a 16 byte hash value that is non-reversible.

When a client (LanManager, Windows for WorkGroups, Windows 95 or Windows NT) wishes to mount a Samba drive (or use a Samba resource), it first requests a connection and negotiates the protocol that the client and server will use. In the reply to this request the Samba server generates and appends an 8 byte, random value - this is stored in the Samba server after the reply is sent and is known as the "challenge". The challenge is different for every client connection.

The client then uses the hashed password (16 byte values described above), appended with 5 null bytes, as three 56 bit DES keys, each of which is used to encrypt the challenge 8 byte value, forming a 24 byte value known as the "response".

In the SMB call SMBsessionsetupX (when user level security is selected) or the call SMBtconX (when share level security is selected), the 24 byte response is returned by the client to the Samba server. For Windows NT protocol levels the above calculation is done on both hashes of the user's password and both responses are returned in the SMB call, giving two 24 byte values.

The Samba server then reproduces the above calculation, using its own stored value of the 16 byte hashed password (read from the {\texttt{\docbookhyphenatefilename{smbpasswd}}} file - described later) and the challenge value that it kept from the negotiate protocol reply. It then checks to see if the 24 byte value it calculates matches the 24 byte value returned to it from the client.

If these values match exactly, then the client knew the correct password (or the 16 byte hashed value - see security note below) and is thus allowed access. If not, then the client did not know the correct password and is denied access.

Note that the Samba server never knows or stores the cleartext of the user's password - just the 16 byte hashed values derived from it. Also note that the cleartext password or 16 byte hashed values are never transmitted over the network - thus increasing security.

% ------------------------   
% Section 
\section{The smbpasswd file}
\label{id2507225}\hypertarget{id2507225}{}%
\hypertarget{SMBPASSWDFILEFORMAT}{}
In order for Samba to participate in the above protocol it must be able to look up the 16 byte hashed values given a user name. Unfortunately, as the UNIX password value is also a one way hash function (ie. it is impossible to retrieve the cleartext of the user's password given the UNIX hash of it), a separate password file containing this 16 byte value must be kept. To minimise problems with these two password files, getting out of sync, the UNIX {\texttt{\docbookhyphenatefilename{/\dbz{}etc/\dbz{}passwd}}} and the {\texttt{\docbookhyphenatefilename{smbpasswd}}} file, a utility, {\bfseries{mksmbpasswd.sh}}, is provided to generate a smbpasswd file from a UNIX {\texttt{\docbookhyphenatefilename{/\dbz{}etc/\dbz{}passwd}}} file.

To generate the smbpasswd file from your {\texttt{\docbookhyphenatefilename{/\dbz{}etc/\dbz{}passwd}}} file use the following command:

{\texttt{\docbookhyphenatedot{\$}}}{\ttfamily\bfseries{\docbookhyphenatedot{cat /\dbz{}etc/\dbz{}passwd | mksmbpasswd.\dbz{}sh > /\dbz{}usr/\dbz{}local/\dbz{}samba/\dbz{}private/\dbz{}smbpasswd}}}

If you are running on a system that uses NIS, use

{\texttt{\docbookhyphenatedot{\$}}}{\ttfamily\bfseries{\docbookhyphenatedot{ypcat passwd | mksmbpasswd.\dbz{}sh > /\dbz{}usr/\dbz{}local/\dbz{}samba/\dbz{}private/\dbz{}smbpasswd}}}

The {\bfseries{mksmbpasswd.sh}} program is found in the Samba source directory. By default, the smbpasswd file is stored in :

{\texttt{\docbookhyphenatefilename{/\dbz{}usr/\dbz{}local/\dbz{}samba/\dbz{}private/\dbz{}smbpasswd}}}

The owner of the {\texttt{\docbookhyphenatefilename{/\dbz{}usr/\dbz{}local/\dbz{}samba/\dbz{}private/\dbz{}}}} directory should be set to root, and the permissions on it should be set to 0500 ({\bfseries{chmod 500 /usr/local/samba/private}}).

Likewise, the smbpasswd file inside the private directory should be owned by root and the permissions on is should be set to 0600 ({\bfseries{chmod 600 smbpasswd}}).

The format of the smbpasswd file is (The line has been wrapped here. It should appear as one entry per line in your smbpasswd file.)


\begin{Verbatim}[]

username:uid:XXXXXXXXXXXXXXXXXXXXXXXXXXXXXXXX:XXXXXXXXXXXXXXXXXXXXXXXXXXXXXXXX:
	[Account type]:LCT-<last-change-time>:Long name
	
\end{Verbatim}


Although only the {\ttfamily\itshape{\docbookhyphenatedot{username}}}, {\ttfamily\itshape{\docbookhyphenatedot{uid}}}, {\ttfamily\itshape{\docbookhyphenatedot{XXXXXXXXXXXXXXXXXXXXXXXXXXXXXXXX}}}, [{\ttfamily\itshape{\docbookhyphenatedot{Account type}}}] and {\ttfamily\itshape{\docbookhyphenatedot{last-change-time}}} sections are significant and are looked at in the Samba code.

It is {\em{VITALLY}} important that there by 32 'X' characters between the two ':' characters in the XXX sections - the smbpasswd and Samba code will fail to validate any entries that do not have 32 characters between ':' characters. The first XXX section is for the Lanman password hash, the second is for the Windows NT version.

When the password file is created all users have password entries consisting of 32 'X' characters. By default this disallows any access as this user. When a user has a password set, the 'X' characters change to 32 ascii hexadecimal digits (0-9, A-F). These are an ascii representation of the 16 byte hashed value of a user's password.

To set a user to have no password (not recommended), edit the file using vi, and replace the first 11 characters with the ascii text {\texttt{\docbookhyphenatedot{"NO PASSWORD"}}} (minus the quotes).

For example, to clear the password for user bob, his smbpasswd file entry would look like :


\begin{Verbatim}[]

bob:100:NO PASSWORDXXXXXXXXXXXXXXXXXXXXX:XXXXXXXXXXXXXXXXXXXXXXXXXXXXXXXX:
	[U          ]:LCT-00000000:Bob's full name:/bobhome:/bobshell
	
\end{Verbatim}


If you are allowing users to use the smbpasswd command to set their own passwords, you may want to give users NO PASSWORD initially so they do not have to enter a previous password when changing to their new password (not recommended). In order for you to allow this the {\bfseries{smbpasswd}} program must be able to connect to the {\bfseries{smbd}} daemon as that user with no password. Enable this by adding the line :

{\bfseries{null passwords = yes}}

to the [global] section of the smb.conf file (this is why the above scenario is not recommended). Preferably, allocate your users a default password to begin with, so you do not have to enable this on your server.

{\em{Note :}}This file should be protected very carefully. Anyone with access to this file can (with enough knowledge of the protocols) gain access to your SMB server. The file is thus more sensitive than a normal unix {\texttt{\docbookhyphenatefilename{/\dbz{}etc/\dbz{}passwd}}} file.
% ------------------------------------------------------------- 
% 
% PART Debugging and tracing
% 
% ------------------------------------------------------------- 
\part{Debugging and tracing}
\label{id2407269}\hypertarget{id2407269}{}%

% -------------------------------------------------------------
% Chapter Tracing samba system calls 
% ------------------------------------------------------------- 	
\chapter{Tracing samba system calls}
\label{tracing}\hypertarget{tracing}{}%

This file describes how to do a system call trace on Samba to work out what its doing wrong. This is not for the faint of heart, but if you are reading this then you are probably desperate.

Actually its not as bad as the the above makes it sound, just don't expect the output to be very pretty :-)

Ok, down to business. One of the big advantages of unix systems is that they nearly all come with a system trace utility that allows you to monitor all system calls that a program is making. This is extremely using for debugging and also helps when trying to work out why something is slower than you expect. You can use system tracing without any special compilation options.

The system trace utility is called different things on different systems. On Linux systems its called strace. Under SunOS 4 its called trace. Under SVR4 style systems (including solaris) its called truss. Under many BSD systems its called ktrace.

The first thing you should do is read the man page for your native system call tracer. In the discussion below I'll assume its called strace as strace is the only portable system tracer (its available for free for many unix types) and its also got some of the nicest features.

Next, try using strace on some simple commands. For example, {\bfseries{strace ls}} or {\bfseries{strace echo hello}}.

You'll notice that it produces a LOT of output. It is showing you the arguments to every system call that the program makes and the result. Very little happens in a program without a system call so you get lots of output. You'll also find that it produces a lot of "preamble" stuff showing the loading of shared libraries etc. Ignore this (unless its going wrong!)

For example, the only line that really matters in the {\bfseries{strace echo hello}} output is:


\begin{Verbatim}[]

write(1, "hello\n", 6)                  = 6

\end{Verbatim}


all the rest is just setting up to run the program.

Ok, now you're familiar with strace. To use it on Samba you need to strace the running smbd daemon. The way I tend ot use it is to first login from my Windows PC to the Samba server, then use smbstatus to find which process ID that client is attached to, then as root I do {\bfseries{strace -p PID}} to attach to that process. I normally redirect the stderr output from this command to a file for later perusal. For example, if I'm using a csh style shell:

{\bfseries{strace -f -p 3872 \textgreater{}\& strace.out}}

or with a sh style shell:

{\bfseries{strace -f -p 3872 \textgreater{} strace.out 2\textgreater{}\&1}}

Note the "-f" option. This is only available on some systems, and allows you to trace not just the current process, but any children it forks. This is great for finding printing problems caused by the "print command" being wrong.

Once you are attached you then can do whatever it is on the client that is causing problems and you will capture all the system calls that smbd makes.

So how do you interpret the results? Generally I search through the output for strings that I know will appear when the problem happens. For example, if I am having touble with permissions on a file I would search for that files name in the strace output and look at the surrounding lines. Another trick is to match up file descriptor numbers and "follow" what happens to an open file until it is closed.

Beyond this you will have to use your initiative. To give you an idea of what you are looking for here is a piece of strace output that shows that {\texttt{\docbookhyphenatefilename{/\dbz{}dev/\dbz{}null}}} is not world writeable, which causes printing to fail with Samba:


\begin{Verbatim}[]

[pid 28268] open("/dev/null", O_RDWR)   = -1 EACCES (Permission denied)
[pid 28268] open("/dev/null", O_WRONLY) = -1 EACCES (Permission denied)

\end{Verbatim}


The process is trying to first open {\texttt{\docbookhyphenatefilename{/\dbz{}dev/\dbz{}null}}} read-write then read-only. Both fail. This means {\texttt{\docbookhyphenatefilename{/\dbz{}dev/\dbz{}null}}} has incorrect permissions.

% -------------------------------------------------------------
% Chapter Samba Printing Internals 
% ------------------------------------------------------------- 	
\chapter{Samba Printing Internals}
\label{devprinting}\hypertarget{devprinting}{}%

% ------------------------   
% Section 
\section{Abstract}
\label{id2495000}\hypertarget{id2495000}{}%

The purpose of this document is to provide some insight into Samba's printing functionality and also to describe the semantics of certain features of Windows client printing.

% ------------------------   
% Section 
\section{Printing Interface to Various Back ends}
\label{id2464708}\hypertarget{id2464708}{}%

Samba uses a table of function pointers to seven functions. The function prototypes are defined in the {\texttt{\docbookhyphenatedot{printif}}} structure declared in {\texttt{\docbookhyphenatefilename{printing.\dbz{}h}}}.

\begin{itemize}
%--- Item
\item 
retrieve the contents of a print queue


%--- Item
\item 
pause the print queue


%--- Item
\item 
resume a paused print queue


%--- Item
\item 
delete a job from the queue


%--- Item
\item 
pause a job in the print queue


%--- Item
\item 
result a paused print job in the queue


%--- Item
\item 
submit a job to the print queue

\end{itemize}

Currently there are only two printing back end implementations defined.

\begin{itemize}
%--- Item
\item 
a generic set of functions for working with standard UNIX printing subsystems


%--- Item
\item 
a set of CUPS specific functions (this is only enabled if the CUPS libraries were located at compile time).

\end{itemize}

% ------------------------   
% Section 
\section{Print Queue TDB's}
\label{id2489697}\hypertarget{id2489697}{}%

Samba provides periodic caching of the output from the "lpq command" for performance reasons. This cache time is configurable in seconds. Obviously the longer the cache time the less often smbd will be required to exec a copy of lpq. However, the accuracy of the print queue contents displayed to clients will be diminished as well.

The list of currently opened print queue TDB's can be found be examining the list of tdb\_print\_db structures ( see print\_db\_head in printing.c ). A queue TDB is opened using the wrapper function printing.c:get\_print\_db\_byname(). The function ensures that smbd does not open more than MAX\_PRINT\_DBS\_OPEN in an effort to prevent a large print server from exhausting all available file descriptors. If the number of open queue TDB's exceeds the MAX\_PRINT\_DBS\_OPEN limit, smbd falls back to a most recently used algorithm for maintaining a list of open TDB's.

There are two ways in which a a print job can be entered into a print queue's TDB. The first is to submit the job from a Windows client which will insert the job information directly into the TDB. The second method is to have the print job picked up by executing the "lpq command".


\begin{Verbatim}[]

/* included from printing.h */
struct printjob {
	pid_t pid; /* which process launched the job */
	int sysjob; /* the system (lp) job number */
	int fd; /* file descriptor of open file if open */
	time_t starttime; /* when the job started spooling */
	int status; /* the status of this job */
	size_t size; /* the size of the job so far */
	int page_count;	/* then number of pages so far */
	BOOL spooled; /* has it been sent to the spooler yet? */
	BOOL smbjob; /* set if the job is a SMB job */
	fstring filename; /* the filename used to spool the file */
	fstring jobname; /* the job name given to us by the client */
	fstring user; /* the user who started the job */
	fstring queuename; /* service number of printer for this job */
	NT_DEVICEMODE *nt_devmode;
};

\end{Verbatim}


The current manifestation of the printjob structure contains a field for the UNIX job id returned from the "lpq command" and a Windows job ID (32-bit bounded by PRINT\_MAX\_JOBID). When a print job is returned by the "lpq command" that does not match an existing job in the queue's TDB, a 32-bit job ID above the \textless{}*vance doesn't know what word is missing here*\textgreater{} is generating by adding UNIX\_JOB\_START to the id reported by lpq.

In order to match a 32-bit Windows jobid onto a 16-bit lanman print job id, smbd uses an in memory TDB to match the former to a number appropriate for old lanman clients.

When updating a print queue, smbd will perform the following steps ( refer to {\texttt{\docbookhyphenatefilename{print.\dbz{}c:print\_\dbz{}queue\_\dbz{}update(\dbz{})\dbz{}}}} ):

\begin{enumerate}
%--- Item
\item 
Check to see if another smbd is currently in the process of updating the queue contents by checking the pid stored in {\texttt{\docbookhyphenatedot{LOCK/{\ttfamily\itshape{\docbookhyphenatedot{printer\_name}}}}}}. If so, then do not update the TDB.


%--- Item
\item 
Lock the mutex entry in the TDB and store our own pid. Check that this succeeded, else fail.


%--- Item
\item 
Store the updated time stamp for the new cache listing


%--- Item
\item 
Retrieve the queue listing via "lpq command"


%--- Item
\item 

\begin{Verbatim}[]

	foreach job in the queue
     	{
		if the job is a UNIX job, create a new entry;
		if the job has a Windows based jobid, then
		{
			Lookup the record by the jobid;
			if the lookup failed, then
				treat it as a UNIX job;
			else
				update the job status only
		}
	}
\end{Verbatim}



%--- Item
\item 
Delete any jobs in the TDB that are not in the in the lpq listing


%--- Item
\item 
Store the print queue status in the TDB


%--- Item
\item 
update the cache time stamp again

\end{enumerate}

Note that it is the contents of this TDB that is returned to Windows clients and not the actual listing from the "lpq command".

The NT\_DEVICEMODE stored as part of the printjob structure is used to store a pointer to a non-default DeviceMode associated with the print job. The pointer will be non-null when the client included a Device Mode in the OpenPrinterEx() call and subsequently submitted a job for printing on that same handle. If the client did not include a Device Mode in the OpenPrinterEx() request, the nt\_devmode field is NULL and the job has the printer's device mode associated with it by default.

Only non-default Device Mode are stored with print jobs in the print queue TDB. Otherwise, the Device Mode is obtained from the printer object when the client issues a GetJob(level == 2) request.

% ------------------------   
% Section 
\section{ChangeID and Client Caching of Printer Information}
\label{id2408737}\hypertarget{id2408737}{}%

[To be filled in later]

% ------------------------   
% Section 
\section{Windows NT/2K Printer Change Notify}
\label{id2408749}\hypertarget{id2408749}{}%

When working with Windows NT+ clients, it is possible for a print server to use RPC to send asynchronous change notification events to clients for certain printer and print job attributes. This can be useful when the client needs to know that a new job has been added to the queue for a given printer or that the driver for a printer has been changed. Note that this is done entirely orthogonal to cache updates based on a new ChangeID for a printer object.

The basic set of RPC's used to implement change notification are

\begin{itemize}
%--- Item
\item 
RemoteFindFirstPrinterChangeNotifyEx ( RFFPCN )


%--- Item
\item 
RemoteFindNextPrinterChangeNotifyEx ( RFNPCN )


%--- Item
\item 
FindClosePrinterChangeNotify( FCPCN )


%--- Item
\item 
ReplyOpenPrinter


%--- Item
\item 
ReplyClosePrinter


%--- Item
\item 
RouteRefreshPrinterChangeNotify ( RRPCN )

\end{itemize}

One additional RPC is available to a server, but is never used by the Windows spooler service:

\begin{itemize}
%--- Item
\item 
RouteReplyPrinter()

\end{itemize}

The opnum for all of these RPC's are defined in include/rpc\_spoolss.h

Windows NT print servers use a bizarre method of sending print notification event to clients. The process of registering a new change notification handle is as follows. The 'C' is for client and the 'S' is for server. All error conditions have been eliminated.


\begin{Verbatim}[]

C:	Obtain handle to printer or to the printer
	server via the standard OpenPrinterEx() call.
S:	Respond with a valid handle to object

C:	Send a RFFPCN request with the previously obtained
	handle with either (a) set of flags for change events
	to monitor, or (b) a PRINTER_NOTIFY_OPTIONS structure
	containing the event information to monitor.  The windows
	spooler has only been observed to use (b).
S:	The <* another missing word*> opens a new TCP session to the client (thus requiring
	all print clients to be CIFS servers as well) and sends
	a ReplyOpenPrinter() request to the client.
C:	The client responds with a printer handle that can be used to
	send event notification messages.
S:	The server replies success to the RFFPCN request.

C:	The windows spooler follows the RFFPCN with a RFNPCN
	request to fetch the current values of all monitored
	attributes.
S:	The server replies with an array SPOOL_NOTIFY_INFO_DATA
	structures (contained in a SPOOL_NOTIFY_INFO structure).

C:	If the change notification handle is ever released by the
	client via a FCPCN request, the server sends a ReplyClosePrinter()
	request back to the client first.  However a request of this
	nature from the client is often an indication that the previous
	notification event was not marshalled correctly by the server
	or a piece of data was wrong.
S:	The server closes the internal change notification handle
	(POLICY_HND) and does not send any further change notification
	events to the client for that printer or job.

\end{Verbatim}


The current list of notification events supported by Samba can be found by examining the internal tables in srv\_spoolss\_nt.c

\begin{itemize}
%--- Item
\item 
printer\_notify\_table[]


%--- Item
\item 
job\_notify\_table[]

\end{itemize}

When an event occurs that could be monitored, smbd sends a message to itself about the change. The list of events to be transmitted are queued by the smbd process sending the message to prevent an overload of TDB usage and the internal message is sent during smbd's idle loop (refer to printing/notify.c and the functions send\_spoolss\_notify2\_msg() and print\_notify\_send\_messages() ).

The decision of whether or not the change is to be sent to connected clients is made by the routine which actually sends the notification. ( refer to srv\_spoolss\_nt.c:recieve\_notify2\_message() ).

Because it possible to receive a listing of multiple changes for multiple printers, the notification events must be split into categories by the printer name. This makes it possible to group multiple change events to be sent in a single RPC according to the printer handle obtained via a ReplyOpenPrinter().

The actual change notification is performed using the RRPCN request RPC. This packet contains

\begin{itemize}
%--- Item
\item 
the printer handle registered with the client's spooler on which the change occurred


%--- Item
\item 
The change\_low value which was sent as part of the last RFNPCN request from the client


%--- Item
\item 
The SPOOL\_NOTIFY\_INFO container with the event information

\end{itemize}

A {\texttt{\docbookhyphenatedot{SPOOL\_NOTIFY\_INFO}}} contains:

\begin{itemize}
%--- Item
\item 
the version and flags field are predefined and should not be changed


%--- Item
\item 
The count field is the number of entries in the SPOOL\_NOTIFY\_INFO\_DATA array

\end{itemize}

The {\texttt{\docbookhyphenatedot{SPOOL\_NOTIFY\_INFO\_DATA}}} entries contain:

\begin{itemize}
%--- Item
\item 
The type defines whether or not this event is for a printer or a print job


%--- Item
\item 
The field is the flag identifying the event


%--- Item
\item 
the notify\_data union contains the new valuie of the attribute


%--- Item
\item 
The enc\_type defines the size of the structure for marshalling and unmarshalling


%--- Item
\item 
(a) the id must be 0 for a printer event on a printer handle. (b) the id must be the job id for an event on a printer job (c) the id must be the matching number of the printer index used in the response packet to the RFNPCN when using a print server handle for notification. Samba currently uses the snum of the printer for this which can break if the list of services has been modified since the notification handle was registered.


%--- Item
\item 
The size is either (a) the string length in UNICODE for strings, (b) the size in bytes of the security descriptor, or (c) 0 for data values.

\end{itemize}
% ------------------------------------------------------------- 
% 
% PART Appendices
% 
% ------------------------------------------------------------- 
\part{Appendices}
\label{id2407290}\hypertarget{id2407290}{}%

% -------------------------------------------------------------
% Chapter Notes to packagers 
% ------------------------------------------------------------- 	
\chapter{Notes to packagers}
\label{Packaging}\hypertarget{Packaging}{}%

% ------------------------   
% Section 
\section{Versioning}
\label{id2474461}\hypertarget{id2474461}{}%

Please, please update the version number in {\texttt{\docbookhyphenatefilename{source/\dbz{}include/\dbz{}version.\dbz{}h}}} to include the versioning of your package. This makes it easier to distinguish standard samba builds from custom-build samba builds (distributions often patch packages). For example, a good version would be:


\begin{Verbatim}[]

Version 2.999+3.0.alpha21-5 for Debian

\end{Verbatim}


% ------------------------   
% Section 
\section{Modules}
\label{id2483416}\hypertarget{id2483416}{}%

Samba3 has support for building parts of samba as plugins. This makes it possible to, for example, put ldap or mysql support in a separate package, thus making it possible to have a normal samba package not depending on ldap or mysql. To build as much parts of samba as a plugin, run:

The option {\texttt{\docbookhyphenatedot{-\dbz{}-\dbz{}with-\dbz{}shared-\dbz{}modules}}} is maintained to support specific modules such as idmap\_XXX and vfs\_XXX. For example, {\texttt{\docbookhyphenatedot{-\dbz{}-\dbz{}with-\dbz{}shared-\dbz{}modules=idmap\_\dbz{}ad}}}. Use of this parameter to the {\bfseries{configure}} command as not been supported in official releases.


\begin{Verbatim}[]

./configure --with-shared-modules=rpc,vfs,auth,pdb,charset

\end{Verbatim}


\end{document}

