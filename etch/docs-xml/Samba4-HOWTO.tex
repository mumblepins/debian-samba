% ------------------------------------------------------------	
% Autogenerated LaTeX file for books	
% ------------------------------------------------------------	
\ifx\pdfoutput\undefined
\documentclass[,letterpaper,11pt,openright,twoside,]{sambadoc}
\else
\documentclass[pdftex,,letterpaper,11pt,openright,twoside,]{sambadoc}
\fi
\label{Samba-HOWTO}\usepackage{ifthen}
% --------------------------------------------
% Check for PDFLaTeX/LaTeX 
% --------------------------------------------
\newif\ifpdf
\ifx\pdfoutput\undefined
\pdffalse % we are not running PDFLaTeX
\else
\pdfoutput=1 % we are running PDFLaTeX
\pdftrue
\fi
% --------------------------------------------
% Load graphicx package with pdf if needed 
% --------------------------------------------
\ifpdf
\usepackage[pdftex]{graphicx}
\pdfcompresslevel=9
\else
\usepackage{graphicx}
\fi

\makeatletter
% redefine the listoffigures and listoftables so that the name of the chapter
% is printed whenever there are figures or tables from that chapter. encourage
% pagebreak prior to the name of the chapter (discourage orphans).
\let\save@@chapter\@chapter
\let\save@@l@figure\l@figure
\let\the@l@figure@leader\relax
\def\@chapter[#1]#2{\save@@chapter[{#1}]{#2}%
\addtocontents{lof}{\protect\def\the@l@figure@leader{\protect\pagebreak[0]\protect\contentsline{chapter}{\protect\numberline{\thechapter}#1}{}{\thepage}}}%
\addtocontents{lot}{\protect\def\the@l@figure@leader{\protect\pagebreak[0]\protect\contentsline{chapter}{\protect\numberline{\thechapter}#1}{}{\thepage}}}%
}
\renewcommand*\l@figure{\the@l@figure@leader\let\the@l@figure@leader\relax\save@@l@figure}
\let\l@table\l@figure
\makeatother
% ---------------------- 
% Most Common Packages   
% ---------------------- 
\usepackage{latexsym}         
\usepackage{enumerate}         
\usepackage{fancybox}      
\usepackage{float}       
\usepackage{ragged2e}       
\usepackage{fancyvrb}         
\makeatletter\@namedef{FV@fontfamily@default}{\def\FV@FontScanPrep{}\def\FV@FontFamily{}}\makeatother
\fvset{obeytabs=true,tabsize=3}
\usepackage{parskip}         
\usepackage{rotating}         
\usepackage{subfigure}         
\usepackage{tabularx}         
\usepackage{url}         
% --------------------------------------------
% Math support                                
% --------------------------------------------
\usepackage{amsmath,amsthm, amsfonts, amssymb, amsxtra,amsopn}
%\newtheorem{thm}{Theorem}[section]
%\newtheorem{cor}[section]{Corollary}
%\newtheorem{lem}[section]{Lemma}
%\newtheorem{defn}[section]{Definition}
%\newtheorem{prop}[section]{Proposition}
%\newtheorem{ax}{Axiom}
%\newtheorem{theorem}[section]{Theorem}
%\newtheorem{corollary}{Corollary}
%\newtheorem{lemma}{Lemma}
%\newtheorem{proposition}{Proposition}
%\theoremstyle{definition}
%\newtheorem{definition}{Definition}
%\theoremstyle{remark}
%\newtheorem{rem}{Remark}
%\newtheorem*{notation}{Notation}
%\newcommand{\ntt}{\normalfont\ttfamily}
%\newcommand{\thmref}[1]{Theorem~\ref{#1}}
%\newcommand{\secref}[1]{\S\ref{#1}}
%\newcommand{\lemref}[1]{Lemma~\ref{#1}}
 \newcommand{\bysame}{\mbox{\rule{3em}{.4pt}}\,}
 \newcommand{\A}{\mathcal{A}}
 \newcommand{\B}{\mathcal{B}}
 \newcommand{\XcY}{{(X,Y)}}
 \newcommand{\SX}{{S_X}}
 \newcommand{\SY}{{S_Y}}
 \newcommand{\SXY}{{S_{X,Y}}}
 \newcommand{\SXgYy}{{S_{X|Y}(y)}}
 \newcommand{\Cw}[1]{{\hat C_#1(X|Y)}}
 \newcommand{\G}{{G(X|Y)}}
 \newcommand{\PY}{{P_{\mathcal{Y}}}}
 \newcommand{\X}{\mathcal{X}}
 \newcommand{\wt}{\widetilde}
 \newcommand{\wh}{\widehat}
 % --------------------------------------------
 %\DeclareMathOperator{\per}{per}
 \DeclareMathOperator{\cov}{cov}
 \DeclareMathOperator{\non}{non}
 \DeclareMathOperator{\cf}{cf}
 \DeclareMathOperator{\add}{add}
 \DeclareMathOperator{\Cham}{Cham}
 \DeclareMathOperator{\IM}{Im}
 \DeclareMathOperator{\esssup}{ess\,sup}
 \DeclareMathOperator{\meas}{meas}
 \DeclareMathOperator{\seg}{seg}
% --------------------------------------------
% --------------------------------------------
% Load hyperref package with pdf if needed 
% --------------------------------------------
\ifpdf
\usepackage[pdftex,bookmarksnumbered,colorlinks,backref,bookmarks,breaklinks,linktocpage,plainpages=false,hyperfigures,hyperindex,citecolor=black,urlcolor=black,filecolor=black,linkcolor=black,menucolor=red,pagecolor=black]{hyperref}
\else
\usepackage[bookmarksnumbered,colorlinks,backref,bookmarks,breaklinks,linktocpage,plainpages=false,]{hyperref}
\fi
% --------------------------------------------
% ----------------------------------------------
% Define a new LaTeX environment (adminipage)
% ----------------------------------------------
\newenvironment{admminipage}%
{ % this code corresponds to the \begin{adminipage} command
 \begin{Sbox}%
 \begin{minipage}%
} %done
{ % this code corresponds to the \end{adminipage} command
 \end{minipage}
 \end{Sbox}
 \fbox{\TheSbox}
} %done
% ----------------------------------------------
% Define a new LaTeX length (admlength)
% ----------------------------------------------
\newlength{\admlength}
% ----------------------------------------------
% Define a new LaTeX environment (admonition)
% With 2 parameters:
% #1 The file (e.g. note.pdf)
% #2 The caption
% ----------------------------------------------
\newenvironment{admonition}[2] 
{ % this code corresponds to the \begin{admonition} command
 \hspace{0mm}\newline\hspace*\fill\newline
 \noindent
 \setlength{\fboxsep}{5pt}
 \setlength{\admlength}{\linewidth}
 \addtolength{\admlength}{-10\fboxsep}
 \addtolength{\admlength}{-10\fboxrule}
 \admminipage{\admlength}
 {\bfseries \sc\large{#2}} \newline
 \\[1mm]
 \sffamily
 \includegraphics[width=1cm]{#1}
 \addtolength{\admlength}{-1cm}
 \addtolength{\admlength}{-20pt}
 \begin{minipage}[lt]{\admlength}
 \parskip=0.5\baselineskip \advance\parskip by 0pt plus 2pt
} %done
{ % this code corresponds to the \end{admonition} command
 \vspace{5mm} 
 \end{minipage}
 \endadmminipage
 \vspace{.5em}
 \par
}
% --------------------------------------------
% Commands to manage/style/create floats      
% figures, tables, algorithms, examples, eqn  
% --------------------------------------------
 \floatstyle{ruled}
 \restylefloat{figure}
 \floatstyle{ruled}
 \restylefloat{table}
 \floatstyle{ruled}
 \newfloat{program}{ht}{lop}[section]
 \floatstyle{ruled}
 \newfloat{example}{ht}{loe}[section]
 \floatname{example}{Example}
 \floatstyle{ruled}
 \newfloat{dbequation}{ht}{loe}[section]
 \makeatletter\def\toclevel@dbequation{0}\makeatother
 \floatname{dbequation}{Equation}
 \floatstyle{boxed}
 \newfloat{algorithm}{ht}{loa}[section]
 \floatname{algorithm}{Algorithm}
\ifpdf
\DeclareGraphicsExtensions{.pdf,.png,.jpg}
\else
\DeclareGraphicsExtensions{.eps}
\fi
% --------------------------------------------
% $latex.caption.swapskip enabled for $formal.title.placement support
\newlength{\docbooktolatextempskip}
\newcommand{\captionswapskip}{\setlength{\docbooktolatextempskip}{\abovecaptionskip}\setlength{\abovecaptionskip}{\belowcaptionskip}\setlength{\belowcaptionskip}{\docbooktolatextempskip}}
% --------------------------------------------
% Better linebreaks
\newcommand{\docbookhyphenatedot}[1]{{\hyphenchar\font=`\.\relax #1\hyphenchar\font=`\-}}
\newcommand{\docbookhyphenatefilename}[1]{{\hyphenchar\font=`\.\relax #1\hyphenchar\font=`\-}}
\newcommand{\docbookhyphenateurl}[1]{{\hyphenchar\font=`\/\relax #1\hyphenchar\font=`\-}}
\usepackage[english]{babel} 
% Guard against a problem with old package versions.
\makeatletter
\AtBeginDocument{
\DeclareRobustCommand\ref{\@refstar}
\DeclareRobustCommand\pageref{\@pagerefstar}
}
\makeatother
% --------------------------------------------
\makeatletter
\newcommand{\dbz}{\penalty \z@}
\newcommand{\docbooktolatexpipe}{\ensuremath{|}\dbz}
\newskip\docbooktolatexoldparskip
\newcommand{\docbooktolatexnoparskip}{\docbooktolatexoldparskip=\parskip\parskip=0pt plus 1pt}
\newcommand{\docbooktolatexrestoreparskip}{\parskip=\docbooktolatexoldparskip}
\def\cleardoublepage{\clearpage\if@twoside \ifodd\c@page\else\hbox{}\thispagestyle{empty}\newpage\if@twocolumn\hbox{}\newpage\fi\fi\fi}
\usepackage[]{ucs}
\usepackage[utf8]{inputenc}

\ifx\dblatex@chaptersmark\@undefined\def\dblatex@chaptersmark#1{\markboth{\MakeUppercase{#1}}{}}\fi
\let\save@makeschapterhead\@makeschapterhead
\def\dblatex@makeschapterhead#1{\vspace*{-80pt}\save@makeschapterhead{#1}}
\def\@makeschapterhead#1{\dblatex@makeschapterhead{#1}\dblatex@chaptersmark{#1}}

			
\AtBeginDocument{\ifx\refname\@undefined\let\docbooktolatexbibname\bibname\def\docbooktolatexbibnamex{\bibname}\else\let\docbooktolatexbibname\refname\def\docbooktolatexbibnamex{\refname}\fi}
% Facilitate use of \cite with \label
\newcommand{\docbooktolatexbibaux}[2]{%
  \protected@write\@auxout{}{\string\global\string\@namedef{docbooktolatexcite@#1}{#2}}
}
% Provide support for bibliography `subsection' environments with titles
\newenvironment{docbooktolatexbibliography}[3]{
   \begingroup
   \let\save@@chapter\chapter
   \let\save@@section\section
   \let\save@@@mkboth\@mkboth
   \let\save@@bibname\bibname
   \let\save@@refname\refname
   \let\@mkboth\@gobbletwo
   \def\@tempa{#3}
   \def\@tempb{}
   \ifx\@tempa\@tempb
      \let\chapter\@gobbletwo
      \let\section\@gobbletwo
      \let\bibname\relax
   \else
      \let\chapter#2
      \let\section#2
      \let\bibname\@tempa
   \fi
   \let\refname\bibname
   \begin{thebibliography}{#1}
}{
   \end{thebibliography}
   \let\chapter\save@@chapter
   \let\section\save@@section
   \let\@mkboth\save@@@mkboth
   \let\bibname\save@@bibname
   \let\refname\save@@refname
   \endgroup
}

		
			
%\usepackage{cite}
%\renewcommand\citeleft{(}  % parentheses around list
%\renewcommand\citeright{)} % parentheses around list
\newcommand{\docbooktolatexcite}[2]{%
  \@ifundefined{docbooktolatexcite@#1}%
  {\cite{#1}}%
  {\def\@docbooktolatextemp{#2}\ifx\@docbooktolatextemp\@empty%
   \cite{\@nameuse{docbooktolatexcite@#1}}%
   \else\cite[#2]{\@nameuse{docbooktolatexcite@#1}}%
   \fi%
  }%
}
\newcommand{\docbooktolatexbackcite}[1]{%
  \ifx\Hy@backout\@undefined\else%
    \@ifundefined{docbooktolatexcite@#1}{%
      % emit warning?
    }{%
      \ifBR@verbose%
        \PackageInfo{backref}{back cite \string`#1\string' as \string`\@nameuse{docbooktolatexcite@#1}\string'}%
      \fi%
      \Hy@backout{\@nameuse{docbooktolatexcite@#1}}%
    }%
  \fi%
}

		
			
% --------------------------------------------
% A way to honour <footnoteref>s
% Blame j-devenish (at) users.sourceforge.net
% In any other LaTeX context, this would probably go into a style file.
\newcommand{\docbooktolatexusefootnoteref}[1]{\@ifundefined{@fn@label@#1}%
  {\hbox{\@textsuperscript{\normalfont ?}}%
    \@latex@warning{Footnote label `#1' was not defined}}%
  {\@nameuse{@fn@label@#1}}}
\newcommand{\docbooktolatexmakefootnoteref}[1]{%
  \protected@write\@auxout{}%
    {\global\string\@namedef{@fn@label@#1}{\@makefnmark}}%
  \@namedef{@fn@label@#1}{\hbox{\@textsuperscript{\normalfont ?}}}%
  }

		
			
% index labeling helper
\newif\ifdocbooktolatexprintindex\docbooktolatexprintindextrue
\let\dbtolatex@@theindex\theindex
\let\dbtolatex@@endtheindex\endtheindex
\def\theindex{\relax}
\def\endtheindex{\relax}
\newenvironment{dbtolatexindex}[1]
   {
\if@openright\cleardoublepage\else\clearpage\fi
\let\dbtolatex@@indexname\indexname
\def\dbtolatex@indexlabel{%
 \ifnum \c@secnumdepth >\m@ne \refstepcounter{chapter}\fi%
 \label{#1}\hypertarget{#1}{\dbtolatex@@indexname}%
 \global\docbooktolatexprintindexfalse}
\def\indexname{\ifdocbooktolatexprintindex\dbtolatex@indexlabel\else\dbtolatex@@indexname\fi}
\dbtolatex@@theindex
   }
   {
\dbtolatex@@endtheindex\let\indexname\dbtolatex@@indexname
   }

\newlength\saveparskip \newlength\saveparindent
\newlength\tempparskip \newlength\tempparindent

		
\def\docbooktolatexgobble{\expandafter\@gobble}
% Prevent multiple openings of the same aux file
% (happens when backref is used with multiple bibliography environments)
\ifx\AfterBeginDocument\undefined\let\AfterBeginDocument\AtBeginDocument\fi
\AfterBeginDocument{
   \let\latex@@starttoc\@starttoc
   \def\@starttoc#1{%
      \@ifundefined{docbooktolatex@aux#1}{%
         \global\@namedef{docbooktolatex@aux#1}{}%
         \latex@@starttoc{#1}%
      }{}
   }
}
% --------------------------------------------
% Hacks for honouring row/entry/@align
% (\hspace not effective when in paragraph mode)
% Naming convention for these macros is:
% 'docbooktolatex' 'align' {alignment-type} {position-within-entry}
% where r = right, l = left, c = centre
\newcommand{\docbooktolatex@align}[2]{\protect\ifvmode#1\else\ifx\LT@@tabarray\@undefined#2\else#1\fi\fi}
\newcommand{\docbooktolatexalignll}{\docbooktolatex@align{\raggedright}{}}
\newcommand{\docbooktolatexalignlr}{\docbooktolatex@align{}{\hspace*\fill}}
\newcommand{\docbooktolatexaligncl}{\docbooktolatex@align{\centering}{\hfill}}
\newcommand{\docbooktolatexaligncr}{\docbooktolatex@align{}{\hspace*\fill}}
\newcommand{\docbooktolatexalignrl}{\protect\ifvmode\raggedleft\else\hfill\fi}
\newcommand{\docbooktolatexalignrr}{}
\ifx\captionswapskip\@undefined\newcommand{\captionswapskip}{}\fi
\makeatother
\title{\bfseries The Official Samba-4 HOWTO}
\author{Jelmer R. Vernooij}
% --------------------------------------------
\makeindex
\makeglossary
% --------------------------------------------

\setcounter{tocdepth}{4}

\setcounter{secnumdepth}{4}
\begin{document}
{\maketitle\pagestyle{plain}
\thispagestyle{empty}}
\setcounter{page}{5} % -------------------------------------------------------------
% Preface 
% -------------------------------------------------------------
\chapter*{Attribution}%
\label{id2406721}\hypertarget{id2406721}{}%

% -------------------------------------------------------------
% Preface 
% -------------------------------------------------------------
\chapter*{Attribution}%
\label{id2438514}\hypertarget{id2438514}{}%

\cleardoublepage 
\docbooktolatexnoparskip

\makeatletter
\def\dbtolatex@contentsid{id2408201}
\let\dbtolatex@@contentsname\contentsname
\newif\ifdocbooktolatexcontentsname\docbooktolatexcontentsnametrue
\def\dbtolatex@contentslabel{%
 \label{\dbtolatex@contentsid}\hypertarget{\dbtolatex@contentsid}{\dbtolatex@@contentsname}%
 \global\docbooktolatexcontentsnamefalse}
\def\contentsname{\ifdocbooktolatexcontentsname\dbtolatex@contentslabel\else\dbtolatex@@contentsname\fi}
\tableofcontents
\let\contentsname\dbtolatex@@contentsname
\Hy@writebookmark{}{\dbtolatex@@contentsname}{\dbtolatex@contentsid}{0}{toc}%
\makeatother
				\docbooktolatexrestoreparskip
\cleardoublepage % ------------------------------------------------------------- 
% 
% PART Introduction to SMB networks
% 
% ------------------------------------------------------------- 
\part{Introduction to SMB networks}
\label{id2408206}\hypertarget{id2408206}{}%

Not everybody is already familiar with SMB networks and the jargon used in such networks. This part describes the evolution of SMB, how the protocol is structured and what Samba can be used for.

NT admins, already familiar with SMB networks, are probably only interested in the chapter about Samba itself.

% -------------------------------------------------------------
% Chapter The SMB/CIFS protocol 
% ------------------------------------------------------------- 	
\chapter{The SMB/CIFS protocol}
\label{protocol}\hypertarget{protocol}{}%

"SMB" (also known as "CIFS") is a file-sharing protocol that has been used since the mid-eighties. Most people know SMB as the protocol behind the "Network Neighbourhood" and remote printing in Windows.

Several parts of the protocol are not discussed in this chapter, such as mailslots, browsing and dfs, to prevent it from getting too complex. CIFS internals are documented in detail in \docbooktolatexcite{Hertel, 2003}{}.

% ------------------------   
% Section 
\section{History}
\label{id2408693}\hypertarget{id2408693}{}%
\subsection{Invention by IBM}
\label{id2408700}\hypertarget{id2408700}{}%

SMB is not very old, but it has a long history of modifications and extensions. The original protocol was meant to run over ``NetBIOS'', which was the name of the DOS interface to a very simple LAN system developed by IBM. NetBIOS was developed because SNA, IBM's other main protocol at the time, was much too advanced for use in DOS.

The NetBIOS API in these days (early eighties) was nothing more then the interface to a very simple link-layer protocol over which several protocols, including SMB, were used. It could do reads and writes to services on remote hosts, which were identified by case-insensitive names, and discover all available hosts and services.

Dr. Barry Feigenbau, an IBM employee, invented the core of the original SMB protocol, which he initially named after himself: ``BAF''. He later changed the name to be ``SMB'' (for ``Server Message Block''). Every packet in the protocol starts with a byte \$0xFF\$ and these three letters.

IBM, Microsoft, 3Com and Intel made up the rest of the initial protocol together. The commands the protocol supported at this stage were basically a mirror of the DOS File IO API calls, which meant the protocol wasn't very efficient. The protocol also lacked authentication support. Everybody on the network could do reads and writes, which meant this protocol wasn't very suitable for large enterprises.

NetBIOS is an API that has had various implementations; there is NetBIOS over TCP/IP (NBT), NetBIOS over IPX, NetBIOS over SNA and even NetBIOS over DECNEt. Mostly used these days is NetBIOS over TCP (NBT).

This is also were things are starting to get hairy. Since NetBIOS identifies hosts by their name, NetBIOS clients had to start doing IP broadcasts to figure out the IP of the host they had to connect to. Several schemes were introduced to do name lookups crossing subnet boundaries, using name servers, etc. We're basically emulating a NetBIOS LAN in order to be able to run SMB.

Doing NetBIOS over IP is not very sane, however, the NBT implementation itself in Windows isn't very nice either. It has horrible limits, special exceptions, several broken schemes for looking up names (including two kinds of name servers). NetBIOS and NetBIOS over TCP/IP are described in RFC1001 and RFC1002.
\subsection{The various incarnations of SMB}
\label{id2408773}\hypertarget{id2408773}{}%

Over the years, several usage models for SMB have been developed. While SMB originally started out as a file sharing protocol, it was later extended to include support for network management and other network services as well.

One of the reasons for the various "upgrades" of the SMB protocol is the fact that networks have become larger and larger and with them the need for privilege separation and scalability has increased.
\subsubsection{DOS}
\label{id2483569}\hypertarget{id2483569}{}%

The original model in which SMB was used was as a simple file-sharing service in a NetBIOS-environment.

File sharing worked basically by specifying a list of directories that had to be shared and what name they had to be shared under ("shares"). Eventually, one could password-protect a share. At most one password per share could be set.
\subsubsection{Windows For Workgroups}
\label{id2483590}\hypertarget{id2483590}{}%

After the ``CORE'' dialect, IBM and Microsoft implemented a new dialect known as ``LANMAN''. This dialect was used by Windows for Workgroups, OS/2 and Windows 9x which all know it under a different name. A 'virtual' file system was also added, which was used for doing remote function calls (RAP, for ``Remote Administration Protocol'').

Computers are grouped into "Workgroups" in this model. Everybody is equal to the others and there is no central point of control.
\subsubsection{Windows NT}
\label{id2483613}\hypertarget{id2483613}{}%

For Windows NT, yet another dialect was added, named 'NT'. The NT dialect had it's own set of file I/O functions (similar to the NT File I/O API) and it had support for yet another way of doing remote function calls: DCE/RPC. RPC's are used for DCOM and several of the subsystems in NT that can be accessed remotely (registry, printing, user management, logging on, etc).

Windows NT works with a new concept for grouping computers called ``domains''\label{id2483633}\begingroup\catcode`\#=12\footnote{In the protocol, domains are actually an upgraded version of a workgroup}\endgroup\docbooktolatexmakefootnoteref{id2483633}. Each computer is member of exactly one domain. There are several roles a computer can have in the domain: PDC (primary domain controller, the "manager" of the domain, that coordinates all authentication and authorization), BDC (Backup domain controller, in case the PDC goes down) or just a regular domain member. The PDC decides who is to be a member of the domain.
\subsubsection{Windows 2000}
\label{id2483647}\hypertarget{id2483647}{}%

In Windows NT 5 (marketing name: Windows 2000), NetBIOS-less SMB was introduced. This means SMB is used directly over TCP port 445 instead of via NetBIOS over TCP/IP. DNS is used for looking up machine names.

Windows 2000 was also the first operating system from Microsoft that had support for Active Directory. Active Directory is very similar to the "domain" concept used by NT4, though it is implemented differently (using modified open protocols), and has some additional features (one of the most important ones being decentralized).

Active directory no longer uses a central point of authority and there are fewer limits to the size of a domain. Several DC's can exist, so there is no longer a single point of failure as well as better scalability.
\subsection{Samba versions and their support for the SMB models}
\label{id2483680}\hypertarget{id2483680}{}%
\subsubsection{Samba 2.2}
\label{id2483686}\hypertarget{id2483686}{}%

Full CORE and Workgroup support. Somewhat basic NT4-style support.
\subsubsection{Samba 3.0}
\label{id2483698}\hypertarget{id2483698}{}%

Full CORE and Workgroup support. Almost complete NT4-style support.
\subsubsection{Samba 3.2}
\label{id2406536}\hypertarget{id2406536}{}%

FIXME
\subsubsection{Samba 4.0}
\label{id2406548}\hypertarget{id2406548}{}%

Full CORE, Workgroup, NT4 and ADS support.

% -------------------------------------------------------------
% Chapter Samba 
% ------------------------------------------------------------- 	
\chapter{Samba}
\label{samba}\hypertarget{samba}{}%

Samba is an Open Source application that provides an SMB implementation for POSIX-compatible operating systems.

Providing support for a Windows-based file-sharing protocol on POSIX servers sometimes requires mapping Windows semantics to POSIX semantics.

Samba has always been catching up with Microsoft. Here is a list of Samba releases and what they support (or will support).

% ------------------------   
% Section 
\section{Samba 2.2}
\label{id2408671}\hypertarget{id2408671}{}%

Full CORE and Workgroup support. Somewhat basic NT4-style support.

% ------------------------   
% Section 
\section{Samba 3.0}
\label{id2483530}\hypertarget{id2483530}{}%

Full CORE and Workgroup support. Almost complete NT4-style support.

% ------------------------   
% Section 
\section{Samba 3.2}
\label{id2483542}\hypertarget{id2483542}{}%

FIXME

% ------------------------   
% Section 
\section{Samba 4.0}
\label{id2483553}\hypertarget{id2483553}{}%

Full CORE, Workgroup, NT4 and ADS support.
% ------------------------------------------------------------- 
% 
% PART Samba 4 Basics
% 
% ------------------------------------------------------------- 
\part{Samba 4 Basics}
\label{id2408487}\hypertarget{id2408487}{}%

FIXME

% -------------------------------------------------------------
% Chapter Configuration 
% ------------------------------------------------------------- 	
\chapter{Configuration}
\label{config}\hypertarget{config}{}%

Chapter describing Samba's configuration in detail.

% -------------------------------------------------------------
% Chapter chapter 
% ------------------------------------------------------------- 	
\chapter{chapter}
\label{ldb}\hypertarget{ldb}{}%
% ------------------------------------------------------------- 
% 
% PART Network model
% 
% ------------------------------------------------------------- 
\part{Network model}
\label{id2408505}\hypertarget{id2408505}{}%

A SMB server is used in a certain network type (workgroup, domain, active directory) with a specific role (stand-alone, primary domain controller, backup domain controller, etc) in that network. Samba has to be configured according to that role. Each of the chapters in this part describes how to configure Samba for a specific role.
% ------------------------------------------------------------- 
% 
% PART Network Model: Share-based security
% 
% ------------------------------------------------------------- 
\part{Network Model: Share-based security}
\label{id2408521}\hypertarget{id2408521}{}%

% -------------------------------------------------------------
% Chapter Share-based stand-alone server 
% ------------------------------------------------------------- 	
\chapter{Share-based stand-alone server}
\label{security=share}\hypertarget{security=share}{}%

One of the simplest
% ------------------------------------------------------------- 
% 
% PART Network Model: User-based security
% 
% ------------------------------------------------------------- 
\part{Network Model: User-based security}
\label{id2408532}\hypertarget{id2408532}{}%

% -------------------------------------------------------------
% Chapter User-based Stand-alone server 
% ------------------------------------------------------------- 	
\chapter{User-based Stand-alone server}
\label{security=user}\hypertarget{security=user}{}%


% ------------------------------------------------------------- 
% 
% PART Network Model: NT4-style domains
% 
% ------------------------------------------------------------- 
\part{Network Model: NT4-style domains}
\label{domain}\hypertarget{domain}{}%

% -------------------------------------------------------------
% Chapter chapter 
% ------------------------------------------------------------- 	
\chapter{chapter}
\label{domain-pdc}\hypertarget{domain-pdc}{}%

% -------------------------------------------------------------
% Chapter Backup Domain Controller 
% ------------------------------------------------------------- 	
\chapter{Backup Domain Controller}
\label{bdc}\hypertarget{bdc}{}%

% -------------------------------------------------------------
% Chapter chapter 
% ------------------------------------------------------------- 	
\chapter{chapter}
\label{domain-member}\hypertarget{domain-member}{}%
% ------------------------------------------------------------- 
% 
% PART Network Model - Active Directory
% 
% ------------------------------------------------------------- 
\part{Network Model - Active Directory}
\label{ad}\hypertarget{ad}{}%

% -------------------------------------------------------------
% Chapter Active Directory - Domain Controller 
% ------------------------------------------------------------- 	
\chapter{Active Directory - Domain Controller}
\label{ad-dc}\hypertarget{ad-dc}{}%

% -------------------------------------------------------------
% Chapter Active Directory - Member 
% ------------------------------------------------------------- 	
\chapter{Active Directory - Member}
\label{ad-member}\hypertarget{ad-member}{}%
% ------------------------------------------------------------- 
% 
% PART Common services
% 
% ------------------------------------------------------------- 
\part{Common services}
\label{id2408582}\hypertarget{id2408582}{}%

Independent of the role the server has, it can provide services. Examples of services are printing, user management, authentication, etc.

% -------------------------------------------------------------
% Chapter chapter 
% ------------------------------------------------------------- 	
\chapter{chapter}
\label{shares}\hypertarget{shares}{}%

% -------------------------------------------------------------
% Chapter Printing 
% ------------------------------------------------------------- 	
\chapter{Printing}
\label{printing}\hypertarget{printing}{}%

% ------------------------   
% Section 
\section{Unix printing types}
\label{unix}\hypertarget{unix}{}%

% -------------------------------------------------------------
% Chapter Authentication 
% ------------------------------------------------------------- 	
\chapter{Authentication}
\label{authentication}\hypertarget{authentication}{}%

% -------------------------------------------------------------
% Chapter Registry 
% ------------------------------------------------------------- 	
\chapter{Registry}
\label{registry}\hypertarget{registry}{}%
% ------------------------------------------------------------- 
% 
% PART Clients
% 
% ------------------------------------------------------------- 
\part{Clients}
\label{id2408613}\hypertarget{id2408613}{}%

Samba includes a number of different clients for accessing SMB file, print and other services. Pretty much every service can be access using a command-line utility, though there are several GUI clients available as well.

% -------------------------------------------------------------
% Chapter smbclient 
% ------------------------------------------------------------- 	
\chapter{smbclient}
\label{smbclient}\hypertarget{smbclient}{}%

% -------------------------------------------------------------
% Chapter CIFSFS 
% ------------------------------------------------------------- 	
\chapter{CIFSFS}
\label{cifsfs}\hypertarget{cifsfs}{}%

% -------------------------------------------------------------
% Chapter chapter 
% ------------------------------------------------------------- 	
\chapter{chapter}
\label{gui-clients}\hypertarget{gui-clients}{}%
% ------------------------------------------------------------- 
% 
% PART Appendices
% 
% ------------------------------------------------------------- 
\part{Appendices}
\label{id2408640}\hypertarget{id2408640}{}%

% -------------------------------------------------------------
% Chapter Compiling 
% ------------------------------------------------------------- 	
\chapter{Compiling}
\label{compiling}\hypertarget{compiling}{}%

% ------------------------   
% Section 
\section{Downloading the source - Tar Ball}
\label{id2455874}\hypertarget{id2455874}{}%

% ------------------------   
% Section 
\section{Downloading the source - Subversion}
\label{id2455882}\hypertarget{id2455882}{}%

% ------------------------   
% Section 
\section{Compiling the source}
\label{id2483446}\hypertarget{id2483446}{}%

% ------------------------   
% Section 
\section{Installing}
\label{id2483454}\hypertarget{id2483454}{}%
\docbooktolatexnoparskip
\listoffigures
\listoftables
\docbooktolatexrestoreparskip
\include{gpl}\chaptermark{Subject index} \setlength\saveparskip\parskip
\setlength\saveparindent\parindent
\begin{dbtolatexindex}{id2408661}
\setlength\tempparskip\parskip \setlength\tempparindent\parindent
\parskip\saveparskip \parindent\saveparindent
\noindent \indexspace
\parskip\tempparskip
\parindent\tempparindent
\makeatletter\@input@{\jobname.ind}\makeatother
\end{dbtolatexindex}

\end{document}

